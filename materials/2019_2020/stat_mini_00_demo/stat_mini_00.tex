\documentclass[12pt]{article}

\usepackage{tikz} % картинки в tikz
\usepackage{microtype} % свешивание пунктуации

\usepackage{array} % для столбцов фиксированной ширины

\usepackage{indentfirst} % отступ в первом параграфе

\usepackage{sectsty} % для центрирования названий частей
\allsectionsfont{\centering}

\usepackage{amsmath} % куча стандартных математических плюшек

\usepackage{comment}
\usepackage{amsfonts}


\usepackage[colorlinks=true, linkcolor=blue]{hyperref}

\usepackage[top=2cm, left=1cm, right=1cm, bottom=2cm]{geometry} % размер текста на странице

\usepackage{lastpage} % чтобы узнать номер последней страницы

\usepackage{enumitem} % дополнительные плюшки для списков
%  например \begin{enumerate}[resume] позволяет продолжить нумерацию в новом списке
\usepackage{caption}

\usepackage{hyperref} % гиперссылки

\usepackage{multicol} % текст в несколько столбцов


\usepackage{fancyhdr} % весёлые колонтитулы
\pagestyle{fancy}
\lhead{Теория вероятностей и математическая статистика-ВШЭ}
\chead{}
\rhead{Миниконтрольная демо.}
\lfoot{2020-03-23}
\cfoot{}
\rfoot{}
\renewcommand{\headrulewidth}{0.4pt}
\renewcommand{\footrulewidth}{0.4pt}



\usepackage{todonotes} % для вставки в документ заметок о том, что осталось сделать
% \todo{Здесь надо коэффициенты исправить}
% \missingfigure{Здесь будет Последний день Помпеи}
% \listoftodos --- печатает все поставленные \todo'шки


% более красивые таблицы
\usepackage{booktabs}
% заповеди из докупентации:
% 1. Не используйте вертикальные линни
% 2. Не используйте двойные линии
% 3. Единицы измерения - в шапку таблицы
% 4. Не сокращайте .1 вместо 0.1
% 5. Повторяющееся значение повторяйте, а не говорите "то же"


\usepackage{fontspec}
\usepackage{polyglossia}

\setmainlanguage{russian}
\setotherlanguages{english}

% download "Linux Libertine" fonts:
% http://www.linuxlibertine.org/index.php?id=91&L=1
\setmainfont{Linux Libertine O} % or Helvetica, Arial, Cambria
% why do we need \newfontfamily:
% http://tex.stackexchange.com/questions/91507/
\newfontfamily{\cyrillicfonttt}{Linux Libertine O}

\AddEnumerateCounter{\asbuk}{\russian@alph}{щ} % для списков с русскими буквами
\setlist[enumerate, 2]{label=\asbuk*),ref=\asbuk*}

%% эконометрические сокращения
\DeclareMathOperator{\Cov}{Cov}
\DeclareMathOperator{\Corr}{Corr}
\DeclareMathOperator{\Var}{Var}
\DeclareMathOperator{\E}{E}
\def \hb{\hat{\beta}}
\def \hs{\hat{\sigma}}
\def \htheta{\hat{\theta}}
\def \s{\sigma}
\def \hy{\hat{y}}
\def \hY{\hat{Y}}
\def \v1{\vec{1}}
\def \e{\varepsilon}
\def \he{\hat{\e}}
\def \z{z}
\def \hVar{\widehat{\Var}}
\def \hCorr{\widehat{\Corr}}
\def \hCov{\widehat{\Cov}}
\def \cN{\mathcal{N}}
\def \P{\mathbb{P}}


\begin{document}

Единственная и Несмущённая Оценка в опасности! 
И только добры молодцы и красны девицы второго курса могут спасти её!
Для спасения Оценки всем студентам вместе нужно совершить аж 16 подвигов. 
Номер требуемого подвига для каждого определяется индивидуально, 
и равен единичке плюс остаток от деления 
\href{https://docs.google.com/spreadsheets/d/e/2PACX-1vT1hkcHBE7txa4maSh9BA8VOvQyZQTOIjsuMukOYr2MZlZkoMpNupBAyyRlZpugVOYXjKpgY2fI1JAc/pubhtml}{\texttt{id\_for\_online}} на 16.

Например, для \verb|id_for_online = 305| номер требуемого подвига равен $N = 1 + (305 \mod 16) = 1 + 1 = 2$.

Полный список подвигов:

\begin{enumerate}
    \item Дайте определение нормально распределённой случайной величины.
      Укажите её диапазон возможных значений, функцию плотности, ожидание, дисперсию.
      Нарисуйте функцию плотности.
    \item С помощью нормальных случайных величин дайте определение случайной величины, имеющей хи-квадрат распределение. 
      Для хи-квадрат распределённой случайной величины укажите диапазон возможных значений,
      математическое ожидание и дисперсию.
      Нарисуйте функцию плотности при разных степенях свободы.
    \item С помощью нормальных случайных величин дайте определение случайной величины, имеющей распределения Стьюдента.
      Для случайной величины, распределённой по Стьюденту, укажите диапазон возможных значений.
      Нарисуйте функцию плотности распределения Стьюдента при разных степенях свободы
      на фоне нормальной стандартной функции плотности.
    \item С помощью нормальных случайных величин дайте определение случайной величины, имеющей распределение Фишера. 
      Для случайной величины, распределённой по Фишеру, укажите диапазон возможных значений.
      Нарисуйте возможную функцию плотности.
  \end{enumerate}
  
  Для следующего блока подвигов предполагается, что
  имеется случайная выборка $X_1$, $X_2$, \ldots, $X_n$ из распределения
  с функцией плотности $f(x, \theta)$, зависящей от от параметра $\theta$.
  
  \begin{enumerate}[resume]
      \item Дайте определение выборочного среднего и выборочной дисперсии;
      \item Дайте определение выборочного начального и выборочного центрального момента порядка $k$;
      \item Дайте определение выборочной функции распределения;
      \item Выпишите формулу несмещённой оценки дисперсии. 
      \item Дайте определение несмещённой оценки $\hat \theta$ параметра $\theta$;
      \item Дайте определение состоятельной последовательности оценок $\hat \theta_n$; 
      Укажите условия на $\E(\hat\theta_n)$ и $\Var(\hat\theta_n)$, достаточные для состоятельности.
      \item Дайте определение эффективности оценки $\hat \theta$ среди множества оценок $\hat \Theta$;  
      \item Сформулируйте неравенство Крамера–Рао для несмещённых оценок;
      \item Дайте определение функции правдоподобия и логарифмической функция правдоподобия;
      \item Дайте определение информации Фишера о параметре $\theta$, содержащейся в одном наблюдении;
      \item Дайте определение оценки метода моментов параметра $\theta$ с использованием первого момента,
      если $\E(X_i)=g(\theta)$ и существует обратная функция $g^{-1}$;
    \item Дайте определение оценки метода максимального правдоподобия параметра $\theta$;
  \end{enumerate}
  

\end{document}
