
\begin{question}
В школе три выпускных класса. В ``А'' классе 50\% мальчиков, в ``Б''
классе — 70\% мальчиков, и в ``В'' классе — 80\%. Я выбираю один
класс равновероятно, а затем одного учащегося из этого класса, также
равновероятно. Вероятность того, что окажется выбран мальчик равна
\begin{answerlist}
  \item \(2/3\)
  \item \(0.75\)
  \item \(0.5\)
  \item \(0.7\)
  \item \(0.6\)
\end{answerlist}
\end{question}

\begin{solution}
\begin{answerlist}
  \item Good answer :)
  \item Bad answer :(
  \item Bad answer :(
  \item Bad answer :(
  \item Bad answer :(
\end{answerlist}
\end{solution}

