
\begin{question}
Вася считает, что контрольные по макроэкономике и статистике нравятся
студентам с одинаковой вероятностью. Чтобы проверить эту гипотезу, он
опросил по 100 случайных однокурсников после каждой контрольной и
выяснил, что макроэкономика понравилась 30 студентам, а статистика ---
50. При расчётах Вася получил P-значение равное 0.0038. Это означает,
что гипотеза
\begin{answerlist}
  \item отвергается на любом возможном уровне значимости
  \item не отвергается на любом возможном уровне значимости
  \item отвергается на уровне значимости 1\%
  \item отвергается на уровне значимости 1\%, но не отвергается на 5\%
  \item отвергается на уровне значимости 5\%, но не отвергается на 1\%
\end{answerlist}
\end{question}

\begin{solution}
\begin{answerlist}
  \item Bad answer :(
  \item Bad answer :(
  \item Good answer :)
  \item Bad answer :(
  \item Bad answer :(
\end{answerlist}
\end{solution}

