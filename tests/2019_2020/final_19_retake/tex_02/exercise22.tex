
\begin{question}
Рассмотрим алгоритм Метрополиса-Гастингса для получения выборки
параметра с апостериорной плотностью пропорциональной \(t^2\).
Предлагаемый переход из \(a\) в \(b\) задаётся правилом, \(b = a + Z\),
где \(Z \sim \cN(0;4)\). Вероятность одобрения перехода из точки \(0.5\)
в точку \(0.3\) равна
\begin{answerlist}
  \item \(0.5\)
  \item \(0.36\)
  \item \(1\)
  \item \(0.64\)
  \item \(0.6\)
\end{answerlist}
\end{question}

\begin{solution}
\begin{answerlist}
  \item Bad answer :(
  \item Good answer :)
  \item Bad answer :(
  \item Bad answer :(
  \item Bad answer :(
\end{answerlist}
\end{solution}

