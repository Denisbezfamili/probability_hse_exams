\documentclass[12pt]{article}

\usepackage{tikz} % картинки в tikz
\usepackage{microtype} % свешивание пунктуации
\usepackage{array} % для столбцов фиксированной ширины
\usepackage{comment} % для комментирования целых окружений
\usepackage{indentfirst} % отступ в первом параграфе

\usepackage{sectsty} % для центрирования названий частей
\allsectionsfont{\centering}

\usepackage{amsmath, amssymb, amsthm, amsfonts} % куча стандартных математических плюшек

\usepackage[top=2cm, left=1cm, right=1cm, bottom=2cm]{geometry} % размер текста на странице
\usepackage{lastpage} % чтобы узнать номер последней страницы
 
\usepackage{enumitem} % дополнительные плюшки для списков
%  например \begin{enumerate}[resume] позволяет продолжить нумерацию в новом списке

\usepackage{caption} % подписи к рисункам
\usepackage{hyperref} % гиперссылки
\usepackage{multicol} % текст в несколько столбцов


\usepackage{fancyhdr} % весёлые колонтитулы
\pagestyle{fancy}
\lhead{Теория вероятностей и математическая статистика, ВШЭ}
\chead{}
\rhead{2019-09-20}
\lfoot{Вариант $\mu$}
\cfoot{Паниковать запрещается!}
% \rfoot{Тест}
\renewcommand{\headrulewidth}{0.4pt}
\renewcommand{\footrulewidth}{0.4pt}

\usepackage{ifthen} % для написания условий

\usepackage{todonotes} % для вставки в документ заметок о том, что осталось сделать
% \todo{Здесь надо коэффициенты исправить}
% \missingfigure{Здесь будет Последний день Помпеи}
% \listoftodos --- печатает все поставленные \todo'шки


% более красивые таблицы
\usepackage{booktabs}
% заповеди из докупентации:
% 1. Не используйте вертикальные линни
% 2. Не используйте двойные линии
% 3. Единицы измерения - в шапку таблицы
% 4. Не сокращайте .1 вместо 0.1
% 5. Повторяющееся значение повторяйте, а не говорите "то же"


\usepackage{fontspec}
\usepackage{polyglossia}

\setmainlanguage{russian}
\setotherlanguages{english}

% download "Linux Libertine" fonts:
% http://www.linuxlibertine.org/index.php?id=91&L=1
\setmainfont{Linux Libertine O} % or Helvetica, Arial, Cambria
% why do we need \newfontfamily:
% http://tex.stackexchange.com/questions/91507/
\newfontfamily{\cyrillicfonttt}{Linux Libertine O}

\AddEnumerateCounter{\asbuk}{\russian@alph}{щ} % для списков с русскими буквами
\setlist[enumerate, 2]{label=\asbuk*),ref=\asbuk*}

%% эконометрические сокращения
\DeclareMathOperator{\Cov}{Cov}
\DeclareMathOperator{\Corr}{Corr}
\DeclareMathOperator{\Var}{Var}
\DeclareMathOperator{\E}{E}
\def \hb{\hat{\beta}}
\def \hs{\hat{\sigma}}
\def \htheta{\hat{\theta}}
\def \s{\sigma}
\def \hy{\hat{y}}
\def \hY{\hat{Y}}
\def \v1{\vec{1}}
\def \e{\varepsilon}
\def \he{\hat{\e}}
\def \z{z}
\def \hVar{\widehat{\Var}}
\def \hCorr{\widehat{\Corr}}
\def \hCov{\widehat{\Cov}}
\def \cN{\mathcal{N}}
\def \P{\mathbb{P}}


\def \putyourname{\fbox{
    \begin{minipage}{42em}
      Фамилия, имя, номер группы:\vspace*{3ex}\par
      \noindent\dotfill\vspace{2mm}
    \end{minipage}
  }
}

\def \checktable{

	\vspace{5pt}
	Табличка для проверяющих работу:

\vspace{5pt}

	\begin{tabular}{|m{2cm}|m{1cm}|m{1cm}|m{1cm}|m{1cm}|m{1cm}|m{2cm}|}
\toprule
		Тест & 1 &  2 & 3 & 4 & 5 & Итого \\
\midrule
		&  &  & & & & \\
		&  &  & & & & \\
 \bottomrule
\end{tabular}
}



\def \testtable{

	\vspace{5pt}
	Внесите сюда ответы на тест:

\vspace{5pt}

\begin{tabular}{|m{2cm}|m{0.6cm}|m{0.6cm}|m{0.6cm}|m{0.6cm}|m{0.6cm}|m{0.6cm}|m{0.6cm}|m{0.6cm}|m{0.6cm}|m{0.6cm}|}
\toprule
		Вопрос & 1 &  2 & 3 & 4 & 5 & 6 & 7 & 8 & 9 & 10 \\
\midrule
		Ответ &  &  & & & & & & & & \\
 \bottomrule
\end{tabular}
}




% [1][3] 1 = one argument, 3 = value if missing
% эта магия создаёт окружение answerlist
% именно в окружении answerlist записаны варианты ответов в подключаемых exerciseXX
% просто \begin{answerlist} сделает ответы в три столбца
% если ответы длинные, то надо в них руками сделать
% \begin{answerlist}[1] чтобы они шли в один столбец
\newenvironment{answerlist}[1][3]{
\begin{multicols}{#1}

\begin{enumerate}[label=\fbox{\emph{\Alph*}},ref=\emph{\alph*}]
}
{
\item Нет верного ответа.
\end{enumerate}
\end{multicols}
}

% BB: unicol version. don't know why \ifthenelse fails in second part of new-env
\newenvironment{answerlistu}{
\begin{enumerate}[label=\fbox{\emph{\Alph*}},ref=\emph{\alph*}]
}
{
\item Нет верного ответа.
\end{enumerate}
}



\excludecomment{solution} % without solutions

\theoremstyle{definition}
\newtheorem{question}{Вопрос}



\begin{document}

\putyourname


%\testtable

%\checktable


\begin{question}
Пусть \(X_1, \, \ldots, \, X_n\) --- случайная выборка из распределения
Пуассона с параметром \(\lambda > 0\). Известно, что оценка
максимального правдоподобия параметра \(\lambda\) равна \(\bar{X}\).
Чему равна оценка максимального правдоподобия для \(1 / \lambda\)?
\begin{answerlist}
  \item \(\ln \bar{X}\)
  \item \(\bar{X} / n\)
  \item \(1 / \bar{X}\)
  \item \(\bar{X}\)
  \item \(e^{\bar{X}}\)
\end{answerlist}
\end{question}

\begin{solution}
\begin{answerlist}
  \item Bad answer :(
  \item Bad answer :(
  \item Good answer :)
  \item Bad answer :(
  \item Bad answer :(
\end{answerlist}
\end{solution}



\begin{question}
Если \(\E(X)=0\), то, согласно неравенству Чебышева,
\(\P(|X| \leq 5 \sqrt{\Var(X)})\) лежит в интервале
\begin{answerlist}
  \item \([0.96;1]\)
  \item \([0;0.2]\)
  \item \([0.8;1]\)
  \item \([0.5;1]\)
  \item \([0;0.04]\)
\end{answerlist}
\end{question}

\begin{solution}
\begin{answerlist}
  \item Отлично
  \item Неверно
  \item Неверно
  \item Неверно
  \item Неверно
\end{answerlist}
\end{solution}



\begin{question}
Случайная выборка состоит из одного наблюдения \(X_1\), которое имеет
плотность распределения \[
f(x; \, \theta) = \begin{cases}
    \tfrac{1}{\theta^2} x e^{-x/\theta} & \text{при } x > 0,  \\
    0 & \text{при }x\leq 0,
  \end{cases}
\] где \(\theta > 0\). Чему равна оценка неизвестного параметра
\(\theta\), найденная с помощью метода максимального правдоподобия?
\begin{answerlist}
  \item \(X_1\)
  \item \(\ln X_1\)
  \item \(X_1 / 2\)
  \item \(\frac{X_1}{\ln X_1}\)
  \item \(1 / \ln X_1\)
\end{answerlist}
\end{question}

\begin{solution}
\begin{answerlist}
  \item Bad answer :(
  \item Bad answer :(
  \item Good answer :)
  \item Bad answer :(
  \item Bad answer :(
\end{answerlist}
\end{solution}



\begin{question}
Число изюминок в булочке — случайная величина, имеющая распределение
Пуассона. Известно, что в среднем каждая булочка содержит 13 изюминок.
Вероятность того, что в случайно выбранной булочке окажется только одна
изюминка равна:
\begin{answerlist}
  \item \(13e^-13\)
  \item \(e^-13/13\)
  \item \(e^-13\)
  \item \(1/13\)
  \item \(e^13/13!\)
\end{answerlist}
\end{question}

\begin{solution}
\begin{answerlist}
  \item Отлично
  \item Неверно
  \item Неверно
  \item Неверно
  \item Неверно
\end{answerlist}
\end{solution}


\newpage

\begin{question}
Функция распределения случайной величины \(X\) имеет вид \[
F(x)=\begin{cases}
0, \; \text{ если } x<0 \\
cx^2, \; \text{ если } x\in [0;1] \\
1, \; \text{ если } x>1
\end{cases}
\]

\vspace{0.5cm}

Константа \(c\) равна
\begin{answerlist}
  \item \(2/3\)
  \item \(0.5\)
  \item \(1.5\)
  \item \(2\)
  \item \(1\)
\end{answerlist}
\end{question}

\begin{solution}
\begin{answerlist}
  \item Неверно
  \item Неверно
  \item Неверно
  \item Неверно
  \item Отлично
\end{answerlist}
\end{solution}



\begin{question}
Величина \(X\) с равными вероятностями принимает только два значения,
\(-1\) и \(1\), и \(\E(Y|X=x)=1\). Ожидание \(\E(Y)\) равно
\begin{answerlist}
  \item \(1\)
  \item \(-1\)
  \item \(0.5\)
  \item \(0.5\)
  \item \(0\)
\end{answerlist}
\end{question}

\begin{solution}
\begin{answerlist}
  \item Good answer :)
  \item Bad answer :(
  \item Bad answer :(
  \item Bad answer :(
  \item Bad answer :(
\end{answerlist}
\end{solution}



\begin{question}
Случайным образом выбирается семья с двумя детьми. Событие \(A\) --- в
семье старший ребенок --- мальчик, событие \(B\) --- в семье только один
из детей --- мальчик, событие \(C\) --- в семье хотя бы один из детей
--- мальчик.
\begin{answerlist}
  \item Любые два события из \(A\), \(B\), \(C\) --- зависимы
  \item События \(A\), \(B\), \(C\) --- независимы в совокупности
  \item \(A\) и \(B\) --- независимы, \(A\) и \(C\) --- зависимы, \(B\) и \(C\)
--- зависимы
  \item \(\P(A\cap B\cap C)=\P(A)\P(B)\P(C)\)
  \item События \(A\), \(B\), \(C\) --- независимы попарно, но зависимы в
совокупности
\end{answerlist}
\end{question}

\begin{solution}
\begin{answerlist}
  \item Неверно
  \item Неверно
  \item Отлично
  \item Неверно
  \item Неверно
\end{answerlist}
\end{solution}



\begin{question}
Для энтропий пары случайных величин выполнено соотношение
\begin{answerlist}
  \item \(H(X) \cdot H(Y) = H(X, Y)\)
  \item \(H(Y|X) + H(X|Y) = H(X, Y)\)
  \item \(H(Y|X) + H(X) = H(X, Y)\)
  \item \(H(X) + H(Y) = H(X, Y)\)
  \item \(H(X\cdot Y) / H(X) = H(Y|X)\)
\end{answerlist}
\end{question}

\begin{solution}
\begin{answerlist}
  \item Bad answer :(
  \item Bad answer :(
  \item Good answer :)
  \item Bad answer :(
  \item Bad answer :(
\end{answerlist}
\end{solution}



\begin{question}
Функция распределения случайной величины \(X\) имеет вид \[
F(x)=\begin{cases}
0, \; \text{ если } x<0 \\
cx^2, \; \text{ если } x\in [0;1] \\
1, \; \text{ если } x>1
\end{cases}
\]

\vspace{0.5cm}

Математическое ожидание \(\E(X)\) равно
\begin{answerlist}
  \item \(3/4\)
  \item \(2/3\)
  \item \(1/4\)
  \item \(2\)
  \item \(1/2\)
\end{answerlist}
\end{question}

\begin{solution}
\begin{answerlist}
  \item Неверно
  \item Отлично
  \item Неверно
  \item Неверно
  \item Неверно
\end{answerlist}
\end{solution}



\begin{question}
Вася бросает 7 правильных игральных кубиков. Пусть величина \(X\) ---
сумма очков, выпавших на первых двух кубиках, а величина \(Y\) — сумма
очков, выпавших на следующих пяти кубиках. Ковариация \(\Cov(X,Y)\)
равна
\begin{answerlist}
  \item \(0\)
  \item \(-2/5\)
  \item \(2/5\)
  \item \(1\)
  \item \(0.5\)
\end{answerlist}
\end{question}

\begin{solution}
\begin{answerlist}
  \item Отлично
  \item Неверно
  \item Неверно
  \item Неверно
  \item Неверно
\end{answerlist}
\end{solution}


\newpage

\begin{question}
Cовместная функция плотности случайных величин \(X\) и \(Y\) имеет вид
\[
f(x,y)=\begin{cases}
\frac{1}{4}xy, \; \text{ если } x\in[0;2], y\in [0;2] \\
0, \; \text{ иначе}
\end{cases}
\]

Найдите вероятность \(\P(Y = X)\)
\begin{answerlist}
  \item \(3/4\)
  \item \(0\)
  \item \(1/2\)
  \item \(1/4\)
  \item невозможно вычислить на основе имеющихся данных
\end{answerlist}
\end{question}

\begin{solution}
\begin{answerlist}
  \item Bad answer :(
  \item Good answer :)
  \item Bad answer :(
  \item Bad answer :(
  \item Bad answer :(
\end{answerlist}
\end{solution}



\begin{question}
Для дискретной случайной величины функция распределения
\begin{answerlist}
  \item не определена
  \item вырождена
  \item непрерывна
  \item имеет разрывы
  \item строго возрастает
\end{answerlist}
\end{question}

\begin{solution}
\begin{answerlist}
  \item Bad answer :(
  \item Bad answer :(
  \item Bad answer :(
  \item Good answer :)
  \item Bad answer :(
\end{answerlist}
\end{solution}



\begin{question}
Совместное распределение пары величин \(X\) и \(Y\) задано таблицей:

\begin{tabular}{@{}c|ccc@{}}
\toprule
       & $Y=-1$ & $Y=0$ & $Y=1$ \\ \midrule
$X=-1$ & $1/4$  & $0$   & $1/4$ \\
$X=1$  & $1/6$  & $1/6$ & $1/6$ \\ \bottomrule
\end{tabular}

\vspace{0.5cm}

Вероятность того, что \(X=1\) при условии, что \(Y<0\) равна
\begin{answerlist}
  \item \(5/12\)
  \item \(1/3\)
  \item \(1/6\)
  \item \(1/12\)
  \item \(2/5\)
\end{answerlist}
\end{question}

\begin{solution}
\begin{answerlist}
  \item Неверно
  \item Неверно
  \item Неверно
  \item Неверно
  \item Отлично
\end{answerlist}
\end{solution}



\begin{question}
Случайная величина \(X\) имеет непрерывное распределение, при этом
\(\P(X\leq 3)=0.25\) и \(\P(X>0.25)=0.8\). Квантиль порядка \(0.25\)
величины \(X\) может быть равен
\begin{answerlist}
  \item \(0.2\)
  \item \(3\)
  \item \(0.25\)
  \item \(0.8\)
  \item \(0.75\)
\end{answerlist}
\end{question}

\begin{solution}
\begin{answerlist}
  \item Bad answer :(
  \item Good answer :)
  \item Bad answer :(
  \item Bad answer :(
  \item Bad answer :(
\end{answerlist}
\end{solution}



\begin{question}
Случайные величины \(X\) и \(Y\) распределены нормально с неизвестным
математическим ожиданием и неизвестной дисперсией. Для тестирования
гипотезы о равенстве дисперсий выбирается \(20\) наблюдений случайной
величины \(X\) и \(30\) наблюдений случайной величины \(Y\). Какое
распределение может иметь статистика, используемая в данном случае?
\begin{answerlist}
  \item \(\chi^2_{48}\)
  \item \(F_{20,30}\)
  \item \(t_{48}\)
  \item \(F_{29,19}\)
  \item \(\chi^2_{49}\)
\end{answerlist}
\end{question}

\begin{solution}
\begin{answerlist}
  \item Bad answer :(
  \item Bad answer :(
  \item Bad answer :(
  \item Good answer :)
  \item Bad answer :(
\end{answerlist}
\end{solution}


\newpage

\begin{question}
Вася бросает 7 правильных игральных кубиков. Вероятность того, что ровно
на пяти из кубиков выпадет шестёрка равна
\begin{answerlist}
  \item \(\frac{525}{12}\left(\frac{1}{6}\right)^7\)
  \item \(\left(\frac{1}{6}\right)^5\)
  \item \(\left(\frac{1}{6}\right)^7\)
  \item \(525\left(\frac{1}{6}\right)^7\)
  \item \(\frac{7}{12}\left(\frac{1}{6}\right)^5\)
\end{answerlist}
\end{question}

\begin{solution}
\begin{answerlist}
  \item Неверно
  \item Неверно
  \item Неверно
  \item Отлично
  \item Неверно
\end{answerlist}
\end{solution}



\begin{question}
Оценка \(\hat \theta_n\) называется эффективной оценкой параметра
\(\theta\) в классе оценок \(K\), если
\begin{answerlist}
  \item \(\Var(\hat \theta_n)=(\theta)^2/n\)
  \item \(\E(\hat \theta_n)=\theta\)
  \item \(\E((\hat \theta_n - \theta)^2) \stackrel{}{\to} 0\) при
\(n\stackrel{}{\to} \infty\)
  \item \(\E((\hat \theta_n - \theta)^2) \leq \E((\tilde \theta - \theta)^2)\)
для всех \(\tilde \theta \in K\)
  \item \(\hat \theta_n \stackrel{\P}{\to} \theta\) при
\(n\stackrel{}{\to} \infty\)
\end{answerlist}
\end{question}

\begin{solution}
\begin{answerlist}
  \item Bad answer :(
  \item Bad answer :(
  \item Bad answer :(
  \item Good answer :)
  \item Bad answer :(
\end{answerlist}
\end{solution}



\begin{question}
Пусть \(X_1\), \(X_2\), \ldots, \(X_n\) — последовательность
независимых одинаково распределенных случайных величин, \(\E(X_i)=3\) и
\(\Var(X_i)=9\). Следующая величина имеет асимптотически стандартное
нормальное распределение
\begin{answerlist}
  \item \(\frac{X_n-3}{3}\)
  \item \(\sqrt{n}\frac{\bar{X}-3}{3}\)
  \item \(\frac{\bar{X}_n-3}{3}\)
  \item \(\frac{\bar{X}_n-3}{3\sqrt{n}}\)
  \item \(\sqrt{n}(\bar{X}-3)\)
\end{answerlist}
\end{question}

\begin{solution}
\begin{answerlist}
  \item Неверно
  \item Отлично
  \item Неверно
  \item Неверно
  \item Неверно
\end{answerlist}
\end{solution}



\begin{question}
Функция распределения случайной величины \(X\) имеет вид \[
F(x)=\begin{cases}
0, \; \text{ если } x<0 \\
cx^2, \; \text{ если } x\in [0;1] \\
1, \; \text{ если } x>1
\end{cases}
\]

\vspace{0.5cm}

Вероятность того, что величина \(X\) примет значение из интервала
\([0.5, 1.5]\) равна
\begin{answerlist}
  \item \(1\)
  \item \(2/3\)
  \item \(3/2\)
  \item \(1/2\)
  \item \(3/4\)
\end{answerlist}
\end{question}

\begin{solution}
\begin{answerlist}
  \item Неверно
  \item Неверно
  \item Неверно
  \item Неверно
  \item Отлично
\end{answerlist}
\end{solution}



\begin{question}
Если функция правдоподобия пропорциональна \(a^2(1-a)^6\), априорная
плотность пропорциональна \(\exp(-a)\), то апостериорная плотность
параметра \(a\) пропорциональна
\begin{answerlist}
  \item \(0.5 a^2(1-a)^6 +0.5\exp(a)\)
  \item \(\frac{a^2(1-a)^6}{\exp(-a)}\)
  \item \(\frac{a^2(1-a)^6}{\exp(a)}\)
  \item \(0.5 a^2(1-a)^6 +0.5\exp(-a)\)
  \item \(\frac{\exp(-a)}{a^2(1-a)^6}\)
\end{answerlist}
\end{question}

\begin{solution}
\begin{answerlist}
  \item Bad answer :(
  \item Bad answer :(
  \item Good answer :)
  \item Bad answer :(
  \item Bad answer :(
\end{answerlist}
\end{solution}


\newpage

\begin{question}
Совместное распределение пары величин \(X\) и \(Y\) задано таблицей:

\begin{tabular}{@{}c|ccc@{}}
\toprule
       & $Y=-1$ & $Y=0$ & $Y=1$ \\ \midrule
$X=-1$ & $1/4$  & $0$   & $1/4$ \\
$X=1$  & $1/6$  & $1/6$ & $1/6$ \\ \bottomrule
\end{tabular}

\vspace{0.5cm}

Дисперсия случайной величины \(Y\) равна
\begin{answerlist}
  \item \(5/12\)
  \item \(1/3\)
  \item \(1/2\)
  \item \(12/5\)
  \item \(5/6\)
\end{answerlist}
\end{question}

\begin{solution}
\begin{answerlist}
  \item Неверно
  \item Неверно
  \item Неверно
  \item Неверно
  \item Отлично
\end{answerlist}
\end{solution}



\begin{question}
Рассмотрим алгоритм Метрополиса-Гастингса для получения выборки
параметра с апостериорной плотностью пропорциональной \(t^2\).
Предлагаемый переход из \(a\) в \(b\) задаётся правилом, \(b = a + Z\),
где \(Z \sim \cN(0;4)\). Вероятность одобрения перехода из точки \(0.5\)
в точку \(0.3\) равна
\begin{answerlist}
  \item \(0.5\)
  \item \(0.36\)
  \item \(1\)
  \item \(0.64\)
  \item \(0.6\)
\end{answerlist}
\end{question}

\begin{solution}
\begin{answerlist}
  \item Bad answer :(
  \item Good answer :)
  \item Bad answer :(
  \item Bad answer :(
  \item Bad answer :(
\end{answerlist}
\end{solution}



\begin{question}
Вася бросает 7 правильных игральных кубиков. Дисперсия суммы выпавших
очков равна
\begin{answerlist}
  \item \(7\)
  \item \(7\cdot \frac{35}{36}\)
  \item \(7/6\)
  \item \(35/36\)
  \item \(7\cdot\frac{35}{12}\)
\end{answerlist}
\end{question}

\begin{solution}
\begin{answerlist}
  \item Неверно
  \item Неверно
  \item Неверно
  \item Неверно
  \item Отлично
\end{answerlist}
\end{solution}



\begin{question}
Рассмотрим хи-квадрат случайную величину с \(n\) степенями свободы.
Укажите множество всех возможных значений, принимаемых данной случайной
величиной с ненулевой вероятностью:
\begin{answerlist}
  \item \([0,n]\)
  \item \((0,\infty )\)
  \item \(\left\{x\in R:\sum\limits_{i=1}^{n}{x_{{}}^{2}}=1\right\}\)
  \item \(\{0,1,\ldots,n\}\)
  \item \([0,n^2]\)
\end{answerlist}
\end{question}

\begin{solution}
\begin{answerlist}
  \item Bad answer :(
  \item Good answer :)
  \item Bad answer :(
  \item Bad answer :(
  \item Bad answer :(
\end{answerlist}
\end{solution}



\begin{question}
Вася бросает 7 правильных игральных кубиков. Наиболее вероятное
количество выпавших шестёрок равно
\begin{answerlist}
  \item \(2\)
  \item \(1\)
  \item \(7/6\)
  \item \(6/7\)
  \item \(0\)
\end{answerlist}
\end{question}

\begin{solution}
\begin{answerlist}
  \item Неверно
  \item Отлично
  \item Неверно
  \item Неверно
  \item Неверно
\end{answerlist}
\end{solution}


\newpage

\begin{question}
Известно, что \(\E(X)=1\), \(\Var(X)=1\), \(\E(Y)=4\), \(\Var(Y)=9\),
\(\Cov(X,Y)=-3\)

\vspace{0.5cm}

Корреляция \(\Corr(2X+3,4Y-5)\) равна
\begin{answerlist}
  \item \(1/6\)
  \item \(-1\)
  \item \(1/3\)
  \item \(1\)
  \item \(-1/8\)
\end{answerlist}
\end{question}

\begin{solution}
\begin{answerlist}
  \item Неверно
  \item Отлично
  \item Неверно
  \item Неверно
  \item Неверно
\end{answerlist}
\end{solution}



\begin{question}
Вася бросает 7 правильных игральных кубиков. Математическое ожидание
суммы выпавших очков равно
\begin{answerlist}
  \item \(42\)
  \item \(7/6\)
  \item \(30\)
  \item \(21\)
  \item \(24.5\)
\end{answerlist}
\end{question}

\begin{solution}
\begin{answerlist}
  \item Неверно
  \item Неверно
  \item Неверно
  \item Неверно
  \item Отлично
\end{answerlist}
\end{solution}



\begin{question}
Имеется три монетки. Две «правильных» и одна — с «орлами» по обеим
сторонам. Вася выбирает одну монетку наугад и подкидывает ее один раз.
Вероятность того, что выпадет орел равна
\begin{answerlist}
  \item \(2/3\)
  \item \(1/2\)
  \item \(1/3\)
  \item \(3/5\)
  \item \(2/5\)
\end{answerlist}
\end{question}

\begin{solution}
\begin{answerlist}
  \item Отлично
  \item Неверно
  \item Неверно
  \item Неверно
  \item Неверно
\end{answerlist}
\end{solution}



\begin{question}
Величина \(X\) принимает три значения \(1\), \(2\) и \(3\). По случайной
выборке из ста наблюдений оказалось, что \(1\) выпало 40 раз, \(2\) ---
40 раз и \(3\) --- 20 раз. Андрей Николаевич хочет проверить гипотезу о
том, что все три вероятности одинаковые. Значение критерия согласия
Колмогорова равно
\begin{answerlist}
  \item \(3/5\)
  \item \(2/15\)
  \item \(3/4\)
  \item \(2/5\)
  \item \(1/4\)
\end{answerlist}
\end{question}

\begin{solution}
\begin{answerlist}
  \item Bad answer :(
  \item Good answer :)
  \item Bad answer :(
  \item Bad answer :(
  \item Bad answer :(
\end{answerlist}
\end{solution}



\begin{question}
Величина \(X\) принимает три значения \(1\), \(2\) и \(3\). По случайной
выборке из ста наблюдений оказалось, что \(1\) выпало 40 раз, \(2\) ---
40 раз и \(3\) --- 20 раз. Карл хочет проверить гипотезу о том, что все
три вероятности одинаковые. Значение критерия согласия Пирсона равно
\begin{answerlist}
  \item \(4\)
  \item \(8\)
  \item \(6\)
  \item \(7\)
  \item \(5\)
\end{answerlist}
\end{question}

\begin{solution}
\begin{answerlist}
  \item Bad answer :(
  \item Good answer :)
  \item Bad answer :(
  \item Bad answer :(
  \item Bad answer :(
\end{answerlist}
\end{solution}


 






\newpage
\lfoot{Вариант $\kappa$}
\setcounter{question}{0}


\putyourname

%\testtable

%\checktable


\begin{question}
Пусть \(X_1, \, \ldots, \, X_n\) --- случайная выборка из распределения
Пуассона с параметром \(\lambda > 0\). Известно, что оценка
максимального правдоподобия параметра \(\lambda\) равна \(\bar{X}\).
Чему равна оценка максимального правдоподобия для \(1 / \lambda\)?
\begin{answerlist}
  \item \(\ln \bar{X}\)
  \item \(\bar{X} / n\)
  \item \(1 / \bar{X}\)
  \item \(\bar{X}\)
  \item \(e^{\bar{X}}\)
\end{answerlist}
\end{question}

\begin{solution}
\begin{answerlist}
  \item Bad answer :(
  \item Bad answer :(
  \item Good answer :)
  \item Bad answer :(
  \item Bad answer :(
\end{answerlist}
\end{solution}



\begin{question}
Если \(\E(X)=0\), то, согласно неравенству Чебышева,
\(\P(|X| \leq 5 \sqrt{\Var(X)})\) лежит в интервале
\begin{answerlist}
  \item \([0.96;1]\)
  \item \([0;0.2]\)
  \item \([0.8;1]\)
  \item \([0.5;1]\)
  \item \([0;0.04]\)
\end{answerlist}
\end{question}

\begin{solution}
\begin{answerlist}
  \item Отлично
  \item Неверно
  \item Неверно
  \item Неверно
  \item Неверно
\end{answerlist}
\end{solution}



\begin{question}
Случайная выборка состоит из одного наблюдения \(X_1\), которое имеет
плотность распределения \[
f(x; \, \theta) = \begin{cases}
    \tfrac{1}{\theta^2} x e^{-x/\theta} & \text{при } x > 0,  \\
    0 & \text{при }x\leq 0,
  \end{cases}
\] где \(\theta > 0\). Чему равна оценка неизвестного параметра
\(\theta\), найденная с помощью метода максимального правдоподобия?
\begin{answerlist}
  \item \(X_1\)
  \item \(\ln X_1\)
  \item \(X_1 / 2\)
  \item \(\frac{X_1}{\ln X_1}\)
  \item \(1 / \ln X_1\)
\end{answerlist}
\end{question}

\begin{solution}
\begin{answerlist}
  \item Bad answer :(
  \item Bad answer :(
  \item Good answer :)
  \item Bad answer :(
  \item Bad answer :(
\end{answerlist}
\end{solution}



\begin{question}
Число изюминок в булочке — случайная величина, имеющая распределение
Пуассона. Известно, что в среднем каждая булочка содержит 13 изюминок.
Вероятность того, что в случайно выбранной булочке окажется только одна
изюминка равна:
\begin{answerlist}
  \item \(13e^-13\)
  \item \(e^-13/13\)
  \item \(e^-13\)
  \item \(1/13\)
  \item \(e^13/13!\)
\end{answerlist}
\end{question}

\begin{solution}
\begin{answerlist}
  \item Отлично
  \item Неверно
  \item Неверно
  \item Неверно
  \item Неверно
\end{answerlist}
\end{solution}



\begin{question}
Функция распределения случайной величины \(X\) имеет вид \[
F(x)=\begin{cases}
0, \; \text{ если } x<0 \\
cx^2, \; \text{ если } x\in [0;1] \\
1, \; \text{ если } x>1
\end{cases}
\]

\vspace{0.5cm}

Константа \(c\) равна
\begin{answerlist}
  \item \(2/3\)
  \item \(0.5\)
  \item \(1.5\)
  \item \(2\)
  \item \(1\)
\end{answerlist}
\end{question}

\begin{solution}
\begin{answerlist}
  \item Неверно
  \item Неверно
  \item Неверно
  \item Неверно
  \item Отлично
\end{answerlist}
\end{solution}


\newpage

\begin{question}
Величина \(X\) с равными вероятностями принимает только два значения,
\(-1\) и \(1\), и \(\E(Y|X=x)=1\). Ожидание \(\E(Y)\) равно
\begin{answerlist}
  \item \(1\)
  \item \(-1\)
  \item \(0.5\)
  \item \(0.5\)
  \item \(0\)
\end{answerlist}
\end{question}

\begin{solution}
\begin{answerlist}
  \item Good answer :)
  \item Bad answer :(
  \item Bad answer :(
  \item Bad answer :(
  \item Bad answer :(
\end{answerlist}
\end{solution}



\begin{question}
Случайным образом выбирается семья с двумя детьми. Событие \(A\) --- в
семье старший ребенок --- мальчик, событие \(B\) --- в семье только один
из детей --- мальчик, событие \(C\) --- в семье хотя бы один из детей
--- мальчик.
\begin{answerlist}
  \item Любые два события из \(A\), \(B\), \(C\) --- зависимы
  \item События \(A\), \(B\), \(C\) --- независимы в совокупности
  \item \(A\) и \(B\) --- независимы, \(A\) и \(C\) --- зависимы, \(B\) и \(C\)
--- зависимы
  \item \(\P(A\cap B\cap C)=\P(A)\P(B)\P(C)\)
  \item События \(A\), \(B\), \(C\) --- независимы попарно, но зависимы в
совокупности
\end{answerlist}
\end{question}

\begin{solution}
\begin{answerlist}
  \item Неверно
  \item Неверно
  \item Отлично
  \item Неверно
  \item Неверно
\end{answerlist}
\end{solution}



\begin{question}
Для энтропий пары случайных величин выполнено соотношение
\begin{answerlist}
  \item \(H(X) \cdot H(Y) = H(X, Y)\)
  \item \(H(Y|X) + H(X|Y) = H(X, Y)\)
  \item \(H(Y|X) + H(X) = H(X, Y)\)
  \item \(H(X) + H(Y) = H(X, Y)\)
  \item \(H(X\cdot Y) / H(X) = H(Y|X)\)
\end{answerlist}
\end{question}

\begin{solution}
\begin{answerlist}
  \item Bad answer :(
  \item Bad answer :(
  \item Good answer :)
  \item Bad answer :(
  \item Bad answer :(
\end{answerlist}
\end{solution}



\begin{question}
Функция распределения случайной величины \(X\) имеет вид \[
F(x)=\begin{cases}
0, \; \text{ если } x<0 \\
cx^2, \; \text{ если } x\in [0;1] \\
1, \; \text{ если } x>1
\end{cases}
\]

\vspace{0.5cm}

Математическое ожидание \(\E(X)\) равно
\begin{answerlist}
  \item \(3/4\)
  \item \(2/3\)
  \item \(1/4\)
  \item \(2\)
  \item \(1/2\)
\end{answerlist}
\end{question}

\begin{solution}
\begin{answerlist}
  \item Неверно
  \item Отлично
  \item Неверно
  \item Неверно
  \item Неверно
\end{answerlist}
\end{solution}



\begin{question}
Вася бросает 7 правильных игральных кубиков. Пусть величина \(X\) ---
сумма очков, выпавших на первых двух кубиках, а величина \(Y\) — сумма
очков, выпавших на следующих пяти кубиках. Ковариация \(\Cov(X,Y)\)
равна
\begin{answerlist}
  \item \(0\)
  \item \(-2/5\)
  \item \(2/5\)
  \item \(1\)
  \item \(0.5\)
\end{answerlist}
\end{question}

\begin{solution}
\begin{answerlist}
  \item Отлично
  \item Неверно
  \item Неверно
  \item Неверно
  \item Неверно
\end{answerlist}
\end{solution}


\newpage

\begin{question}
Cовместная функция плотности случайных величин \(X\) и \(Y\) имеет вид
\[
f(x,y)=\begin{cases}
\frac{1}{4}xy, \; \text{ если } x\in[0;2], y\in [0;2] \\
0, \; \text{ иначе}
\end{cases}
\]

Найдите вероятность \(\P(Y = X)\)
\begin{answerlist}
  \item \(3/4\)
  \item \(0\)
  \item \(1/2\)
  \item \(1/4\)
  \item невозможно вычислить на основе имеющихся данных
\end{answerlist}
\end{question}

\begin{solution}
\begin{answerlist}
  \item Bad answer :(
  \item Good answer :)
  \item Bad answer :(
  \item Bad answer :(
  \item Bad answer :(
\end{answerlist}
\end{solution}



\begin{question}
Для дискретной случайной величины функция распределения
\begin{answerlist}
  \item не определена
  \item вырождена
  \item непрерывна
  \item имеет разрывы
  \item строго возрастает
\end{answerlist}
\end{question}

\begin{solution}
\begin{answerlist}
  \item Bad answer :(
  \item Bad answer :(
  \item Bad answer :(
  \item Good answer :)
  \item Bad answer :(
\end{answerlist}
\end{solution}



\begin{question}
Совместное распределение пары величин \(X\) и \(Y\) задано таблицей:

\begin{tabular}{@{}c|ccc@{}}
\toprule
       & $Y=-1$ & $Y=0$ & $Y=1$ \\ \midrule
$X=-1$ & $1/4$  & $0$   & $1/4$ \\
$X=1$  & $1/6$  & $1/6$ & $1/6$ \\ \bottomrule
\end{tabular}

\vspace{0.5cm}

Вероятность того, что \(X=1\) при условии, что \(Y<0\) равна
\begin{answerlist}
  \item \(5/12\)
  \item \(1/3\)
  \item \(1/6\)
  \item \(1/12\)
  \item \(2/5\)
\end{answerlist}
\end{question}

\begin{solution}
\begin{answerlist}
  \item Неверно
  \item Неверно
  \item Неверно
  \item Неверно
  \item Отлично
\end{answerlist}
\end{solution}



\begin{question}
Случайная величина \(X\) имеет непрерывное распределение, при этом
\(\P(X\leq 3)=0.25\) и \(\P(X>0.25)=0.8\). Квантиль порядка \(0.25\)
величины \(X\) может быть равен
\begin{answerlist}
  \item \(0.2\)
  \item \(3\)
  \item \(0.25\)
  \item \(0.8\)
  \item \(0.75\)
\end{answerlist}
\end{question}

\begin{solution}
\begin{answerlist}
  \item Bad answer :(
  \item Good answer :)
  \item Bad answer :(
  \item Bad answer :(
  \item Bad answer :(
\end{answerlist}
\end{solution}



\begin{question}
Случайные величины \(X\) и \(Y\) распределены нормально с неизвестным
математическим ожиданием и неизвестной дисперсией. Для тестирования
гипотезы о равенстве дисперсий выбирается \(20\) наблюдений случайной
величины \(X\) и \(30\) наблюдений случайной величины \(Y\). Какое
распределение может иметь статистика, используемая в данном случае?
\begin{answerlist}
  \item \(\chi^2_{48}\)
  \item \(F_{20,30}\)
  \item \(t_{48}\)
  \item \(F_{29,19}\)
  \item \(\chi^2_{49}\)
\end{answerlist}
\end{question}

\begin{solution}
\begin{answerlist}
  \item Bad answer :(
  \item Bad answer :(
  \item Bad answer :(
  \item Good answer :)
  \item Bad answer :(
\end{answerlist}
\end{solution}


\newpage

\begin{question}
Вася бросает 7 правильных игральных кубиков. Вероятность того, что ровно
на пяти из кубиков выпадет шестёрка равна
\begin{answerlist}
  \item \(\frac{525}{12}\left(\frac{1}{6}\right)^7\)
  \item \(\left(\frac{1}{6}\right)^5\)
  \item \(\left(\frac{1}{6}\right)^7\)
  \item \(525\left(\frac{1}{6}\right)^7\)
  \item \(\frac{7}{12}\left(\frac{1}{6}\right)^5\)
\end{answerlist}
\end{question}

\begin{solution}
\begin{answerlist}
  \item Неверно
  \item Неверно
  \item Неверно
  \item Отлично
  \item Неверно
\end{answerlist}
\end{solution}



\begin{question}
Оценка \(\hat \theta_n\) называется эффективной оценкой параметра
\(\theta\) в классе оценок \(K\), если
\begin{answerlist}
  \item \(\Var(\hat \theta_n)=(\theta)^2/n\)
  \item \(\E(\hat \theta_n)=\theta\)
  \item \(\E((\hat \theta_n - \theta)^2) \stackrel{}{\to} 0\) при
\(n\stackrel{}{\to} \infty\)
  \item \(\E((\hat \theta_n - \theta)^2) \leq \E((\tilde \theta - \theta)^2)\)
для всех \(\tilde \theta \in K\)
  \item \(\hat \theta_n \stackrel{\P}{\to} \theta\) при
\(n\stackrel{}{\to} \infty\)
\end{answerlist}
\end{question}

\begin{solution}
\begin{answerlist}
  \item Bad answer :(
  \item Bad answer :(
  \item Bad answer :(
  \item Good answer :)
  \item Bad answer :(
\end{answerlist}
\end{solution}



\begin{question}
Пусть \(X_1\), \(X_2\), \ldots, \(X_n\) — последовательность
независимых одинаково распределенных случайных величин, \(\E(X_i)=3\) и
\(\Var(X_i)=9\). Следующая величина имеет асимптотически стандартное
нормальное распределение
\begin{answerlist}
  \item \(\frac{X_n-3}{3}\)
  \item \(\sqrt{n}\frac{\bar{X}-3}{3}\)
  \item \(\frac{\bar{X}_n-3}{3}\)
  \item \(\frac{\bar{X}_n-3}{3\sqrt{n}}\)
  \item \(\sqrt{n}(\bar{X}-3)\)
\end{answerlist}
\end{question}

\begin{solution}
\begin{answerlist}
  \item Неверно
  \item Отлично
  \item Неверно
  \item Неверно
  \item Неверно
\end{answerlist}
\end{solution}



\begin{question}
Функция распределения случайной величины \(X\) имеет вид \[
F(x)=\begin{cases}
0, \; \text{ если } x<0 \\
cx^2, \; \text{ если } x\in [0;1] \\
1, \; \text{ если } x>1
\end{cases}
\]

\vspace{0.5cm}

Вероятность того, что величина \(X\) примет значение из интервала
\([0.5, 1.5]\) равна
\begin{answerlist}
  \item \(1\)
  \item \(2/3\)
  \item \(3/2\)
  \item \(1/2\)
  \item \(3/4\)
\end{answerlist}
\end{question}

\begin{solution}
\begin{answerlist}
  \item Неверно
  \item Неверно
  \item Неверно
  \item Неверно
  \item Отлично
\end{answerlist}
\end{solution}



\begin{question}
Если функция правдоподобия пропорциональна \(a^2(1-a)^6\), априорная
плотность пропорциональна \(\exp(-a)\), то апостериорная плотность
параметра \(a\) пропорциональна
\begin{answerlist}
  \item \(0.5 a^2(1-a)^6 +0.5\exp(a)\)
  \item \(\frac{a^2(1-a)^6}{\exp(-a)}\)
  \item \(\frac{a^2(1-a)^6}{\exp(a)}\)
  \item \(0.5 a^2(1-a)^6 +0.5\exp(-a)\)
  \item \(\frac{\exp(-a)}{a^2(1-a)^6}\)
\end{answerlist}
\end{question}

\begin{solution}
\begin{answerlist}
  \item Bad answer :(
  \item Bad answer :(
  \item Good answer :)
  \item Bad answer :(
  \item Bad answer :(
\end{answerlist}
\end{solution}


\newpage

\begin{question}
Совместное распределение пары величин \(X\) и \(Y\) задано таблицей:

\begin{tabular}{@{}c|ccc@{}}
\toprule
       & $Y=-1$ & $Y=0$ & $Y=1$ \\ \midrule
$X=-1$ & $1/4$  & $0$   & $1/4$ \\
$X=1$  & $1/6$  & $1/6$ & $1/6$ \\ \bottomrule
\end{tabular}

\vspace{0.5cm}

Дисперсия случайной величины \(Y\) равна
\begin{answerlist}
  \item \(5/12\)
  \item \(1/3\)
  \item \(1/2\)
  \item \(12/5\)
  \item \(5/6\)
\end{answerlist}
\end{question}

\begin{solution}
\begin{answerlist}
  \item Неверно
  \item Неверно
  \item Неверно
  \item Неверно
  \item Отлично
\end{answerlist}
\end{solution}



\begin{question}
Рассмотрим алгоритм Метрополиса-Гастингса для получения выборки
параметра с апостериорной плотностью пропорциональной \(t^2\).
Предлагаемый переход из \(a\) в \(b\) задаётся правилом, \(b = a + Z\),
где \(Z \sim \cN(0;4)\). Вероятность одобрения перехода из точки \(0.5\)
в точку \(0.3\) равна
\begin{answerlist}
  \item \(0.5\)
  \item \(0.36\)
  \item \(1\)
  \item \(0.64\)
  \item \(0.6\)
\end{answerlist}
\end{question}

\begin{solution}
\begin{answerlist}
  \item Bad answer :(
  \item Good answer :)
  \item Bad answer :(
  \item Bad answer :(
  \item Bad answer :(
\end{answerlist}
\end{solution}



\begin{question}
Вася бросает 7 правильных игральных кубиков. Дисперсия суммы выпавших
очков равна
\begin{answerlist}
  \item \(7\)
  \item \(7\cdot \frac{35}{36}\)
  \item \(7/6\)
  \item \(35/36\)
  \item \(7\cdot\frac{35}{12}\)
\end{answerlist}
\end{question}

\begin{solution}
\begin{answerlist}
  \item Неверно
  \item Неверно
  \item Неверно
  \item Неверно
  \item Отлично
\end{answerlist}
\end{solution}



\begin{question}
Рассмотрим хи-квадрат случайную величину с \(n\) степенями свободы.
Укажите множество всех возможных значений, принимаемых данной случайной
величиной с ненулевой вероятностью:
\begin{answerlist}
  \item \([0,n]\)
  \item \((0,\infty )\)
  \item \(\left\{x\in R:\sum\limits_{i=1}^{n}{x_{{}}^{2}}=1\right\}\)
  \item \(\{0,1,\ldots,n\}\)
  \item \([0,n^2]\)
\end{answerlist}
\end{question}

\begin{solution}
\begin{answerlist}
  \item Bad answer :(
  \item Good answer :)
  \item Bad answer :(
  \item Bad answer :(
  \item Bad answer :(
\end{answerlist}
\end{solution}



\begin{question}
Вася бросает 7 правильных игральных кубиков. Наиболее вероятное
количество выпавших шестёрок равно
\begin{answerlist}
  \item \(2\)
  \item \(1\)
  \item \(7/6\)
  \item \(6/7\)
  \item \(0\)
\end{answerlist}
\end{question}

\begin{solution}
\begin{answerlist}
  \item Неверно
  \item Отлично
  \item Неверно
  \item Неверно
  \item Неверно
\end{answerlist}
\end{solution}


\newpage

\begin{question}
Известно, что \(\E(X)=1\), \(\Var(X)=1\), \(\E(Y)=4\), \(\Var(Y)=9\),
\(\Cov(X,Y)=-3\)

\vspace{0.5cm}

Корреляция \(\Corr(2X+3,4Y-5)\) равна
\begin{answerlist}
  \item \(1/6\)
  \item \(-1\)
  \item \(1/3\)
  \item \(1\)
  \item \(-1/8\)
\end{answerlist}
\end{question}

\begin{solution}
\begin{answerlist}
  \item Неверно
  \item Отлично
  \item Неверно
  \item Неверно
  \item Неверно
\end{answerlist}
\end{solution}



\begin{question}
Вася бросает 7 правильных игральных кубиков. Математическое ожидание
суммы выпавших очков равно
\begin{answerlist}
  \item \(42\)
  \item \(7/6\)
  \item \(30\)
  \item \(21\)
  \item \(24.5\)
\end{answerlist}
\end{question}

\begin{solution}
\begin{answerlist}
  \item Неверно
  \item Неверно
  \item Неверно
  \item Неверно
  \item Отлично
\end{answerlist}
\end{solution}



\begin{question}
Имеется три монетки. Две «правильных» и одна — с «орлами» по обеим
сторонам. Вася выбирает одну монетку наугад и подкидывает ее один раз.
Вероятность того, что выпадет орел равна
\begin{answerlist}
  \item \(2/3\)
  \item \(1/2\)
  \item \(1/3\)
  \item \(3/5\)
  \item \(2/5\)
\end{answerlist}
\end{question}

\begin{solution}
\begin{answerlist}
  \item Отлично
  \item Неверно
  \item Неверно
  \item Неверно
  \item Неверно
\end{answerlist}
\end{solution}



\begin{question}
Величина \(X\) принимает три значения \(1\), \(2\) и \(3\). По случайной
выборке из ста наблюдений оказалось, что \(1\) выпало 40 раз, \(2\) ---
40 раз и \(3\) --- 20 раз. Андрей Николаевич хочет проверить гипотезу о
том, что все три вероятности одинаковые. Значение критерия согласия
Колмогорова равно
\begin{answerlist}
  \item \(3/5\)
  \item \(2/15\)
  \item \(3/4\)
  \item \(2/5\)
  \item \(1/4\)
\end{answerlist}
\end{question}

\begin{solution}
\begin{answerlist}
  \item Bad answer :(
  \item Good answer :)
  \item Bad answer :(
  \item Bad answer :(
  \item Bad answer :(
\end{answerlist}
\end{solution}



\begin{question}
Величина \(X\) принимает три значения \(1\), \(2\) и \(3\). По случайной
выборке из ста наблюдений оказалось, что \(1\) выпало 40 раз, \(2\) ---
40 раз и \(3\) --- 20 раз. Карл хочет проверить гипотезу о том, что все
три вероятности одинаковые. Значение критерия согласия Пирсона равно
\begin{answerlist}
  \item \(4\)
  \item \(8\)
  \item \(6\)
  \item \(7\)
  \item \(5\)
\end{answerlist}
\end{question}

\begin{solution}
\begin{answerlist}
  \item Bad answer :(
  \item Good answer :)
  \item Bad answer :(
  \item Bad answer :(
  \item Bad answer :(
\end{answerlist}
\end{solution}








 

\end{document}
