% !TEX root = ../probability_hse_exams.tex
\thispagestyle{empty}
\section{Решения контрольной номер 4}

\subsection[2018-2019]{\hyperref[sec:kr_04_2018_2019]{2018-2019}}
\label{sec:sol_kr_04_2018_2019}


\begin{enumerate} 
\item	
Из $\gamma = 0.95$ следует, что $\alpha / 2 = 0.025$ и $1 - \alpha / 2 = 0.975$.
	 
$T=\frac{\bar x - p}{\sqrt{\frac{\bar x \cdot (1 - \bar x) }{n}}}\sim \cN (0;1)$
	
$\bar x = \frac{10}{100} = 0.1$

\[
\bar x - B \cdot \sqrt{\frac{\bar x \cdot (1 - \bar x) }{n}} \le p \le \bar x - A \cdot \sqrt{\frac{\bar x \cdot (1 - \bar x) }{n}}
\]

\[	
0.1 - 1.96 \cdot \sqrt{\frac{0.1 \cdot 0.9}{100}} \le p \le 0.1 + 1.96 \cdot \sqrt{\frac{0.1 \cdot 0.9}{100}}
\]

\[
0.0412 \le p \le 0.1588
\]
	
\item
$\alpha = 0.05$
\[
\begin{cases}
H_0: p = \frac{1}{100} \\
H_1: p > \frac{1}{100}
\end{cases}
\]
	
\[	
T_{crit} = 1.64
\]

\[
T_{obs} = \frac{\bar x - p}{\sqrt{\frac{\bar x \cdot (1 - \bar x) }{n}}} = \frac{0.1 - 0.01}{\sqrt{\frac{0.1 \cdot 0.9}{100}}} = 3 
\]

Заметим, что $T_{crit} < T_{obs}$, поэтому гипотеза $H_0$ отвергается.

\[
P-value = 1 - 0.9987 = 0.0013 \approx 0
\]
\item	
$\alpha = 0.1$	
\[
\begin{cases}
H_0: p_1 = \frac{3}{7}, p_2 = \frac{3}{7}, p_3 = \frac{1}{7} \\
H_1: p_1 \neq \frac{3}{7}, p_2 \neq \frac{3}{7}, p_3 \neq \frac{1}{7} 
\end{cases}
\]

\[	
LR = -2 \cdot (\ell(\hat \theta_R)-\ell(\hat \theta_{UR})) \sim \chi_3^2
\]

\[
\Lambda(x_1, \dots ,x_n;\theta) = \prod\nolimits_{i=1}^n P_\theta (\{X_i = x_i\})= p_1^{40} \cdot p_2^{50} \cdot p_3^{10}
\]

\[
\ell((x_1, \dots ,x_n;\theta)= 40\ln p_1 + 50\ln p_2 + 10\ln p_3
\]

\[
\frac{\delta \ell}{\delta p} = \frac{40}{p_1} + \frac{50}{p_2} + \frac{10}{p_3} 
\]

Таким образом, решая уравнение мы получаем $\hat p_i = \frac{a_i}{n}$

\[
\ell(\hat \theta_R) = 40\ln {\frac{3}{7}} + 50\ln {\frac{3}{7}} + 10\ln {\frac{1}{7}} \approx -95.7
\]

\[
\ell(\hat \theta_{UR}) = 40\ln {\frac{4}{10}} + 50\ln {\frac{5}{10}} + 10\ln {\frac{1}{10}} \approx -94.3
\]

\[
LR_{obs} = -2 \cdot (-1.4) = 2.8
\]

\[
LR_{crit} = 6.25
\]
	
Заметим, что $LR_{crit} > LR_{obs}$, поэтому гипотеза $H_0$ не отвергается.

\item
$\alpha = 0.1$
	
\[
\begin{cases}
H_0: p_1 = \frac{3}{7}, p_2 = \frac{3}{7}, p_3 = \frac{1}{7} \\
H_1: p_1 \neq \frac{3}{7}, p_2 \neq \frac{3}{7}, p_3 \neq \frac{1}{7} 
\end{cases}
\]

\[	
W_n = \sum_{i=1}^r\frac{(\nu_i - n \cdot p_i)^2}{n \cdot p_i}\sim \chi_2^2
\]

\[
W_{obs} = \frac{(40 - 100 \cdot \frac{3}{7})^2}{100 \cdot \frac{3}{7}} + \frac{(50 - 100 \cdot \frac{3}{7})^2}{100 \cdot \frac{3}{7}} + \frac{(10 - 100 \cdot \frac{1}{7})^2}{100 \cdot \frac{1}{7}} \approx 2.58 
\]

\[
W_{crit} = 4.61
\]

Заметим, что $W_{crit} > W_{obs}$, поэтому гипотеза $H_0$ не отвергается.	
\end{enumerate} 


\subsection[2017-2018]{\hyperref[sec:kr_04_2017_2018]{2017-2018}}
\label{sec:sol_kr_04_2017_2018}

\begin{enumerate}
\item Проверяем следующую гипотезу:
\[
\begin{cases}
H_0: \mu_{D} = 100 \\
H_a: \mu_{D} > 100
\end{cases}
\]
Считаем наблюдаемое значение статистики:
\[
t_{obs} = \frac{\bar X - \mu_{D}}{\frac{\sigma_D}{\sqrt{n_D}}} = \frac{136 - 100}{\frac{55}{\sqrt{40}}} \approx 4.14
\]
При верной $H_0$ $t$-статистика имеет распределение $t_{40 - 1}$, значит, $t_{crit} \approx 1.68$.
Поскольку $t_{crit} > t_{obs}$, основная гипотеза отвергается, $p-value \approx 0$.

\item Проверяем следующую гипотезу:
\[
\begin{cases}
H_0: \sigma^2_D = \sigma^2_T \\
H_a: \sigma^2_D \neq \sigma^2_T
\end{cases}
\]
Считаем наблюдаемое значение статистики:
\[
F_{obs} = \frac{\hat{\sigma}^2_D}{\hat{\sigma}^2_T} = \frac{55^2}{60^2} \approx 0.84
\]
При верной $H_0$ $F$-статистика имеет распределение $F_{40-1, 60-1}$.
Находим критические значения: $F_{left} \approx 0.6$, $F_{right} \approx 1.6$.
Поскольку $F_{left} < F_{obs} < F_{right}$, нет оснований отвергать $H_0$.
\item
\begin{enumerate}
Проверяем гипотезу
\[
\begin{cases}
H_0: \mu_{D} = \mu_{T} \\
H_a: \mu_{D} < \mu_{T}
\end{cases}
\]
\item Когда $n_D$, $n_T$ велики,
\[
\frac{\bar D - \bar T - (\mu_D - \mu_T)}{\sqrt{\frac{\sigma^2_D}{n_D} + \frac{\sigma^2_T}{n_T}}} \stackrel{H_0}{\sim} \cN(0, 1)
\]
Считаем наблюдаемое значение статистики:
\[
z_{obs} = \frac{136 - 139}{\sqrt{\frac{3025}{40} + \frac{3600}{60}}} \approx -0.25
\]
По таблице находим $z_{crit} = -1.28$.
Так как $z_{crit} < z_{obs}$, нет оснований отвергать $H_0$.
\item Когда считаем дисперсии одинаковыми, то:
\[
\hat{\sigma}^2_0 = \frac{\hat{\sigma}^2_D (n_D - 1) + \hat{\sigma}^2_T (n_T - 1)}{n_D + n_T - 2} = \frac{3025 \cdot 39 + 3600 \cdot 59}{30 + 60 - 2} \approx 3371
\]
и
\[
\frac{\bar D - \bar T - (\mu_D - \mu_T)}{\hat{\sigma}^2_0\sqrt{\frac{1}{n_D} + \frac{1}{n_T}}} \stackrel{H_0}{\sim} t_{n_D + n_T - 2}
\]
Считаем наблюдаемое значение статистики:
\[
t_{obs} = \frac{136 - 139}{\sqrt{3371}\sqrt{\frac{1}{40} + \frac{1}{60}}} \approx -0.25
\]
По таблице находим критическое значение: $t_{crit} \approx -1.29$.
Поскольку $t_{crit} < t_{obs}$, нет оснований отвергать $H_0$.
\end{enumerate}
\item
\begin{enumerate}
\item Сначала найдём оценку максимального правдоподобия параметра $\lambda$:
\begin{align*}
L &= \prod_{i=1}^n \lambda e^{-\lambda x_i} = \lambda^n e^{-\lambda \sum_{i=1}^n x_i} \\
\ell &= n \ln \lambda - \lambda \sum_{i=1}^n x_i \\
\frac{\partial \ell}{\partial \lambda} &= \left. \frac{n}{\lambda} \right|_{\lambda = \hat \lambda} = 0 \\
\frac{\partial^2 \ell}{\partial \lambda^2} &= -\frac{n}{\lambda^2} \\
\hat \lambda &= \frac{1}{0.52}
\end{align*}
Так как
\[
\frac{\hat \lambda - \lambda}{\sqrt{\frac{1}{I(\lambda)}}} \stackrel{as}{\sim} \cN(0,1),
\]
доверительный интервал имеет вид
\[
\frac{1}{0.52} - 1.96 \frac{1}{\frac{10}{0.52}} < \lambda < \frac{1}{0.52} + 1.96 \frac{1}{\frac{10}{0.52}}
\]
\item Найдём вероятность того, что наушник проработает без сбоев 45 минут:
\[
g(\lambda) = \P(X > 0.75) = 1 - F(0.75) = e^{-0.75\lambda}
\]
Тогда
\begin{align*}
g(\hat \lambda) &= e^{-0.75 / 0.52} \\
g'(\hat \lambda) &= -0.75 e^{-0.75 / 0.52}
\end{align*}
И доверительный интервал имеет вид:
\[
e^{-0.75 / 0.52} - 1.96 \cdot \frac{0.75 \cdot 0.52}{10} \cdot e^{-0.75 / 0.52} < g(\lambda) < e^{-0.75 / 0.52} + 1.96 \cdot \frac{0.75 \cdot 0.52}{10} \cdot e^{-0.75 / 0.52}
\]
\end{enumerate}
\item Выпишем функцию правдоподобия:
\begin{align*}
L &= p_1^{10} \cdot p_2^{10} \cdot p_3^{15} \cdot p_4^{15} \cdot p_5^{25} \cdot (1 - p_1 - p_2 - p_3 - p_4 - p_5)^{25} \\
\ell &= 10 \ln p_1 + 10 \ln p_2 + 15 \ln p_3 + 15 \ln p_4 + 25 \ln p_5 + 25 \ln (1 - p_1 - p_2 - p_3 - p_4 - p_5)
\end{align*}
Максимизируя логарифмическую функцию правдоподобия по всем параметрам,
получим следующие оценки для неограниченной модели:
\begin{align*}
& \hat p_1 = \hat p_2 = 0.1 \\
& \hat p_3 = \hat p_4 = 0.15 \\
& \hat p_5 = 0.25
\end{align*}
Подставив найденные значения в логарифмическую функцию правдоподобия, получим
\[
\ell_{UR} \approx -172
\]
В ограниченной модели $p_1 = \ldots = p_6 = 1/6$, и значение функции правдоподобия
будет
\[
\ell_R \approx -179
\]
Теперь можно посчитать наблюдаемое значение:
\[
LR = 2(\ell_{UR} - \ell_R) = 2(-172 - (-179)) = 14
\]
Критическое значение $\chi_{0.95, 5} = \approx 11 < 14$, значит, основная гипотеза
отвергается.
\end{enumerate}



\subsection[2016-2017]{\hyperref[sec:kr_04_2016_2017]{2016-2017}}
\label{sec:sol_kr_04_2016_2017}


\begin{enumerate}
\item
\begin{enumerate}
\item $t_{obs} = \frac{\bar X - \mu}{\hat\sigma/\sqrt{n}} = \frac{9.5-10}{0.5/\sqrt{100}} = -10$

В таблице для $t_{0.975;100-1}$ находим $-t_{crit} = -1.66$.

Поскольку $t_{obs} < -t_{crit}$, основная гипотеза отвергается.
\item $p-value \approx 0$
\item $X_1, \ldots, X_{100} \sim \cN(\mu, \sigma^2)$
\item $\gamma_{obs} = \frac{\hat{\sigma}^2}{\sigma^2_0}(n-1) =
\frac{0.5^2}{0.3}(100-1) = 82.5$

В таблице находим нужные значения $\chi^2_{0.975;100-1}$, $\gamma_{crit, r} = 128$
и $\chi^2_{0.025;100-1}$, $\gamma_{crit, l} = 73$.

Так $\gamma_{crit, l} < \gamma_{obs} < \gamma_{crit, r}$, нет оснований отвергать $H_0$.
\end{enumerate}
\item $\hat{p} = \frac{16}{40} = 0.4$, проверять будем двустороннюю гипотезу.

$z_{obs} = \frac{\hat{p} - p}{\sqrt{\hat{p}(1-\hat{p})/n}} =
\frac{0.4-0.5}{\sqrt{0.4\cdot0.6/40}} \approx -1.3$

В таблице нормального распределения находим значение $z_{0.975}$, $z_{crit} = 1.96$.

Так как $\vert z_{obs} \vert < z_{crit}$, нет оснований отвергать $H_0$.
\item
\begin{enumerate}
\item $\hat{\sigma}_0^2 = \frac{\hat{\sigma}_\alpha^2(n_\alpha -1) +
\hat{\sigma}_\beta^2(n_\beta-1)}{n_\alpha + n_\beta-2} = \frac{0.25\cdot19 + 0.36\cdot24}{20+25-2} = 0.31$

$\frac{\bar{X} - \bar{Y}}{\hat{\sigma}_0 \sqrt{\frac{1}{n_\alpha} + \frac{1}{n_\beta}}} \sim t_{n_\alpha + n_\beta-2}$

$t_{obs} = \frac{9.5-9.8}{0.56\sqrt{\frac{1}{20}+ \frac{1}{25}}} = -1.79$

$t_{crit} = 2.02$

Поскольку $\vert t_{obs} \vert < t_{crit}$, нет оснований отвергать $H_0$.
\item Выборки независимы, дисперсии неизвестны, но равны,
$X_1, \ldots, X_{n_\alpha} \sim \cN\left(\mu_X, \sigma^2_X\right)$,
$X_1, \ldots, Y_{n_\beta} \sim \cN\left(\mu_Y, \sigma^2_Y\right)$.
\item $\frac{\hat{\sigma}_\alpha^2}{\hat{\sigma}_\beta^2} \sim F_{n_{\alpha-1}, n_{\beta-1}}$

$F_{obs} = \frac{0.5^2}{0.6^2} \approx 0.69$

$F_{crit, 0.975} = 2.35$, $F_{crit, 0.025} = 0.41$

Поскольку $F_{crit, 0.025} < F_{obs} < F_{crit, 0.975}$, нет оснований отвергать $H_0$.
\end{enumerate}
\item $H_0: p = 0.5$, где $p$ — вероятность того, что бутерброд упадёт маслом вниз.

$\hat{p} = \frac{105}{200} = 0.525$

$z_{obs} = \frac{\hat p - p}{\sqrt{\hat p (1 - \hat p)/n}} = \frac{0.525-0.5}{\sqrt{0.525\cdot0.475/200}} \approx 0.7$

$z_{crit} = 1.96$

Так как $z_{obs} < z_{crit}$, нет оснований отвергать $H_0$.
\item $LR \sim \chi^2_1$, так как в основной гипотезе одно уравнение.
Выпишем функцию правдоподобия и найдём $\hat{\mu}_{ML}$ и $\hat{\nu}_{ML}$.
\begin{align*}
L &= \prod_{i=1}^{100} \frac{1}{\sqrt{2\pi\nu}} \exp\left(-\frac{1}{2} \frac{(x_i-\mu)^2}{\nu} \right) = \frac{1}{(\sqrt{2\pi\nu})^{100}} \exp \left(-\frac{1}{2\nu}\sum_{i=1}^{100} (x_i - \mu)^2 \right) \\
\ell &= -\frac{100}{2}\ln (2\pi) - \frac{100}{2} \ln \nu - \frac{1}{2\nu}\sum_{i=1}^{100} (x_i - \mu)^2 \\
\frac{\partial \ell}{\partial \mu} &= \frac{1}{\nu} \sum_{i=1}^{100} (x_i - \mu) \Rightarrow \hat{\mu}_{ML} = \frac{\sum_{i=1}^{100} x_i}{100} = 0.3 \\
\frac{\partial \ell}{\partial \nu} &= -\frac{100}{2\nu} + \frac{1}{2\nu^2}\sum_{i=1}^{100} (x_i - \mu)^2 \Rightarrow \hat{\nu}_{ML} = 1.37
\end{align*}
Тогда $LR=2(\ell(\hat{\mu}_{ML}, \hat{\nu}_{ML}) - \ell(\hat{\mu}_{ML}, \nu = 1))$ имеет вид:
\[
Q_{obs}= 2 \left(-50 \ln(2\pi) - 50 \ln 1.37 - \frac{1}{2\cdot1.37}\cdot 137 + 50 \ln(2\pi) + 0 + \frac{1}{2} \cdot 137 \right) \approx 5.5
\]
Из таблицы: $Q_{crit} = 3.84$. Поскольку $Q_{obs} > Q_{crit}$, основная гипотеза отвергается.
\end{enumerate}




\subsection[2015-2016]{\hyperref[sec:kr_04_2015_2016]{2015-2016}}
\label{sec:sol_kr_04_2015_2016}

\begin{enumerate}
\item[2.]
\begin{enumerate}
\item
\begin{align*}
  L(x, \lambda) &= \prod_{i=1}^{250} e^{-\lambda} \frac{\lambda^{x_i}}{x_i!} = e^{-250\lambda} \lambda^{\sum_{i=1}^{250} x_i} \prod_{i=1}^{250} \frac{1}{x_i!} \\
  \ell(x, \lambda) &= -250\lambda + \ln\lambda \sum_{i=1}^{250} x_i - \sum_{i=1}^{250} \ln x_i! \\
  \frac{\partial \ell}{\partial \lambda} &= -250 + \frac{1}{\lambda} \sum_{i=1}^{250} x_i \\
  \hat{\lambda}_{ML} &= \bar{X}
\end{align*}
\item $\E\left(\hat{\lambda}_{ML}\right) = \E\left(\bar{X}\right) = \lambda \Rightarrow$ оценка несмещённая.

$\Var\left(\hat{\lambda}_{ML}\right) = \Var\left(\bar{X}\right) = \frac{1}{n^2}\cdot n\Var(X_1) = \frac{\lambda}{n} \to_{n \to \infty} 0 \Rightarrow$ оценка состоятельная.

$\frac{\partial^2 \ell}{\partial \lambda^2} = -\frac{1}{\lambda^2} \sum_{i=1}^{n} x_i$,
$I(\lambda) = - \E\left(-\frac{1}{\lambda^2} \sum_{i=1}^{n} x_i\right) = \frac{n}{\lambda}$.
Так как $\Var\left(\hat{\lambda}_{ML}\right) = \frac{1}{I(\lambda)}$, оценка является эффективной.
\item $\P(X=0) = \frac{\lambda^0 e^{-\lambda}}{0!} = e^{-\lambda} \Rightarrow \widehat{\P(X=0)} = e^{-\hat\lambda} = e^{-\bar{X}}$
\item В данном случае: $g\left(\hat{\lambda}\right) = e^{-\hat\lambda}$, $g'\left(\hat\lambda\right) = -e^{-\hat\lambda}$.
И доверительный интервал имеет вид:
\[
  \left[e^{-\bar{X}} - 1.96 \sqrt{\frac{e^{-2\bar{X}}\bar{X}}{n}}; e^{-\bar{X}} + 1.96 \sqrt{\frac{e^{-2\bar{X}}\bar{X}}{n}} \right]
\]
\end{enumerate}


\item[3.]
\begin{enumerate}
\item $\hat p \stackrel{as.}{\sim}\cN\left(p, \frac{p(1-p)}{n}\right)$

О1Р: лекарство помогает в $80\%$ случаев, но в данной выборке оно помогло менее чем 12 людям.

$\alpha = \P(\text{О1Р}) = \P\left(\hat p < \left. \frac{12}{20} \right| p=0.8 \right) = \P \left(\frac{\hat p - 0.8}{\sqrt{\frac{0.8\cdot0.2}{20}}} < \frac{\frac{12}{20} - 0.8}{\sqrt{\frac{0.8\cdot0.2}{20}}} \right) = 0.0125$
\item О2Р: лекарство помогает в $60\%$ случаев, но $Y \geq 12$.

$\hat p \sim \cN\left(0.6, \frac{0.6\cdot0.4}{20} \right)$

$\beta = \P\left( \hat p \geq \frac{12}{20} \right) = \frac{1}{2}$
\item $\P(Z < a) =0.1$, из таблицы находим, что $a=-1.28$.
\[
a = \frac{\frac{c}{20} - 0.8}{\sqrt{\frac{0.8\cdot0.2}{20}}} = -1.28 \Rightarrow c \approx 13.7
\]
\item $\P(\vert \hat p - p \vert \leq 0.01) \geq 0.95$, будем считать, что $p=0.6$.
\[
\P(\vert \hat p - p \vert \leq 0.01) = \P(-0.01 \leq \hat p - p \leq 0.01) = \P\left(-\frac{0.01}{\sqrt{\frac{0.6\cdot0.4}{n}}} \leq Z \leq \frac{0.01}{\sqrt{\frac{0.6\cdot0.4}{n}}} \right) =0.95
\]
Из таблицы находим
\[
\frac{0.01}{\sqrt{\frac{0.6\cdot0.4}{n}}} = 1.96 \Rightarrow n = \frac{0.6\cdot0.4\cdot1.96^2}{0.01^2}
\]
\end{enumerate}

\item[4.] $H_0: p_{\text{c}} = \frac{1}{7}, p_{\text{л}} = \frac{2}{7}, p_{\text{к}} = \frac{4}{7}$

$Q = \sum_{i=1}^{s=3} \frac{(\nu_i - np_i)^2}{np_i} \sim \chi^2_{s-k-1} = \chi^2_2$

$Q_{obs} = \frac{\left(10-50\frac{1}{7}\right)^2}{50\frac{1}{7}} + \frac{\left(1-50\frac{2}{7}\right)^2}{50\frac{2}{7}} + \frac{\left(39-50\frac{4}{7}\right)^2}{50\frac{4}{7}} = 17.29$

$Q_{crit} = 5.99$, $Q_{crit} < Q_{obs} \Rightarrow$ гипотеза отвергается

\item[5.] $LR \sim \chi^2_1$, так как основная гипотеза содержит одно уравнение

$L(x, \lambda) = \prod_{i=1}^{n=50} \lambda e^{-\lambda x} = \lambda^{50} e^{-\lambda \sum_{i=1}^{n=50} x_i}$

$\ln L (x, \lambda) = 50\ln\lambda - \lambda \sum_{i=1}^{n=50} x_i \to \max_\lambda$

$\frac{\partial \ln L}{\partial \lambda} = \frac{50}{\lambda} - \sum_{i=1}^{n=50} x_i \mid_{\lambda=\hat{\lambda}} = 0 \Rightarrow \hat{\lambda}_{ML} = \frac{1}{\bar{X}} = \frac{10}{11}$

При верной $H_0:  \lambda=1$, тогда $\ln L (\lambda=1) = 50 \ln 1 - 1 \cdot 1.1 \cdot 50 = -55$

При верной $H_1: \lambda=\lambda_{ML}$, тогда $\ln L \left(\lambda=\frac{10}{11}\right) = 50 \ln \frac{10}{11} - \frac{10}{11} \cdot 50 \cdot 1.1 = -54.77$

$LR_{obs} = 2(\ln L (H_1) - \ln(H_0)) = 2(-54.77- (-55)) = 0.46$

$LR_{crit} = 2.71$, $LR_{crit} > LR_{obs} \Rightarrow$ оснований отвергать $H_0$ нет

\item[6.]  Будем проверять гипотезы на уровне значимости $0.05$
\begin{enumerate}
\item $\hat{\sigma}^2_{\text{в}} = 484$, $\hat{\sigma}^2_{\text{р}} = 400$

$\frac{\hat{\sigma}^2_{\text{в}} }{\hat{\sigma}^2_{\text{р}}} \sim F_{21-1 , 19-1}$

$F_{obs} = \frac{484}{400} = 1.21$, $F_{crit, left} = 0.4$, $F_{crit, right} = 2.6  \Rightarrow$ оснований отвергать $H_0$ нет

\item $\hat{\sigma}_0^2 = \frac{484 \cdot (21-1) + 400 \cdot (19-1)}{21 + 19 - 2} \approx 444$

$t_{obs} = \frac{78-67}{\sqrt{444} \sqrt{\frac{1}{21}+ \frac{1}{19}}} \approx 1.8 $

$t_{crit}  \sim t_{21+19-2} = t_{38}$, $t_{crit} = \pm 2.02 \Rightarrow$ нет оснований отвергать $H_0$
\end{enumerate}
\item[7.] $\gamma = \sum_{i=1}^s \sum_{j=1}^m \frac{\left(n_{ij} - \frac{n_{i\cdot}n_{\cdot j}}{n}\right)^2}{\frac{n_{i\cdot}n_{\cdot j}}{n}} \sim \chi^2_{(s-1)(m-1)}$

$\gamma_{obs} = \frac{\left(12-\frac{44\cdot48}{100}\right)^2}{\frac{44\cdot48}{100}} + \frac{\left(36-\frac{56\cdot48}{100}\right)^2}{\frac{56\cdot48}{100}} + \frac{\left(32-\frac{44\cdot52}{100}\right)^2}{\frac{44\cdot52}{100}} + \frac{\left(20-\frac{50\cdot52}{100}\right)^2}{\frac{50\cdot52}{100}} \approx 12$

$\gamma_{crit} = 3.84 \Rightarrow$  гипотеза отвергается

\end{enumerate}



\subsection[2014-2015]{\hyperref[sec:kr_04_2014_2015]{2014-2015}}
\label{sec:sol_kr_04_2014_2015}


\begin{enumerate}

\item[1.] \textbf{Задача для первого потока.}

\begin{enumerate}
\item \begin{align*}
-Z_{\frac{\alpha}{2}}<\frac{\hat{p}-p}{\sqrt{\frac{\hat{p}(1-\hat{p})}{n}}}<Z_{\frac{\alpha}{2}} \\
-1.96<\frac{0.4-p}{\sqrt{\frac{0.4\cdot0.6}{40}}}<1.96 \\
-0.56<p<1.36
\end{align*}

\item $H_0$ не отвергается, так как
\[
-0.56<0.5<1.36
\]

\item $H_0$ отвергается, если $p=0.5$ не лежит в построенном доверительном интервале:
\begin{align*}
0.5\geq0.5Z_{\frac{\alpha}{2}}+0.4 \\
0.2\geq Z_{\frac{\alpha}{2}} \Rightarrow \alpha=0.0456
\end{align*}

\end{enumerate}

\item[1.] \textbf{Задача для второго потока.}
\begin{enumerate}
\item
При верной $H_0: \mu=55$.
Находим $t_{crit} = 1.721$ и $t_{obs} = \frac{51-55}{0.45}= -8.9$, $H_0$ отвергается.

\item
При верной $H_0: \frac{\sigma^2_1}{\sigma^2_2}=1$.
Находим $F_{left} = 0.5$, $F_{right}=2.07$ и $F_{obs} = 2/3$, $H_0$ не отвергается.
\end{enumerate}

\item[2.] \textbf{Задача для первого потока.}

При верной $H_0: \mu_{\alpha}=\mu_{\beta}$.
Находим $t_{crit} = 2.048$ и $t_{obs} =\frac{9.5-9.8}{0.216}= -1.4$, $H_0$ не отвергается.

\item[2.] \textbf{Задача для второго потока.}
\begin{enumerate}
\item
\begin{align*}
\hat{\sigma}^2_1 &= 0.2 \cdot 0.8 = 0.16 \\
\hat{\sigma}^2_2 &= 0.17 \cdot 0.83 = 0.1411
\end{align*}

При верной $H_0: \mu_1=\mu_2$.
Находим $Z_{crit} = 2.58$ и $t_{obs} =\frac{0.2-0.17}{0.06}= 0.5$, $H_0$ не отвергается.
\item
\[
p_{value} = 1 - F(0.5) \approx 1 - 0.6915 = 0.3085
\]
\end{enumerate}

\item[3.]
\begin{align*}
&L(x, \theta) = \theta^{-2n} x^n \exp \left(-\frac{1}{\theta}\sum_ix_i\right) \\
&\ell=\ln(L)=-2n\ln(\theta)+n\ln(x)-\frac{1}{\theta}\sum_ix_i \\
&\ell'_{\theta}=-\frac{2n}{\theta} + \frac{1}{\theta^2}\sum_ix_i \\
&-2n\frac{1}{\hat{\theta}}\sum_ix_i=0 \\
&\hat{\theta}_{ML}=\frac{1}{2}\bar{X} \\
\end{align*}


\item[4.]
Пусть $\ell=\ln(L)$ — логарифмическая функция правдоподобия.
\[
\ell=-\frac{n}{2}\ln(2\pi)-\frac{n}{2}\ln(\nu)-\frac{1}{2\nu}\sum_i(x_i-\bar{x})^2
\]
Известно, что для выборки из нормального распределения
$\hat{\mu}_{ML} = \bar{x} = 0.3$, $\hat{\nu}_{ML} = S^2 = 1.37$, поэтому
\begin{align*}
&\ell_{UR} = -50\ln(2\pi) - 50\ln(1.37)-\frac{137}{2 \cdot 1.37} \\
&\ell_{R} = -50\ln(2\pi) - \frac{137}{2} \\
& LR_{obs}=2\left(-50\ln(2\pi)-50\ln(1.37)-\frac{137}{2\cdot1.37}+50\ln(2\pi)+\frac{137}{2}\right) \approx 12
\end{align*}
При верной $H_0$ $LR\sim\chi^2_1 \Rightarrow LR_{crit}\approx3.8$, основная гипотеза отвергается.

\item[5.] \textbf{Исследовательская задача.}
\begin{enumerate}
\item
Пусть $\ell=\ln(L)$ — логарифмическая функция правдоподобия.
Воспользуемся также тем, что для выборки из нормального распределения
$\hat{\mu}_{ML}=\bar{X}$, $\hat{\nu}_{ML}=S^2$.
\begin{align*}
\ell &= -\frac{n}{2}\ln(2\pi) - \frac{n}{2}\ln(\nu) - \frac{1}{2\nu}\sum^n_{i=1}(x_i-\bar{x})^2 \\
\Var\left(\ell'(\hat{\mu})\right) &= \frac{1}{\nu^2} \Var\left(\sum^n_{i=1}(X_i-\bar{X})\right) = \frac{n}{\nu^2}\Var(X_1) = \frac{n}{\nu} \\
LM &= \frac{\left(\ell'(\hat{\mu}) - \ell'(0)\right)^2}{\Var(\ell'(\hat{\mu}))} = \frac{\left(\frac{1}{\nu}\sum^n_{i=1}(X_i-\bar{X})-\frac{1}{\nu}\sum^n_{i=1}X_i\right)^2}{\frac{n}{\nu}} = \frac{n}{\nu}\bar{X}^2 \\
W &= \frac{(\hat{\mu}-0)^2}{\Var(\hat{\mu})} = \frac{\left(\frac{\sum^n_{i=1}X_i}{n}\right)^2}{\frac{\nu}{n}}=\frac{n}{\nu}\bar{X}^2 \\
\end{align*}
\begin{align*}
LR &= 2\left(-\frac{1}{2\nu}\sum^n_{i=1}(X_i-\bar{X}) + \frac{1}{2\nu}\sum^n_{i=1}X_i\right) = \\
&= \frac{1}{\nu}\left(\sum^n_{i=1}X^2_i-\sum^n_{i=1}X^2_i+2\bar{X} \cdot \sum^n_{i=1}X_i-n\bar{X}^2\right) = \\
&= \frac{1}{\nu} \left(2\bar{X}\cdot n\cdot\bar{X}-n\bar{X}^2\right)=\frac{n}{\nu}\bar{X}^2
\end{align*}
Как видим, $LR=LM=W$.
\item
\end{enumerate}

\item[6.] \textbf{Исследовательская задача.}

\begin{enumerate}

\item
\begin{align*}
\E(X) &= \int^{+\infty}_{0}a^2x^2e^{-ax} dx = -ax^2e^{-ax}\bigg|^{+\infty}_0 + 2\int^{+\infty}_{0}axe^{-ax} dx= \\
&= -2xe^{-ax}\bigg|^{+\infty}_0 + 2\int^{+\infty}_{0}e^{-ax} dx = -\frac{2}{a}e^{-ax}\bigg|^{+\infty}_0=\frac{2}{a} \\
\frac{2}{\hat{a}_{MM}} &= \frac{\sum^{n}_{i=1}x_i}{n} \Rightarrow \hat{a}_{MM}=\frac{2}{3}
\end{align*}

\item
\[
\hat{a}_{MM}=\frac{2}{\bar{X}}
\]

Разложим $\hat{a}_{MM}$ по формуле Тейлора в окрестности точки $\bar{X}=3$:
\begin{align*}
\hat{a}_{MM} &\approx \frac{2}{3} - \frac{2}{9}\left(\bar{X}-3\right) \\
\widehat{\Var}(X) &= \frac{\sum^n_{i=1}(X_i-\bar{X})^2}{n}=\frac{\sum^n_{i=1}X_i^2-2\bar{X}\sum^n_{i=1}X_i+n\bar{X}^2}{n} \\
&= \frac{1000 - 6\cdot300 + 100\cdot9}{100} = 1 \\
\Var\left(\hat{a}_{MM}\right) &\approx \frac{4}{81}\Var\left(\bar{X}\right) = \frac{4\cdot100}{81\cdot100^2} \Var(X)\Rightarrow \widehat{\Var}\left(\hat{a}_{MM}\right) = \frac{4}{8100}
\end{align*}
\item
\begin{align*}
-t_{\frac{\alpha}{2};n-1}<\frac{\bar{X}-\mu}{\frac{\hat{\sigma}}{\sqrt{n}}}<t_{\frac{\alpha}{2};n-1} \\
-1.98<\frac{3-\mu}{\frac{1}{10}}<1.98
\end{align*}
Воспользуемся тем, что $\mu=\frac{2}{a}$:

\begin{align*}
-1.98<\frac{3-\frac{2}{a}}{\frac{1}{10}}<1.98 \\
-0.198<3-\frac{2}{a}<0.198 \\
0.63<a<0.71
\end{align*}

\end{enumerate}
\end{enumerate}


\subsection[2009-2010]{\hyperref[sec:kr_04_2009_2010]{2009-2010}}
\label{sec:sol_kr_04_2009_2010}
