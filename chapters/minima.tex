% !TEX root = ../probability_hse_exams.tex
\newpage
\thispagestyle{empty}
\section{Минимумы}

\subsection{Контрольная работа 1}


\subsubsection*{Теоретический минимум}


\begin{enumerate}
	\item Сформулируйте классическое определение вероятности.
	\item Выпишите формулу условной вероятности.
	\item Дайте определение независимости (попарной и в совокупности) для $n$ случайных событий.
	\item Выпишите формулу полной вероятности, указав условия её применимости.
	\item Выпишите формулу Байеса, указав условия её применимости.
	\item Дайте определение функции распределения $F_{X}(x)$ случайной величины $X$. 
	Выпишите свойства функции $F_{X}(x)$. 
	% Запишите пример функции распределения для дискретной и для непрерывной случайных величин.
	\item Дайте определение функции плотности $f_{X}(x)$ случайной величины $X$. 
	Выпишите свойства функции $f_{X}(x)$.
	\item Дайте определение математического ожидания для дискретных и абсолютно непрерывных случайных величин. 
	Укажите, чему равно $\E(\alpha X+\beta Y)$, где $X$ и $Y$ — случайные величины, а $\alpha$ и $\beta$ — произвольные константы.
	\item Дайте определение дисперсии случайной величины. 
	Укажите, чему равно $\Var(\alpha X+\beta)$, где $X$ — случайная величина, 
	а $\alpha$ и $\beta$ — произвольные константы.
	\item Укажите математическое ожидание, дисперсию, множество значений, 
	принимаемых с ненулевой вероятностью, а также функцию плотности или функцию вероятности для случайной величины $X$, 
	имеющей следующее распределение:
	\begin{enumerate}
	\item биномиальное;
	\item Пуассона;
	\item геометрическое;
	\item равномерное;
	\item экспоненциальное.
	\end{enumerate}
\end{enumerate}

\newpage
\subsubsection*{\hyperref[sec:sol_minimum_kr_01]{Задачный минимум}}\label{sec:minimum_kr_01}

% \textbf{Важно}: на контрольной работе, в каждой из задач минимума могут меняться любые числа, а также некоторые иные параметры и условия, указанные в примечаниях к задачам.

\textbf{Важно:} на контрольной работе каждая из задач минимума может несущественно изменяться!

\begin{enumerate}
\item  Пусть $\P(A) = 0.3, \P(B) = 0.4, \P(A\cap B) = 0.1$.
	\begin{enumerate}
		\item  Найдите $\P(A|B)$;
		\item  Найдите $\P(A\cup B)$;
		\item  Являются ли события $A$ и $B$ независимыми?
	\end{enumerate}

% \textbf{Примечание}: могут изменяться известные и незивестные вероятности. Например, в задаче выше требуется найти $\P(A\cup B)$ при известной $\P(A\cap B)$, а может потребоваться найти $\P(A\cap B)$ при известной $\P(A\cup B)$. При этом количество рассматриваемых событий, а также отношения между ними, остаются неизменным: в задаче будут фигурировать только события $A$ и $B$ и отношения условия, объединения и пересечения.

\item  Карлсон выложил кубиками слово КОМБИНАТОРИКА.
Малыш выбирает наугад четыре кубика и выкладывает их в случайном порядке.
Найдите вероятность того, что при этом получится слово КОРТ.

% \textbf{Примечание}: может изменяться как слово, выложенное Карлсоном, так и слово, вероятность получить которое требуется найти.

\item  В первой урне 7 белых и 3 черных шара, во второй урне 8 белых и 4 черных
шара, в третьей урне 2 белых и 13 черных шаров.

Из этих урн наугад выбирается одна урна.
\begin{enumerate}
\item  Вычислите вероятность того, что шар, взятый наугад из выбранной урны, окажется белым.
\item  Посчитайте вероятность того, что была выбрана первая урна,
если шар, взятый наугад из выбранной урны, оказался белым.
\end{enumerate}

% \textbf{Примечание}: может изменяться количество урн.

\item  В операционном отделе банка работает 80\% опытных сотрудников и 20\%
неопытных. Вероятность совершения ошибки при очередной банковской операции
опытным сотрудником равна 0.01, а неопытным — 0.1.

\begin{enumerate}
\item  Найдите вероятность совершения ошибки при очередной банковской операции в этом отделе.
\item  Сотрудником банка была совершена ошибка. Найдите вероятность того, 
что ошибку допустил неопытный сотрудник.
\end{enumerate}

\item  Пусть случайная величина $X$ имеет таблицу распределения:

\begin{tabular}{lccc}
	\toprule
	$x$ & $-1$  & $0$  & $1$ \\
	$\P(X = x)$ & $0.25$  & $c$  & $0.25$ \\
  \bottomrule
\end{tabular}

	\begin{enumerate}
	\item Найдите константу $c$.
	\item Найдите $\P(X \geq 0)$.
	\item Найдите $\P(X < -3)$.
	\item Найдите $\P\left(X \in \left[-\frac{1}{2}; \frac{1}{2}\right]\right)$.
	\item Выпишите функцию распределения случайной величины $X$ 
	и постройте её график. 
	\item Имеет ли случайная величина $X$ функцию плотности?
	\end{enumerate}
	
%\textbf{Примечание} может изменяться закон распределения величины $X$. 
% Например, случайная величины $X$ может принимать с ненулевой вероятностью не три, а пять значений.

\item  Пусть случайная величина $X$ имеет таблицу распределения: % задача 10

\begin{tabular}{lccc}
\toprule
$x$ & $-1$  & $0$  & $1$ \\
$\P(X = x)$ & $0.25$ & $c$  & $0.25$ \\
\bottomrule
\end{tabular}

Найдите
\begin{enumerate}
	\item константу $c$
	\item $\E(X)$
	\item $\E\left(X^2\right)$
	\item $\Var(X)$
	\item $\E(|X|)$
\end{enumerate}

% \textbf{Примечание} может изменяться закон распределения величины $X$. 
% Также, вместо $\E(|X|)$ может потребоваться найти $\E(\alpha X^{\beta})$, $\E(\alpha\ln(X))$ 
% или $\E(\alpha e^{\beta X})$, где $\alpha$ и $\beta$ — заранее известные константы.

\item Пусть случайная величина $X$ имеет биномиальное распределение с
параметрами $n = 4$ и $p = 0.75$.
 Найдите
\begin{enumerate}
	\item $\P(X = 0)$;
	\item $\P(X > 0)$;
	\item $\P(X < 0)$;
	\item $\E(X)$;
	\item $\Var(X)$;
	\item наиболее вероятное значение, которое принимает случайная величина $X$.
\end{enumerate}

\item  Пусть случайная величина $X$ имеет распределение Пуассона с параметром $\lambda = 100$.
Найдите
\begin{enumerate}
	\item $\P(X = 0)$;
	\item $\P(X > 0)$;
	\item $\P(X < 0)$;
	\item $\E(X)$;
	\item $\Var(X)$;
	\item наиболее вероятное значение, которое принимает случайная величина $X$.
\end{enumerate}

\item В лифт 10-этажного дома на первом этаже вошли 5 человек.
Они выходят на каждом этаже начиная со второго равновероятно и независимо
друг от друга.
\begin{enumerate}
	\item Вычислите вероятность того, что на 6-м этаже выйдет хотя бы один человек.
    \item Вычислите вероятность того, что на 6-м этаже не выйдет ни одного человека.
    \item Вычислите вероятность того, что все выйдут на 6-м этаже или выше. 
    \item Вычислите вероятность того, что никто не выйдет на 6-м этаже или выше. 
\end{enumerate}

% \textbf{Примечание}: может изменяться в том числе количество людей, вероятность выхода на этаже которых следует найти. Например, может спрашиваться вероятность того, что на $6$-м этаже выйдет не более двух человек.

\item При работе некоторого устройства время от времени возникают сбои.
Количество сбоев за сутки имеет распределение Пуассона и не зависет от количество сбоев в любые другие сутки. Среднее количество сбоев за сутки равно 3.

\begin{enumerate}
	\item Найдите вероятность того, что в течение суток произойдет хотя бы один сбой.
	\item Вычислите вероятность того, что за \textbf{двое} суток не произойдет ни одного сбоя.
\end{enumerate}

\item Пусть случайная величина $X$ имеет следующую функцию плотности:

\[
f_X(x) =
	\begin{cases}
	cx,\text{ при }  x \in [0; 1] \\
	0,\text{ при } x \notin  [0; 1] \\
	\end{cases}
\]

Найдите
\begin{enumerate}
	\item константу $c$
	\item $\P\left(X \leq \frac{1}{2}\right)$
	\item $\P\left(X \in \left[\frac{1}{2}; \frac{3}{2}\right]\right)$
	\item $\P(X \in [2;3])$
	\item $F_X(x)$
\end{enumerate}

% \textbf{Примечание}: вместо функции $cx$ могут фигурировать функции $cx^{\alpha}+\beta$, $c\ln(x)+\beta$ и $ce^{x}+\beta$, где $\alpha$ и $\beta$ — заранее известные константы.

\item Пусть случайная величина $X$ имеет следующую функцию плотности:

\[
f_X(x) =
	\begin{cases}
	cx,\text{ при }  x \in [0; 1] \\
	0,\text{ при } x \notin  [0; 1] \\
	\end{cases}
\]

Найдите
\begin{enumerate}
	\item константу $c$;
	\item $\E(X)$;
	\item $\E\left(X^2\right)$;
	\item $\Var(X)$;
	\item $\E(\sqrt{X})$.
\end{enumerate}

% \textbf{Примечание}: вместо функции $cx$ могут фигурировать функции $cx^{\alpha}+\beta$, $c\ln(x)+\beta$ и $ce^{x}+\beta$, где $\alpha$ и $\beta$ — заранее известные константы. Также, вместо $\E(\sqrt{X})$ может потребоваться найти $\E(\alpha X^{\beta})$, $\E(\alpha\ln(X))$ или $\E(\alpha e^{\beta X})$, где $\alpha$ и $\beta$ — заранее известные константы.


\end{enumerate}

\newpage
\subsection{Контрольная работа 2}
\label{sec:minimum_kr_02}

\subsubsection*{Теоретический минимум}

Первые девять вопросов второго минимума в точности дублируют вопросы 
первого минимума. 
В десятом вопросе про распределения появляется нормальное распределение. 

\begin{enumerate}
	\item Сформулируйте классическое определение вероятности.
	\item Выпишите формулу условной вероятности.
	\item Дайте определение независимости (попарной и в совокупности) для $n$ случайных событий.
	\item Выпишите формулу полной вероятности, указав условия её применимости.
	\item Выпишите формулу Байеса, указав условия её применимости.
	\item Дайте определение функции распределения $F_{X}(x)$ случайной величины $X$. 
	Выпишите свойства функции $F_{X}(x)$. 
	% Запишите пример функции распределения для дискретной и для непрерывной случайных величин.
	\item Дайте определение функции плотности $f_{X}(x)$ случайной величины $X$. 
	Выпишите свойства функции $f_{X}(x)$.
	\item Дайте определение математического ожидания для дискретных и абсолютно непрерывных случайных величин. 
	Укажите, чему равно $\E(\alpha X+\beta Y)$, где $X$ и $Y$ — случайные величины, а $\alpha$ и $\beta$ — произвольные константы.
	\item Дайте определение дисперсии случайной величины. 
	Укажите, чему равно $\Var(\alpha X+\beta)$, где $X$ — случайная величина, 
	а $\alpha$ и $\beta$ — произвольные константы.
	\item Укажите математическое ожидание, дисперсию, множество значений, 
	принимаемых с ненулевой вероятностью, а также функцию плотности или функцию вероятности для случайной величины $X$, 
	имеющей следующее распределение:
	\begin{enumerate}
	\item биномиальное;
	\item Пуассона;
	\item геометрическое;
	\item равномерное;
	\item экспоненциальное;
	\item нормальное.
	\end{enumerate}

\item Сформулируйте определение функции совместного распределения двух случайных величин, 
независимости случайных величин. 
Укажите, как связаны совместное распределение и частные распределения компонент случайного вектора.
\item Сформулируйте определение и свойства совместной функции плотности двух случайных величин, сформулируйте определение независимости случайных величин.
\item Сформулируйте определение и свойства ковариации случайных величин.
\item Сформулируйте определение и свойства корреляции случайных величин.
\item Сформулируйте определение и свойства условной функции плотности.
\item Сформулируйте определение условного математического ожидания $\E(Y|X=x)$ для совместного дискретного и совместного абсолютно непрерывного распределений.
\item Сформулируйте определение математического ожидания и ковариационной матрицы случайного вектора и их свойства.
\item Сформулируйте неравенство Чебышёва и неравенство Маркова.
\item Сформулируйте закон больших чисел в слабой форме.
\item Сформулируйте центральную предельную теорему.
\item Сформулируйте теорему Муавра—Лапласа.
\item Сформулируйте определение сходимости по вероятности для последовательности случайных величин.
\end{enumerate}

\newpage
\subsubsection*{\hyperref[sec:sol_minimum_kr_02]{Задачный минимум}}\label{sec:minimum_kr_02}

\begin{enumerate}
\item Пусть задана таблица совместного распределения случайных величин $X$ и $Y$.

\begin{center}
\begin{tabular}{lccc}
\toprule
                       & $Y=-1$  & $Y=0$   & $Y=1$   \\ \midrule
$X=-1$                 & $0.2$ & $0.1$ & $0.2$ \\
 $X=1$                 & $0.1$ & $0.3$ & $0.1$ \\ \bottomrule
\end{tabular}
\end{center}

Найдите
\begin{enumerate}
\item $\P(X = -1)$
\item $\P(Y = -1)$
\item $\P(X = -1 \cap Y = -1 )$
\item Являются ли случайные величины $X$ и $Y$ независимыми?
\item $F_{X,Y}(-1,0)$
\item Таблицу распределения случайной величины $X$
\item Функцию $F_{X}(x)$ распределения случайной величины $X$
\item Постройте график функции $F_{X}(x)$ распределения случайной величины $X$
\end{enumerate}

\item Пусть задана таблица совместного распределения случайных величин $X$ и $Y$.

\begin{center}
\begin{tabular}{lccc}
\toprule
                       & $Y=-1$  & $Y=0$  & $Y=1$   \\ \midrule
$X=-1$                 & $0.2$ & $0.1$ & $0.2$ \\
 $X=1$                 & $0.2$ & $0.1$ & $0.2$ \\ \bottomrule
\end{tabular}
\end{center}

Найдите
\begin{enumerate}
\item $\P(X = 1)$
\item $\P(Y = 1)$
\item $\P(X = 1 \cap Y = 1)$
\item Являются ли случайные величины $X$ и $Y$ независимыми?
\item $F_{X,Y}(1,0)$
\item Таблицу распределения случайной величины $Y$
\item Функцию $F_{Y}(y)$ распределения случайной величины $Y$
\item Постройте график функции $F_{Y}(y)$ распределения случайной величины $Y$
\end{enumerate}

\item Пусть задана таблица совместного распределения случайных величин $X$ и $Y$.

\begin{center}
\begin{tabular}{lccc}
\toprule
                       & $Y=-1$  & $Y=0$   & $Y=1$   \\ \midrule
$X=-1$                 & $0.2$ & $0.1$ & $0.2$ \\
 $X=1$                 & $0.1$ & $0.3$ & $0.1$ \\ \bottomrule
\end{tabular}
\end{center}

Найдите
\begin{enumerate}
\item $\E(X)$
\item $\E(X^{2})$
\item $\Var(X)$
\item $\E(Y)$
\item $\E(Y^{2})$
\item $\Var(Y)$
\item $\E(XY)$
\item $\Cov(X,Y)$
\item $\Corr(X,Y)$
\item Являются ли случайные величины $X$ и $Y$ некоррелированными?
\end{enumerate}

\item Пусть задана таблица совместного распределения случайных величин $X$ и $Y$.

\begin{center}
\begin{tabular}{lccc}
\toprule
                       & $Y=-1$  & $Y=0$  & $Y=1$  \\ \midrule
$X=-1$                 & $0.2$ & $0.1$ & $0.2$ \\
 $X=1$                 & $0.2$ & $0.1$ & $0.2$ \\ \bottomrule
\end{tabular}
\end{center}

Найдите
\begin{enumerate}
\item $\E(X)$
\item $\E(X^{2})$
\item $\Var(X)$
\item $\E(Y)$
\item $\E(Y^{2})$
\item $\Var(Y)$
\item $\E(XY)$
\item $\Cov(X,Y)$
\item $\Corr(X,Y)$
\item Являются ли случайные величины $X$ и $Y$ некоррелированными?
\end{enumerate}

\item
Пусть задана таблица совместного распределения случайных величин $X$ и $Y$.

\begin{center}
\begin{tabular}{lccc}
\toprule
                       & $Y=-1$  & $Y=0$   & $Y=1$   \\ \midrule
$X=-1$                 & $0.2$ & $0.1$ & $0.2$ \\
 $X=1$                 & $0.1$ & $0.3$ & $0.1$ \\ \bottomrule
\end{tabular}
\end{center}

Найдите
\begin{enumerate}
\item $\P(X = -1 | Y = 0)$
\item $\P(Y = 0 | X = -1)$
\item Таблицу условного распределения случайной величины $Y$ при условии $X = -1$
\item Условное математическое ожидание случайной величины $Y$ при $X = -1$
\item Условную дисперсию случайной величины $Y$ при условии $X = -1$
\end{enumerate}

\item Пусть задана таблица совместного распределения случайных величин $X$ и $Y$.

\begin{center}
\begin{tabular}{lccc}
\toprule
                   & $Y=-1$  & $Y=0$   & $Y=1$   \\ \midrule
$X=-1$                 & $0.2$ & $0.1$ & $0.2$ \\
 $X=1$                 & $0.2$ & $0.1$ & $0.2$ \\ \bottomrule
\end{tabular}
\end{center}

Найдите
\begin{enumerate}
\item $\P(X = 1 | Y = 0)$
\item $\P(Y = 0 | X = 1)$
\item Таблицу условного распределения случайной величины $Y$ при условии $X = 1$
\item Условное математическое ожидание случайной величины $Y$ при $X = 1$
\item Условную дисперсию случайной величины $Y$ при условии $X = 1$
\end{enumerate}

\item Пусть $\E(X)=1$, $\E(Y)=2$, $\Var(X) = 3$, $\Var(Y) = 4$, $\Cov(X,Y) = -1$. Найдите
\begin{enumerate}
\item $\E(2X + Y - 4)$
\item $\Var(3Y + 3)$
\item $\Var(X - Y)$
\item $\Var(2X - 3Y +1)$
\item $\Cov(X+ 2Y + 1,3X - Y -1)$
\item $\Corr(X + Y, X - Y)$
\item Ковариационную матрицу случайного вектора $Z = (X \qquad Y)$
\end{enumerate}


\item Пусть $\E(X)=-1$, $\E(Y)=2$, $\Var(X) = 1$, $\Var(Y) = 2$, $\Cov(X,Y) = 1$. Найдите
\begin{enumerate}
\item $\E(2X + Y - 4)$
\item $\Var(2Y + 3)$
\item $\Var(X - Y)$
\item $\Var(2X - 3Y +1)$
\item $\Cov(3X+ Y + 1,X - 2Y -1)$
\item $\Corr(X + Y, X - Y)$
\item Ковариационную матрицу случайного вектора $Z = (X \qquad Y)$
\end{enumerate}

\item Пусть случайная величина $X$ имеет стандартное нормальное распределение.
Найдите
\begin{enumerate}
\item $\P(0 < X < 1)$
\item $\P(X > 2)$
\item $\P(0 < 1 - 2X \leq 1)$
\end{enumerate}

\item Пусть случайная величина $X$ имеет стандартное нормальное распределение.
Найдите
\begin{enumerate}
\item $\P(-1 < X < 1)$
\item $\P(X < -2)$
\item $\P(-2 < -X + 1 \leq 0)$
\end{enumerate}

\item Пусть случайная величина $X \sim \cN(1,4)$. Найдите $\P(1<X<4)$.

\item Пусть случайная величина $X \sim \cN(2,4)$. Найдите $\P(-2<X<4)$.

\item Случайные величины $X$ и $Y$ независимы и  имеют нормальное распределение,
$\E(X) = 0 $, $\Var(X) = 1$, $\E(Y) = 2$, $\Var(Y) = 6$. Найдите $\P(1 < X + 2Y < 7)$.

\item Случайные величины $X$ и $Y$ независимы и  имеют нормальное распределение,
$\E(X) = 0 $, $\Var(X) = 1$, $\E(Y) = 3$, $\Var(Y) = 7$. Найдите $\P(1 < 3X + Y < 7)$.

\item Игральная кость подбрасывается $420$ раз.
При помощи центральной предельной теоремы приближенно найти вероятность того,
что суммарное число очков будет находиться в пределах от $1400$ до $1505$?

\item При выстреле по мишени стрелок попадает в десятку с вероятностью $0.5$,
в девятку – $0.3$, в восьмерку – $0.1$, в семерку – $0.05$, в шестерку – $0.05$.
Стрелок сделал $100$ выстрелов.
При помощи центральной предельной теоремы приближенно найти вероятность того,
что он набрал не менее $900$ очков?

\item Предположим, что на станцию скорой помощи поступают вызовы,
число которых распределено по закону Пуассона с параметром $\lambda = 73$,
и в разные сутки их количество не зависит друг от друга.
При помощи центральной предельной теоремы приближенно найти вероятность того,
что в течение года (365 дней) общее число вызовов будет в пределах от $26500$ до $26800$.

\item Число посетителей магазина (в день) имеет распределение Пуассона
с математическим ожиданием $289$.
При помощи центральной предельной теоремы приближенно найти вероятность того,
что за $100$ рабочих дней суммарное число посетителей составит
от $28550$ до $29250$ человек.

\item Пусть плотность распределения случайного вектора $(X,Y)$ имеет вид
\[
f_{X,Y}(x,y) =
\begin{cases}
x+y, & \text{при } (x,y) \in [0;1] \times [0;1] \\
0 , & \text{при } (x,y) \not\in [0;1] \times [0;1]
\end{cases}
\]
Найдите
\begin{enumerate}
\item $\P(X \leq \frac{1}{2} \cap Y \leq \frac{1}{2})$,
\item $\P(X\leq Y)$,
\item $f_{X}(x)$,
\item $f_{Y}(y)$,
\item Являются ли случайные величины $X$ и $Y$ независимыми?
\end{enumerate}

\item Пусть плотность распределения случайного вектора $(X,Y)$ имеет вид
\[
f_{X,Y}(x,y) =
\begin{cases} 4xy, & \text{при } (x,y) \in [0;1] \times [0;1] \\
0 , & \text{при } (x,y) \not\in [0;1] \times [0;1]
\end{cases}
\]
Найдите
\begin{enumerate}
\item $\P(X \leq \frac{1}{2} \cap Y \leq \frac{1}{2})$,
\item $\P(X\leq Y)$,
\item $f_{X}(x)$,
\item $f_{Y}(y)$,
\item Являются ли случайные величины $X$ и $Y$ независимыми?
\end{enumerate}

\item Пусть плотность распределения случайного вектора $(X,Y)$ имеет вид
\[
f_{X,Y}(x,y) =
\begin{cases} x+y, & \text{при } (x,y) \in [0;1] \times [0;1] \\
0 , & \text{при } (x,y) \not\in [0;1] \times [0;1]
\end{cases}
\]
Найдите
\begin{enumerate}
\item $\E(X)$,
\item $\E(Y)$,
\item $\E(XY)$,
\item $\Cov(X,Y)$,
\item $\Corr(X,Y)$.
\end{enumerate}

\item Пусть плотность распределения случайного вектора $(X,Y)$ имеет вид
\[
f_{X,Y}(x,y) =
\begin{cases}
4xy, & \text{при } (x,y) \in [0;1] \times [0;1] \\
0 , & \text{при } (x,y) \not\in [0;1] \times [0;1]
\end{cases}
\]
Найдите
\begin{enumerate}
\item $\E(X)$,
\item $\E(Y)$,
\item $\E(XY)$,
\item $\Cov(X,Y)$,
\item $\Corr(X,Y)$.
\end{enumerate}

\item Пусть плотность распределения случайного вектора $(X,Y)$ имеет вид
\[
f_{X,Y}(x,y) =
\begin{cases}
x+y, & \text{при } (x,y) \in [0;1] \times [0;1] \\
0 , & \text{при } (x,y) \not\in [0;1] \times [0;1]
\end{cases}
\]
Найдите
\begin{enumerate}
\item $f_{Y}(y)$,
\item $f_{X|Y}\left(x|\frac{1}{2}\right)$
\item $\E\left(X|Y = \frac{1}{2}\right)$
\item $\Var\left(X|Y = \frac{1}{2}\right)$
\end{enumerate}

\item Пусть плотность распределения случайного вектора $(X,Y)$ имеет вид
\[
f_{X,Y}(x,y) =
\begin{cases}
4xy, & \text{при } (x,y) \in [0;1] \times [0;1] \\
0 , & \text{при } (x,y) \not\in [0;1] \times [0;1]
\end{cases}
\]
Найдите
\begin{enumerate}
\item $f_{Y}(y)$,
\item $f_{X|Y}\left(x|\frac{1}{2}\right)$
\item $\E\left(X|Y = \frac{1}{2}\right)$
\item $\Var\left(X|Y = \frac{1}{2}\right)$
\end{enumerate}
\end{enumerate}

\newpage
\subsection{Контрольная работа 3}

\subsubsection*{Теоретический минимум}

\begin{enumerate}
  \item Дайте определение нормально распределённой случайной величины.
	Укажите диапазон возможных значений, функцию плотности, ожидание, дисперсию.
	Нарисуйте функцию плотности.
  \item Дайте определение хи-квадрат распределения с помощью нормальных распределений.
	Укажите диапазон возможных значений,
	математическое ожидание.
	Нарисуйте функцию плотности при разных степенях свободы.
  \item Дайте определение распределения Стьюдента с помощью нормальных распределений.
	Укажите диапазон возможных значений.
	Нарисуйте функцию плотности распределения Стьюдента при разных степенях свободы
	на фоне нормальной стандартной функции плотности.
  \item Дайте определение распределения Фишера с помощью нормальных распределений.
	Укажите диапазон возможных значений.
	Нарисуйте возможную функцию плотности.
\end{enumerate}

Для следующего блока вопросов предполагается, что
имеется случайная выборка $X_1$, $X_2$, \ldots, $X_n$ из распределения
с функцией плотности $f(x, \theta)$, зависящей от от параметра $\theta$.
Дайте определение каждого понятия из списка или сформулируйте соответствующую теорему:

\begin{enumerate}[resume]
  \item Выборочное среднее и выборочная дисперсия;
  \item Формула несмещённой оценки дисперсии;
  \item Выборочный начальный момент порядка $k$;
  \item Выборочный центральный момент порядка $k$;
  \item Выборочная функция распределения;
  \item Несмещённая оценка $\hat \theta$ параметра $\theta$;
  \item Состоятельная последовательность оценок $\hat \theta_n$;
  \item Эффективность оценки $\hat \theta$ среди множества оценок $\hat \Theta$;
  \item Неравенство Крамера–Рао для несмещённых оценок;
  \item Функция правдоподобия и логарифмическая функция правдоподобия;
  \item Информация Фишера о параметре $\theta$, содержащаяся в одном наблюдении;
  \item Оценка метода моментов параметра $\theta$ при использовании первого момента,
	если $\E(X_i)=g(\theta)$ и существует обратная функция $g^{-1}$;
  \item Оценка метода максимального правдоподобия параметра $\theta$;
\end{enumerate}

Для следующего блока вопросов предполагается, что величины $X_1$, $X_2$, \ldots, $X_n$ независимы и нормальны $\cN(\mu;\sigma^2)$.

\begin{enumerate}[resume]
  \item Укажите закон распределения выборочного среднего,
	величины $\frac{\bar X - \mu}{\sigma/\sqrt{n}}$,
	величины $\frac{\bar X - \mu}{\hat\sigma/\sqrt{n}}$,
	величины $\frac{\hat\sigma^2(n-1)}{\sigma^2}$;
  \item Укажите формулу доверительного интервала с уровнем доверия
	$(1-\alpha)$ для $\mu$ при известной дисперсии,
	для $\mu$ при неизвестной дисперсии, для $\sigma^2$;
\end{enumerate}


\newpage
\subsubsection*{\hyperref[sec:sol_minimum_kr_03]{Задачный миннимум}}
\label{sec:minimum_kr_03}

\begin{enumerate}
\item Для взрослого мужчины рост в сантиметрах, величина $X$, и вес в килограммах,
 величина $Y$, являются компонентами нормального случайного вектора $Z = (X, Y)$
 с математическим ожиданием $\E(Z) = (175, 74)$ и ковариационной матрицей

\[
\Var(Z) =
\begin{pmatrix}
 49 & 28 \\
28 & 36
\end{pmatrix}
\]

Рассмотрим разницу роста и веса, $U = X - Y$.
Считается, что человек страдает избыточным весом, если $U < 90$.

\begin{enumerate}
\item Определите вероятность того, что рост мужчины отклоняется от среднего более, чем на $10$ см.
\item Укажите распределение случайной величины $U$. Выпишите её плотность распределения.
\item Найдите вероятность того, что случайно выбранный мужчина страдает избыточным весом.
\end{enumerate}

\item Рост в сантиметрах, случайная величина $X$, и вес в килограммах,
случайная величина $Y$, взрослого мужчины является нормальным случайным вектором
$Z = (X, Y)$ с математическим ожиданием $\E(Z) = (175, 74)$ и ковариационной матрицей

\[
\Var(Z) =
\begin{pmatrix}
 49 & 28 \\
28 & 36
\end{pmatrix}
\]

\begin{enumerate}
\item Найдите средний вес мужчины при условии, что его рост составляет $170$ см.
\item Выпишите условную плотность распределения веса мужчины при условии, что его рост составляет $170$ см.
\item Найдите условную вероятность того, что человек будет иметь вес, больший $90$ кг, при условии, что его рост составляет $170$ см.
\end{enumerate}

\item Для реализации случайной выборки $x=(1,0,-1,1)$ найдите:

\begin{enumerate}
\item выборочное среднее,
\item неисправленную выборочную дисперсию,
\item исправленную выборочную дисперсию,
\item выборочный второй начальный момент,
\item выборочный третий центральный момент.
\end{enumerate}

\item Для реализации случайной выборки $x=(1, 0, -1, 1)$ найдите:

\begin{enumerate}
\item вариационный ряд,
\item первый член вариационного ряда,
\item последний член вариационного ряда,
\item график выборочной функции распределения.
\end{enumerate}

\item Пусть $X_1, \ldots, X_n$ — случайная выборка из дискретного распределения, заданного с помощью таблицы

\begin{center}
\begin{tabular}{cccc}
\toprule
 $x$ & $-3$  & $0$  & $2$  \\
 \midrule
 $\P(X_i = x)$ & $2/3 - \theta$ & $1/3$ & $\theta$ \\
 \bottomrule
\end{tabular}
\end{center}

Рассмотрите оценку $\hat{\theta} = \dfrac{\bar{X}+2}{5}$.

\begin{enumerate}
    \item Найдите $\E\left(\hat{\theta}\right)$.
    \item Является ли оценка $\hat{\theta}$ несмещённой оценкой неизвестного
		параметра $\theta$?
\end{enumerate}

\item Пусть $X_1, \ldots, X_n$ — случайная выборка из распределения
с плотностью распределения

\[
f(x,\theta) = \begin{cases}
\dfrac{6x(\theta - x)}{\theta^3} & \text{при } x \in [0;\theta], \\
0 & \text{при } x \not\in [0;\theta],
\end{cases}
\]

где $\theta > 0$ — неизвестный параметр распределения и $\hat{\theta} = \bar{X}$.

\begin{enumerate}
\item Является ли оценка $\hat{\theta} = \bar{X}$ несмещённой оценкой
неизвестного параметра $\theta$?
\item Подберите константу $c$ так, чтобы оценка $\tilde{\theta} = c\bar{X}$
оказалась несмещенной оценкой неизвестного параметра $\theta$.
\end{enumerate}

\item Пусть $X_1, X_2, X_3$ — случайная выборка из распределения
Бернулли с неизвестным параметром $p \in (0,1)$.
Какие из следующих ниже оценкой являются несмещенными?
Среди перечисленных ниже оценок найдите наиболее эффективную оценку:

\begin{itemize}
  \item $\hat{p}_1 = \dfrac{X_1+X_3}{2}$,
  \item $\hat{p}_2 = \frac{1}{4}X_1+\frac{1}{2}X_2+\frac{1}{4}X_3$,
  \item $\hat{p}_3 = \frac{1}{3}X_1+\frac{1}{3}X_2+\frac{1}{3}X_3$.
\end{itemize}

\item Пусть $X_1, \ldots,X_n$ — случайная выборка из распределения с плотностью

\[
f(x,\theta) =
\begin{cases}
\frac{1}{\theta} \ e^{-\frac{x}{\theta}} & \text{при } x \geq 0, \\
0 & \text{при } x < 0,
\end{cases}
\]

где $\theta > 0$ — неизвестный параметр.
Является ли оценка  $\hat{\theta}_n = \frac{X_1+ \ldots + X_n}{n+1}$ состоятельной?

\item Пусть $X_1, \ldots, X_n$ — случайная выборка из распределения
с плотностью распределения

\[
f(x,\theta) = \begin{cases}
\dfrac{6x(\theta-x)}{\theta^3} & \text{при } x \in [0;\theta], \\
0 & \text{при } x \not\in [0;\theta], \end{cases}
\]

где $\theta > 0$ — неизвестный параметр распределения.
Является ли оценка $\hat{\theta}_n = \frac{2n+1}{n}\bar{X}_n$ состоятельной оценкой
неизвестного параметра $\theta$?

\item Пусть $X_1, \ldots, X_n$ — случайная выборка из распределения
с плотностью распределения

\[
f(x,\theta) =
\begin{cases}
\dfrac{6x(\theta-x)}{\theta^3} & \text{при } x \in [0;\theta], \\
0 & \text{при } x \not\in [0;\theta],
\end{cases}
\]

где $\theta > 0$ — неизвестный параметр распределения.
Используя центральный момент второго порядка, при помощи метода моментов
найдите оценку для неизвестного параметра $\theta$.

\item Пусть $X_1, \ldots, X_n$ — случайная выборка. Случайные величины $X_1, \ldots, X_n$ имеют дискретное распределение, которое задано при помощи таблицы

\begin{center}
\begin{tabular}{cccc}
\toprule
 $x$ & $-3$  &$ 0 $  & $2 $  \\
 \midrule
 $\P(X_i = x)$ & $2/3 - \theta$ & $1/3$ & $\theta$ \\
 \bottomrule
\end{tabular}
\end{center}

Используя второй начальный момент, при помощи метода моментов
найдите оценку неизвестного параметра $\theta$.
Для реализации случайной выборки $x=(0, 0, -3, 0, 2)$
найдите числовое значение найденной оценки параметра $\theta$.

\item Пусть $X_1, \ldots, X_n$ — случайная выборка из распределения
с функцией плотности

\[
f(x,\theta) =
\begin{cases}
\frac{2x}{\theta} \ e^{-\frac{x^2}{\theta}} & \text{при } x>0, \\
0 & \text{при } x \leq 0,
\end{cases}
\]

где $\theta > 0$. При помощи метода максимального правдоподобия найдите оценку неизвестного параметра $\theta$.

\item Пусть $X_1, \ldots, X_n$ – случайная выборка из распределения Бернулли
с параметром $p \in (0;1)$.
При помощи метода максимального правдоподобия найдите оценку
неизвестного параметра $p$.

\item Пусть $X=(X_1, \ldots, X_n)$ — случайная выборка из распределения с плотностью

\[
f(x,\theta) =
\begin{cases}
\frac{1}{\theta} \ e^{-\frac{x}{\theta}} & \text{при } x \geq 0, \\
0 & \text{при } x < 0, \end{cases}
\]

где $\theta > 0$ — неизвестный параметр. Является ли оценка  $\hat{\theta} = \bar{X}$ эффективной?

\item Стоимость выборочного исследования генеральной совокупности,
состоящей из трех страт, определяется по формуле
$TC = c_1n_1 + c_2n_2 + c_3n_3$, где $c_i$ — цена одного наблюдения в $i$-ой страте,
a $n_i$ — число наблюдений, которые приходятся на $i$-ую страту.
Найдите $n_1$, $n_2$ и $n_3$, при которых дисперсия стратифицированного среднего
достигает наименьшего значения,
если бюджет исследования 8000 и имеется следующая информация:

\begin{center}
\begin{tabular}{cccc}
\toprule
 Страта & $1$ & $2$ & $3$  \\
 \midrule
 Среднее значение & $30$ & $40$ & $50$ \\
 Стандартная ошибка  & $5$ & $10$ & $20$ \\
 Вес & $25\%$ & $25\%$ & $50\%$ \\
 Цена наблюдения & $1$ & $5$ & $10$ \\
 \bottomrule
\end{tabular}
\end{center}
\end{enumerate}


\newpage
\subsection{Контрольная работа 4}

\subsubsection*{Теоретический минимум}

\begin{enumerate}
  \item Дайте определение ошибки первого и второго рода, критической области.
  \item Укажите формулу доверительного интервала с уровнем доверия $(1 - \alpha)$ для вероятности успеха,
	построенного по случайной выборке большого размера из распределения Бернулли $Bin(1, p)$.
\end{enumerate}

Для следующего блока вопросов предполагается, что величины $X_1$, $X_2$, \ldots, $X_n$ независимы и нормальны $\cN(\mu;\sigma^2)$.
Укажите формулу для статистики:

\begin{enumerate}[resume]
  \item Статистика, проверяющая гипотезу о математическом ожидании при известной дисперсии $\sigma^2$,
    и её распределение при справедливости основной гипотезы  $H_0$: $\mu = \mu_0$.
  \item Статистика, проверяющая гипотезу о математическом ожидании при неизвестной дисперсии $\sigma^2$,
    и её распределение при справедливости основной гипотезы  $H_0$: $\mu = \mu_0$.
\end{enumerate}


Для следующего блока вопросов предполагается, что есть две независимые случайные выборки:
выборка $X_1$, $X_2$, \ldots{ }размера $n_x$ из нормального распределения $\cN(\mu_x;\sigma^2_x)$
и выборка $Y_1$, $Y_2$, \ldots{ }размера $n_y$ из нормального распределения $\cN(\mu_y;\sigma^2_y)$.

Укажите формулу для статистики или границ доверительного интервала:

\begin{enumerate}[resume]
  \item Доверительный интервал для разницы математических ожиданий, когда дисперсии известны;
  \item Доверительный интервал для разницы математических ожиданий, когда дисперсии не известны, но равны;
  \item Статистика, проверяющая гипотезу о разнице математических ожиданий при известных дисперсиях,
    и её распределение при справедливости основной гипотезы $H_0$: $\mu_x - \mu_y = \Delta_0$;
  \item Статистика, проверяющая гипотезу о разнице математических ожиданий при неизвестных, но равных дисперсиях,
    и её распределение при справедливости основной гипотезы $H_0$: $\mu_x - \mu_y = \Delta_0$;
  \item Статистика, проверяющая гипотезу о равенстве дисперсий,
    и её распределение при справедливости основной гипотезы $H_0$: $\sigma^2_x = \sigma^2_y$.
\end{enumerate}


\newpage
\subsubsection*{\hyperref[sec:sol_minimum_kr_04]{Задачный минимум}}
\label{sec:minimum_kr_04}

\begin{enumerate}

\item Пусть $X_{1}, \ldots, X_{n}$  — случайная выборка из нормального
распределения с параметрами $\mu$ и ${\sigma}^2 = 4$.
Используя реализацию случайной выборки,
\[
x_{1} = -1.11, \quad x_{2} = -6.10, \quad x_{3} =  2.42,
\]
постройте 90\%-ый доверительный интервал для неизвестного параметра $\mu$.

\item Пусть $X_{1}, \ldots, X_{n}$ — случайная выборка
из нормального распределения с неизвестными параметрами $\mu$ и ${\sigma}^2$.
Используя реализацию случайной выборки,
\[
x_{1} = -1.11, \quad x_{2} = -6.10, \quad x_{3} = 2.42,
\]
постройте 90\%-ый доверительный интервал для неизвестного параметра $\mu$.

\item Пусть $X_{1}, \ldots, X_{n}$ — случайная выборка из нормального распределения
с неизвестными параметрами $\mu$ и ${\sigma}^2$.
Используя реализацию случайной выборки,
\[
x_{1} = 1.07, \quad x_{2} = 3.66, \quad x_{3} = -4.51,
\]
постройте 80\%-ый доверительный интервал для неизвестного параметра ${\sigma}^2$.

\item Пусть $X_{1}, \ldots, X_{n}$ и $Y_{1}, \ldots, Y_{m}$ —
независимые случайные выборки из нормального распределения с параметрами
$(\mu_{X},{\sigma^2_{X}})$ и $(\mu_{Y},{\sigma^2_{Y}})$ соответственно,
причем $\sigma^2_{X} = 2$ и $\sigma^2_{Y} = 1$.
Используя реализации случайных выборок
\begin{align*}
x_{1} &= -1.11, \quad x_{2} = -6.10, \quad x_{3} = 2.42, \\
y_{1} &= -2.29, \quad y_{2} = -2.91,
\end{align*}
постройте 95\%-ый доверительный интервал для разности математических ожиданий
$\mu_{X} - \mu_{Y}$.

\item Пусть $X_{1}, \ldots, X_{n}$ и $Y_{1}, \ldots, Y_{m}$ —
независимые случайные выборки из нормального распределения с параметрами
$(\mu_{X},{\sigma^2_{X}})$ и $(\mu_{Y},{\sigma^2_{Y}})$ соответственно.
Известно, что $\sigma^2_{X} = \sigma^2_{Y}$.
Используя реализации случайных выборок
\begin{align*}
x_{1} &= 1.53, \quad x_{2} = 2.83, \quad x_{3} = -1.25 \\
y_{1} &= -0.8, \quad y_{2} = 0.06
\end{align*}
постройте 95\%-ый доверительный интервал для разности математических ожиданий
$\mu_{X} - \mu_{Y}$.

\item Пусть $X_{1}, \ldots, X_{n}$ — случайная выборка из распределения Бернулли
с параметром $p$.
Используя реализацию случайной выборки $X_{1}, \ldots, X_{n}$,
в которой 55 нулей и 45 единиц,
постройте приближенный 95\%-ый доверительный интервал для неизвестного параметра $p$.

\item Пусть $X_{1}, \ldots, X_{n}$ и $Y_{1}, \ldots, Y_{m}$ — независимые случайные
выборки из распределения Бернулли с параметрами $p_{X} \in (0;1)$ и $p_{Y} \in (0;1)$ соответственно.
Известно, что $n = 100$, $\bar{x}_{n} = 0.6$, $m = 200$, $\bar{y}_{m} = 0.4$.
Постройте приближенный 95\%-ый доверительный интервал для отношения разности
вероятностей успеха $p_{X} - p_{Y}$.

\item Дядя Вова (Владимир Николаевич) и Скрипач (Гедеван) зарабатывают на Плюке чатлы,
чтобы купить гравицапу.
Число заработанных за $i$-ый день чатлов имеет распределение Пуассона с неизвестным параметром $\lambda$.
Заработки в различные дни независимы. За прошедшие 100 дней они заработали 250 чатлов.

С помощью метода максимального правдоподобия постройте приближенный
95\%-ый доверительный интервал для неизвестного параметра $\lambda$.

\item Пусть $X_{1}, \ldots, X_{n}$ — случайная выборка из показательного
(экспоненциального) распределения с плотностью распределения
\[
f(x,\lambda)=
\begin{cases}
\lambda e^{-\lambda x}\text{ при } x\geq 0 \\
0 \text{ при } x < 0 \\
\end{cases}
\]
где $\lambda > 0$ — неизвестный параметр распределения.
Известно, что $n = 100$ и $\bar{x}_n = 0.52$.

С помощью метода максимального правдоподобия постройте приближенный
95\%-ый доверительный интервал для параметра $\lambda$.

% \item Пусть $X_{1}, \ldots, X_{n}$  — случайная выборка из равномерного распределения на отрезке $[0; \theta]$, где  $\theta > 0$ — неизвестный параметр распределения. Известно, что $n = 100$ и $\bar{x}_n = 0.57$.

% С помощью метода максимального правдоподобия постройте приближенный 95\%-ый доверительный интервал для параметра $\theta$.
% тут брутальный дельта-метод :) это не минимум :)


\item Пусть $X_{1}, \ldots, X_{n}$ — случайная выборка из нормального распределения
с неизвестным математическим ожиданием $\mu$ и известной дисперсией $\sigma^2 = 4$.
Объем выборки $n = 16$. Для тестирования основной гипотезы $H_{0}:\mu = 0$ против
альтернативной гипотезы $H_{1}:\mu = 2$ вы используете критерий: если $\bar{X} \leq 1$,
то вы не отвергаете гипотезу $H_{0}$,
в противном случае вы отвергаете гипотезу $H_{0}$ в пользу гипотезы $H_{1}$. Найдите

\begin{enumerate}
   \item  вероятность ошибки 1-го рода;
   \item вероятность ошибки 2-го рода;
   \item мощность критерия.
\end{enumerate}


\item На основе случайной выборки, содержащей одно наблюдение $X_{1}$,
тестируется гипотеза $H_{0} : X_{1} \sim U[-0.7;0.3]$ против альтернативной гипотезы
$H_{1}: X_{1} \sim U[-0.3;0.7]$.
Рассматривается критерий вида: если $X_{1} > c$,
то гипотеза $H_{0}$ отвергается в пользу гипотезы $H_{1}$.
Выберите константу $c$ так, чтобы уровень значимости этого критерия составлял $0.1$.

\item Пусть $X_{1}, \ldots, X_{n}$ — случайная выборка из нормального распределения
с параметрами $\mu$ и $\sigma^2 = 4$.
Уровень значимости  $\alpha = 0.1$.
Используя реализацию случайной выборки $x_{1} = -1.11, x_{2} = -6.10, x_{3} = 2.42$,
проверьте следующую гипотезу:
\[
\begin{cases}
H_{0}:\mu = 0, \\
H_{1}:\mu > 0 \\
\end{cases}
\]

\item Пусть $X_{1}, \ldots, X_{n}$ — случайная выборка из нормального
распределения с параметрами $\mu$ и $\sigma^2$.
Уровень значимости  $\alpha = 0.1$.
Используя реализацию случайной выборки $x_{1} = -1.11, x_{2} = -6.10, x_{3} = 2.42$,
проверьте следующую гипотезу:
\[
\begin{cases}
H_{0}:\mu = 0, \\
H_{1}:\mu > 0 \\
\end{cases}
\]

\item Пусть $X_{1}, \ldots, X_{n}$ и $Y_{1}, \ldots, Y_{m}$ —
независимые случайные выборки из нормального распределения
с параметрами $(\mu_{X},\sigma^2_{X})$ и $(\mu_{Y},\sigma^2_{Y})$ соответственно,
причем  $\sigma^2_{X} = 2$ и $\sigma^2_{Y} = 1$. Уровень значимости $\alpha = 0.05$.
Используя реализации случайных выборок

\begin{align*}
x_{1} &= -1.11, \quad x_{2} = -6.10, \quad x_{3} = 2.42, \\
y_{1} &= -2.29, \quad y_{2} = -2.91,
\end{align*}

проверьте следующую гипотезу:
\[
\begin{cases}
H_{0}:\mu_{X} = \mu_{Y}, \\
H_{1}:\mu_{X} < \mu_{Y} \\
\end{cases}
\]

\item Пусть $X_{1}, \ldots, X_{n}$ и $Y_{1}, \ldots, Y_{m}$ — независимые случайные
выборки из нормального распределения с параметрами $(\mu_{X},\sigma^2_{X})$ и
$(\mu_{Y},\sigma^2_{Y})$ соответственно.
Известно, что $\sigma^2_{X} = \sigma^2_{Y}$.
Уровень значимости $\alpha = 0.05$. Используя реализации случайных выборок

\begin{align*}
x_{1} &= 1.53, \quad x_{2} = 2.83, \quad x_{3} = -1.25 \\
y_{1} &= -0.8, \quad y_{2} = 0.06
\end{align*}

проверьте следующую гипотезу:
\[
\begin{cases}
H_{0}:\mu_{X} = \mu_{Y}, \\
H_{1}:\mu_{X} < \mu_{Y} \\
\end{cases}
\]

\item Пусть $X_{1}, \ldots, X_{n}$ и $Y_{1}, \ldots, Y_{m}$ — независимые случайные
выборки из нормального распределения с параметрами $(\mu_{X},\sigma^2_{X})$ и
$(\mu_{Y},\sigma^2_{Y})$ соответственно.
Уровень значимости $\alpha = 0.05$.
Используя реализации случайных выборок\newline
\begin{align*}
x_{1} &= -1.11, \quad x_{2} = -6.10, \quad x_{3} = 2.42, \\
y_{1} &= -2.29, \quad y_{2} = -2.91,
\end{align*}
проверьте следующую гипотезу:
\[
\begin{cases}
H_{0}:\sigma^2_{X} = \sigma^2_{Y}, \\
H_{1}:\sigma^2_{X} > \sigma^2_{Y} \\
\end{cases}
\]

\item Пусть  $X_{1}, \ldots, X_{n}$ — случайная выборка из распределения Бернулли с
неизвестным параметром $p \in (0;1)$.Имеется следующая информация о реализации
случайной выборки, содержащей $n = 100$ наблюдений: $\sum_{i=0}^{n} x_{i} = 60$.
На уровне значимости $\alpha = 0.05$ требуется протестировать следующую гипотезу:
\[
\begin{cases}
H_{0}:p = 0.5, \\
H_{1}:p > 0.5 \\
\end{cases}
\]

\item Пусть $X_{1}, \ldots, X_{n}$ и $Y_{1}, \ldots, Y_{m}$ —
две независимые случайные выборки из распределения Бернулли с неизвестными параметрами
$p_{X} \in (0; 1)$ и $p_{Y} \in (0; 1)$.
Имеется следующая информация о реализациях этих случайных выборок: $n = 100$,
$\sum_{i=1}^{n} x_{i} = 60$, $m = 150$,$\sum_{j=1}^{m} y_{j} = 50$.
На уровне значимости $\alpha = 0.05$ требуется протестировать следующую гипотезу:
\[\begin{cases}
H_{0}:p_{X} = p_{Y}, \\
H_{1}:p_{X} \neq p_{Y} \\
\end{cases}\]

\item Вася Сидоров утверждает, что ходит в кино в два раза чаще, чем в спортзал,
а в спортзал в два раза чаще, чем в театр.
За последние полгода он 10 раз был в театре, 17 раз – в спортзале и 39 раз в кино.
На уровне значимости 5\% проверьте утверждение Васи.

\item Вася очень любит тестировать статистические гипотезы.
В этот раз Вася собирается проверить утверждение о том,
что его друг Пётр звонит Васе исключительно в то время, когда Вася ест.
Для этого Вася трудился целый год и провел серию из 365 испытаний.
Результаты приведены в таблице ниже.

\begin{center}\begin{tabular}{r|rr}
\toprule
   & Пётр звонит   & Пётр не звонит  \\ \midrule
Вася ест           & $200$ & $40$ \\
 Вася не ест       & $25$ & $100$  \\ \bottomrule
\end{tabular}\end{center}

На уровне значимости 5\% протестируйте гипотезу о том, что Пётр звонит Васе
независимо от момента приема пищи Васей.

\item Пусть $X_{1}, \ldots, X_{n}$ — случайная выборка из нормального распределения
с математическим ожиданием $\mu \in \mathbb{R}$ и дисперсией $v > 0$,
где $\mu$ и $v$ — неизвестные параметры.
Известно, что выборка состоит из $n = 100$ наблюдений,
$\sum_{i=1}^{n} x_{i} = 30$, $\sum_{i=1}^{n} x^2_{i} = 146$.
При помощи теста отношения правдоподобия протестируйте гипотезу $H_{0}:v = 1$
на уровне значимости 5\%.

\end{enumerate}
