% !TEX root = ../probability_hse_exams.tex
% \newpage
\thispagestyle{empty}
\section{Решения контрольной работы 2}

\subsection[2018-2019]{\hyperref[sec:kr_02_2018_2019]{2018-2019}}
\label{sec:sol_kr_02_2018_2019}

\begin{enumerate}
\item

\item
\begin{enumerate}
    \item Пусть $A$ — количество очков 1-го стрелка, $B$ — количество очков 2-го, тогда $\E(A) = 50 \cdot 0.5 = 25$, $\Var(A) = 50 \cdot 0.5 \cdot (1 - 0.5) = 12.5$.
        
    \[\P(A \ge 15) = \P\biggl(\frac{A - \E(A)}{\sqrt{\Var(A)}} \ge \frac{15 - \E(A)}{\sqrt{\Var(A)}}\biggl) 
    = \P\biggl(\frac{A - 25}{\sqrt{12.5}} \ge -2.83\biggl)\]
    
    Согласно ЦПТ, $\frac{A - 25}{\sqrt{12.5}} \sim \cN(0, 1)$, значит $\P\biggl(\frac{A - 25}{\sqrt{12.5}} \ge -2.83\biggl) = 0.9977$
    
    \item $\P(A > B) = \P(A - B > 0)$.~~Пусть $Z = A - B$.
    
    \[\E(Z) = \E(A) - \E(B) = 25 - 30 = -5\] 
    
    \[\Var(Z) = \Var(A) + \Var(B) - 2\cov(A, B) = 12.5 + 12 = 24.5\] 
    
    \[\P(A - B > 0) = \P(Z > 0) = \P\biggl(\frac{Z + 5}{\sqrt{24.5}} > \frac{5}{\sqrt{24.5}}\biggl) = \P\biggl(\frac{Z + 5}{\sqrt{24.5}} > 1.01\biggl)\]
    
    Согласно ЦПТ, $\frac{Z + 5}{\sqrt{24.5}} \sim \cN(0, 1)$, значит $\P\Bigl(\frac{Z + 5}{\sqrt{24.5}} > 1.01\Bigl) = 0.1562$
\end{enumerate}

\end{enumerate}

\subsection[2017-2018]{\hyperref[sec:kr_02_2017_2018]{2017-2018}}
\label{sec:sol_kr_02_2017_2018}

\begin{enumerate}
\item[7.]
\begin{enumerate}
\item Всем хватит места, если число явившихся на рейс пассажиров ($X$) не превысит $300$,
то есть нужно найти $\P(X \leq 300)$. Найдём матожидание и дисперсию
случайной величины $X$:
\begin{align*}
\E(X) &= np = 330 \cdot 0.9 = 297 \\
\Var(X) &= np(1-p) = 330 \cdot 0.9 \cdot 0.1 = 29.7
\end{align*}
Теперь посчитаем нужную вероятность:
\[
\P(X \leq 300) = \P \left(\frac{X - 297}{\sqrt{29.7}} \leq \frac{300 - 297}{\sqrt{29.7}} \right) = \P(\cN(0,1) \leq 0.55) \approx 0.709
\]
\item Вероятность переполнения не должна превышать $0.1$:
\begin{align*}
&\P(X > 300) < 0.1 \\
&\P\left(\frac{X - 0.9 \cdot n}{\sqrt{0.9 \cdot 0.1 \cdot n}} > \frac{300 - 0.9 \cdot n}{\sqrt{0.9 \cdot 0.1 \cdot n}} \right) < 0.1 \\
&\frac{300 - 0.9 \cdot n}{\sqrt{0.9 \cdot 0.1 \cdot n}}  > 1.28 \\
&300 - 0.9n > 1.28 \cdot 0.3 \sqrt{n} \\
&n < 325.6
\end{align*}
\end{enumerate}
\item[8.]
\begin{enumerate}
\item Выпишем случайную величину $X_i$ — цену акции после $i$-ого дня:
\[
X_i =
\begin{cases}
1.01, & p = 0.7 \\
0.99, & p = 0.2999 \\
0, & p = 0.0001
\end{cases}
\]
Нужно посчитать ожидание цены акциии после 20 дней:
\[
\E(X_1 \cdot \ldots \cdot X_{20}) \stackrel{\text{незав-ть}}{=} \E(X_1) \cdot \ldots \cdot \E(X_{20}) = 1.004^{20} \approx 1.083
\]
\item По ЗБЧ:
\[
\plim_{n\to\infty} \frac{1}{n} \sum_{i=1}^n X_i = \E(X_i) = 1.004
\]
\item Аналогично пункту (а):
\[
\E(X_1 \cdot \ldots \cdot X_{n}) = (\E(X_1))^n = 1.004^n
\]
И понятно, что $1.004^n \to_{n\to\infty} +\infty$.
\item
\begin{align*}
\P(\text{разорения}) &= 1 - \P(X_1 > 0, \ldots, X_n >0) = 1 - \prod_{i=1}^n \P(X_i > 0) \\
&= 1 - (1 - 0.0001)^n \to_{n\to\infty} 1
\end{align*}
\end{enumerate}
\end{enumerate}




\subsection[2016-2017]{\hyperref[sec:kr_02_2016_2017]{2016-2017}}
\label{sec:sol_kr_02_2016_2017}

\begin{enumerate}
\item \begin{enumerate}
\item $\E (2\xi - \eta +1) = 2 \E (\xi) - \E (\eta) + 1 = 2\cdot 1 - (-2) + 1 = 5 $

$\Cov (\xi, \eta) = \E (\xi \eta) - \E(\xi) \E(\eta) = -1 - \cdot 1 \cdot (-2) = 1$

$\Corr (\xi, \eta) = \frac{\Cov(\xi, \eta)}{\sqrt[]{\Var(X) \cdot \Var(Y)}} = \frac{1}{\sqrt{1 \cdot (8-(-2)^2)}} = \frac{1}{2}$

$\Var (2\xi - \eta + 1) = 4\Var(\xi) + \Var(\eta) - 2 \Cov (2\xi, \eta) = 4 \cdot 1 + 4 - 4 \cdot 1 = 4$

\item $\Cov(\xi + \eta, \xi + 1) = \Cov(\xi) + \Cov(\xi, 1) + \Cov(\eta, \xi) + \Cov(\eta, 1) = 1 +1 = 2$

$\Corr(\xi + \eta , \xi + 1) = \frac{\Cov(\xi + \eta , \xi + 1)}{\sqrt{\Var(\xi + \eta)\cdot \Var (\xi + 1)}} = \frac{2}{\sqrt{(1+4+2\cdot 1) \cdot 1}} = \frac{2}{\sqrt{7}}$

$\Corr(\xi + \eta - 24, 365 - \xi - \eta) = -1$

$\Cov(2016\cdot \xi, 2017) = 0$

\end{enumerate}
\item
\begin{enumerate}
\item Частные распределения:

\begin{center}
\begin{tabular}{cccc}
\toprule
$x$ & $-1$ & $0$ & $2$ \\
$\P(\xi = x)$ & $0.3$ & $0.4$ & $0.3$ \\ \bottomrule
\end{tabular}
\hspace{1cm}
\begin{tabular}{ccc}
\toprule
$y$ & $-1$ & $1$ \\
$\P(\eta = y)$ & $0.5$ & $0.5$ \\ \bottomrule
\end{tabular}
\end{center}

$\E(\xi) = -1 \cdot 0.3 + 0 \cdot 0.4 + 2 \cdot 0.3 = 0.3$

$\E (\xi^2) = (-1)^2 \cdot 0.3 + 0^2 \cdot 0.4 + 2^2 \cdot 0.3 = 1.5$

$\Var(\xi) = \E(\xi^2) - (\E(\xi))^2 = 1.5 - 0.3^2 = 1.41$

$\E(\eta) = -1 \cdot 0.5 + 1 \cdot 0.5 = 0$

$\E(\eta^2) = (-1)^2 \cdot 0.5 + 1^2 \cdot 0.5 = 1$

$\Var(\eta) = \E(\eta^2)-(\E(\eta))^2 = 1 - 0^2 = 1$

\item
\begin{center}
\begin{tabular}{cccccc}
\toprule
$xy$ & $-2$ & $-1$ & $0$ & $1$ & $2$ \\
$\P(\xi \cdot \eta = xy)$ & $0.2$ & $0.2$ & $0.4$ & $0.1$ & $0.1$ \\ \bottomrule
\end{tabular}
\end{center}

$\E(\xi\cdot\eta) = (-2) \cdot 0.2 + (-1) \cdot 0.2 + 0 \cdot 0.4 + 1\cdot 0.1 + 2 \cdot 0.1 = -0.3$

$\Cov(\xi, \eta) = \E(\xi\cdot\eta) - \E(\xi)\cdot\E(\eta) = -0.3 - 0.3 \cdot 0 = -0.3$
\item Пусть случайная величина $X$ принимает значения $a_1, \ldots, a_m$, случайная веилчина $Y$ принимает значения $b_1, \ldots, b_n$. Тогда случйаня величина $X$ и $Y$ называются независимыми, если $\forall i=1, \ldots, m \quad \forall j=1, \ldots, n: \P(X = a_i \cap Y = b_j) = P(X = a_i) \cdot P(Y = b_j)$
\item Заметим, что $\P (\xi = -1 \cap \eta=-1)=0.1$, $\P(\xi=-1)=0.3$ и $\P(\eta=-1)=0.5$.

Тогда поскольку $\P (\xi = -1 \cap \eta=-1) \neq \P(\xi=-1) \cdot \P(\eta=-1)$, случайные величины $\xi$ и $\eta$ не являются независимыми.
\item $\P (\xi = -1 \cap \eta=1) = \frac{\P (\xi = -1 \cap \eta=1)}{\P(\eta=1)} = \frac{0.2}{0.5} = \frac{2}{5}$

$\P (\xi = 0 \cap \eta=1) = \frac{\P (\xi = 0 \cap \eta=1)}{\P(\eta=1)} = \frac{0.2}{0.5} = \frac{2}{5}$

$\P (\xi = 2 \cap \eta=1) = \frac{\P (\xi = 2 \cap \eta=1)}{\P(\eta=1)} = \frac{0.1}{0.5} = \frac{1}{5}$

Следовательно, условное распределение случайной величины $\xi$ при условии $\{\eta=1\}$ может быть описано следующей таблицей:

\begin{center}
\begin{tabular}{cccc}
\toprule
$x$ & $-1$ & $0$ & $2$ \\
$\P(\xi = x)$ & $2/5$ & $2/5$ & $1/5$ \\ \bottomrule
\end{tabular}
\end{center}

\item $\E(\xi \mid \eta = 1) = -1 \cdot \frac{2}{5} + 0 \cdot \frac{2}{5} + 2 \cdot \frac{1}{5} = 0$
\item $\E(\pi) = \E(0.5 \xi + 0.5 \eta) = 0.5 \E(\xi) + 0.5 \E(\eta) = 0.15$

\begin{align*}
\Var(\pi) &= \Var(0.5 \xi + 0.5 \eta) = \Var(0.5 \xi) + \Var(0.5\eta) + 2 \Cov (0.5\xi, 0.5\eta) \\
&= 0.25\Var(\xi) + 0.25\Var(\eta) + 2 \cdot 0.5 \cdot 0.5 \Cov(\xi, \eta) \\
&= 0.25 \cdot 1.41 + 0.25 \cdot 1 + 2 \cdot 0.5 \cdot 0.5 \cdot (-0.3) = 0.4525
\end{align*}
\item
\begin{align*}
\Var(\pi(\alpha)) &= \Var(\alpha \xi + (1-\alpha)\eta) = \alpha^2\Var(\xi) + (1-\alpha)^2 \Var(\eta) \\
&+ 2\alpha(1-\alpha) \Cov(\xi, \eta) = 1.41 \cdot \alpha^2 + 1\cdot (1-\alpha)^2 + 2\alpha(1-\alpha) \cdot (-0.3) \\
&= 1.41 \cdot \alpha^2 + (1-\alpha)^2 - 0.6 \cdot (\alpha - \alpha^2) \to \min_\alpha \\
\frac{\partial}{\partial \alpha} \Var(\pi(\alpha)) &= 2 \cdot 1.41 \cdot \alpha -2(1-\alpha) -0.6\cdot(1-2\alpha) \\
&= 2.82 \cdot \alpha - 2 + 2\alpha - 0.6 + 1.2 \cdot \alpha = 6.02 \cdot \alpha - 2.6 = 0 \\
\alpha &= \frac{2.6}{6.02} = 0.4319
\end{align*}
\end{enumerate}
\item \begin{enumerate}
\item Для любой неотрицательной случайной величины $X$ и любого числа $\lambda > 0$ справедлива оценка: $\P(X>\lambda) \leq \frac{\E(X)}{\lambda}$

Пусть случайная величина $\xi_i$ означает число посетителей сайта за $i$-ый день. По условию, $\xi_i \sim Pois(\lambda=250)$. Известно, что если $\xi \sim Pois(\lambda)$, то $\E(\xi) = \Var(\xi) = \lambda$.

Имеем:
\[
\P(\xi_i >500) \leq \P(\xi_i \geq 500) \leq \frac{\E(\xi_i)}{500} = \frac{250}{500} = \frac{1}{2}
\]
\item Для любой случайной величины $X$ с конечным $\E(X)$ и любого положительного числа $\epsilon > 0$ имеет место неравенство: $\P(X-\E(X)\geq\epsilon)\leq\frac{\Var(X)}{\epsilon^2}$

Обозначим $\bar{\xi}_n := \frac{1}{n} \left(\xi_1 + \ldots + \xi_n\right)$ – среднее число посетителей сайта за $n$ дней. Тогда
\[
\E\left(\bar{\xi}_n\right) = \E\left(\frac{1}{n} \sum_{i=1}^{n} \xi_i\right) = \frac{1}{n} \sum_{i=1}^{n} \E(\xi_i) = \frac{1}{n} \cdot n \cdot \lambda = \lambda = 250
\]
\[
\Var\left(\bar{\xi}_n\right) = \Var\left(\frac{1}{n} \sum_{i=1}^{n} \xi_i\right) = \frac{1}{n^2} \sum_{i=1}^{n} \Var (\xi_i) = \frac{n \cdot \lambda}{n^2} = \frac{\lambda}{n} = \frac{250}{n}
\]
Оценим вероятность
\[
\P\left(\left\vert\bar{\xi}_n-250\right\vert > 10\right) \leq \frac{\Var\left(\bar{\xi}_n\right)}{100} = \frac{250}{100\cdot n}
\]
Следовательно, $1 - \frac{250}{100\cdot n} \leq \P\left(\left\vert\bar{\xi}_n-250\right\vert \leq 10\right)$.

Найдём наименьшее целое $n$, при котором левая часть неравенства ограничена снизу $0.99 \leq 1 - \frac{250}{100\cdot n}$.

Имеем:
\[
0.99 \leq 1 - \frac{250}{100\cdot n} \Leftrightarrow \frac{250}{100\cdot n} \leq 0.01 \Leftrightarrow n \geq \frac{250}{100 \cdot 0.01} \Leftrightarrow n  \geq 250
\]
Значит, $n=250$ – наименьшее число дней, при котором с вероятностью не менее $99\%$ среднее число посетителей будет отличаться от $250$ не более чем на $10$.
\item  Требуется найти наименьшее целое $n$, при котором $\P\left(\left\vert\bar{\xi}_n-250\right\vert \leq 10\right) = 0.99$

Имеем:
\begin{align*}
&\P(\vert\bar{\xi}_n - 250\vert \leq 10) = 0.99 \Leftrightarrow \P(-10\leq \bar{\xi}_n - 250 \leq 10) = 0.99 \\
&\P(-10n \leq S_n - 250n \leq 10n) = 0.99 \\
&\E(S_n) = \E(\xi_1 + \ldots + \xi_n) = \E(\xi_1) + \ldots + \E(\xi_n) = 250 \cdot n\\
&\Var(S_n) = \Var(\xi_1 + \ldots + \xi_n) = \Var(\xi_1) + \ldots + \Var(\xi_n) = 250 \cdot n \\
&\P\left(\frac{-10n}{\sqrt{250n}} \leq \frac{S_n - \E (S_n)}{\sqrt{\Var(S_n)}} \leq \frac{10n}{\sqrt{250n}}\right) = 0.99  \\
&2 \Phi \left(\frac{10n}{\sqrt{250n}}\right) -1 = 0.99 \\
&\Phi \left(\frac{10n}{\sqrt{250n}}\right) = \frac{1 + 0.99}{2}  \\
&\frac{10n}{\sqrt{250n}} = 2.58 \\
&\sqrt{n} = 2.58 \cdot \frac{\sqrt{250}}{10}  \\
&n = 16.641
\end{align*}
Следовательно, наименьшее целое $n$, есть $n=17$.
\item Пусть $X_1, X_2, \ldots, X_n, \ldots$ – последовательность независимых случайных величин с одинаковыми конечными математическими ожиданимяи и фиксированными конечными дисперсиями. Тогда $\frac{X_1 + \ldots + X_n}{n} \stackrel{\P}{\to} \E(X_i)$ при $n \to \infty$.

В нашем случае случаные величины $\xi_1^2, \xi_2^2, \ldots, \xi_n^2, \ldots$ – независимы,

$\E(\xi_1^2) = \ldots = \E(\xi_n^2) = \ldots < + \infty$ и $\Var(\xi_1^2) = \ldots = \Var(\xi_n^2) = \ldots < + \infty$ . Поэтому в соответствии с ЗБЧ имеем:
\[
\frac{\xi_1^2 +\ldots+ \xi_n^2}{n} \stackrel{\P}{\to} \E(\xi_i^2) = \Var(\xi_i) +\E(\xi_i)^2 = \lambda + \lambda^2 = \lambda(\lambda+1) = 250\cdot251 = 62750
\]
\end{enumerate}

\item \begin{enumerate}
\item Пусть $X_1, X_2, \ldots, X_n, \ldots$ – последовательность независимых,
одинаково распределённых случайных величин с $0<\Var(X_i)<\infty$, $i \in \mathbb{N}$.
Тогда для любого (борелевского) множества $B \subseteq R$ имеет место
\[
\lim_{n \to \infty} \P\left(\frac{S_n - \E(S_n)}{\sqrt{\Var(S_n)}} \in B\right) = \int_B \frac{1}{\sqrt{2\pi}}e^{-t^2/2} dt,
\]
где $S_n := X_1, \ldots, X_n$, $n \in \mathbb{N}$

\item Введём случайную величину
\[
X_i = \begin{cases}
1, & \text{если на i-ом шаге Винни-Пух пошёл направо} \\
-1, & \text{если пошёл налево}
\end{cases}
\quad i=1,\ldots, n;
\]
Тогда $S_n := X_1 +\ldots+X_n$ означает местоположение Винни-Пуха в $n$-ую минуту его блужданий по прямой.

$\E(X_i) = -1 \cdot 1/2 + 1 \cdot 1/2 = 0$,

$\E(X_i^2) = (-1)^2 \cdot 1/2 + (1)^2 \cdot 1/2 = 1$,

$\Var(X_i) = \E(X_i^2) - \E(X_i)^2 = 1$,

$\E(S_n) = \E(X_1 + \ldots + X_n ) = \E(X_1) + \ldots + \E(X_n) = 0$,

$\Var(S_n) = \Var(X_1 + \ldots + X_n ) = \Var(X_1) + \ldots + \Var(X_n) = n$

\begin{align*}
\P(S_n \in (-\infty, -5]) &= \P(S_n \leq -5) = \P\left( \frac{S_n-\E(S_n)}{\sqrt{\Var(S_n)}} \leq \frac{-5-0}{\sqrt{n}} \right)  \\
&\stackrel{n=60}{=}\P\left( \frac{S_n-\E(S_n)}{\sqrt{\Var(S_n)}} \leq -0.6454\right) \approx \int_{-\infty}^{-0.6454} \frac{1}{\sqrt{2\pi}} e^{-t^2/2} dt \\
&= \Phi(-0.6454) = 1-\Phi(0.6454) \approx0.2593
\end{align*}
\item Для любых $n \in \mathbb{N}$ и всех $x \in \mathbb{R}$ имеет место оценка:
\[
\bigl|F_{S_n^{*}}(x) - \Phi(x)\bigr| \leq 0.48 \cdot \frac{\E(|\xi_i - \E\xi_i|^3)}{\Var^{3/2}(\xi_i)\cdot\sqrt{n}} \text{,}
\]
где $\Phi(x) = \int_{-\infty}^{x}\frac{1}{\sqrt{2\pi}}e^{-\frac{t^2}{2}}\,dt$, \; $S_n^* = \frac{S_n - \E(S_n)}{\sqrt{\Var(S_n)}}$, \; $S_n = \xi_1 + \ldots + \xi_n$

В нашем случае:
\[
\P\left( \frac{S_{60} - \E(S_{60})}{\sqrt{\Var(S_{60})}} \leq -0.6454 \right) = \P(S^*_{60} \leq -0.6454) =
F_{S^*_{60}} (-0.6454)
\]
Согласно неравенству Берри-Эссеена, погрешность $\vert F_{S^*_{60}} (-0.6454) - \Phi(-0.6454) \vert$ оценивается сверху величиной
\[
0.48 \cdot \frac{\E(\vert X_i - \E(X_i) \vert^3 )}{\Var(X_i)^{3/2} \cdot \sqrt{n}} = 0.48 \cdot \frac{\E(\vert X_i \vert^3)}{1\cdot\sqrt{60}} = \frac{0.48}{\sqrt{60}} \approx0.062
\]
\end{enumerate}
\item \begin{enumerate}
\item Сначала найдём плотность распределения случайной величины $X$.
Пусть $x \leq 0 $, тогда $f_X (X) = \int_{-\infty}^{+\infty} f_{X, Y} (x, y) dy  = 0$.

Пусть $x >0 $, тогда
\begin{align*}
f_X (X) &= \int_{-\infty}^{+\infty} f_{X, Y} (x, y) dy = \int_{0}^{+\infty} 0.005 e^{-0.05x-0.1y} dy \\
&= 0.005e^{-0.05x} \int_{0}^{+\infty} e^{-0.1y} dy = 0.005e^{-0.05x} \cdot \left(-10e^{-0.1y} \right) \bigg\vert_{y=0}^{y=+\infty} = 0.05 e^{-0.05x}
\end{align*}
Таким образом, имеем:
\[
f_X (x) = \begin{cases}
0.05 e^{-0.05x} & \text{при } x>0 \\
0 & \text{при } x \leq 0
\end{cases}
\]
То есть $X \sim Exp(\lambda=0.05)$ – случайная величина $X$ имеет показательное
распределение с параметром $\lambda = 0.05$.

Теперь найдём плотность распределения случайной величины $Y$.

Пусть $y \leq 0 $, тогда $f_Y (y) = \int_{-\infty}^{+\infty} f_{X, Y} (x, y) dx  = 0$.

Пусть $y > 0 $, тогда
\begin{align*}
f_Y (y) &= \int_{-\infty}^{+\infty} f_{X, Y} (x, y) dx  = \int_{0}^{+\infty} 0.005 e^{-0.05x-0.1y} dx \\
&= 0.005e^{-0.1y} \int_{0}^{+\infty} e^{-0.05x} dx = 0.005e^{-0.1y} \cdot \left(-20e^{-0.05x} \right) \bigg\vert_{x=0}^{x=+\infty} = 0.1 e^{-0.1y}
\end{align*}
Таким образом, имеем:
\[
f_Y (y) = \begin{cases}
0.1 e^{-0.1y} & \text{при } y>0 \\
0 & \text{при } y \leq 0
\end{cases}
\]
То есть $Y \sim Exp(\lambda=0.1)$ – случайная величина $Y$ имеет показательное распределение с параметром $\lambda = 0.1$.
\item Поскольку для любых точек $x, y \in \mathbb{R}$ справедливо равенство $f_{X, Y} (x, y) = f_X (x) \cdot f_Y (y)$, случайные величины $X$ и $Y$ являются независимыми.
\item Найдём вероятность $\P(Y>5)$:
\[
\P(Y>5) = \int_{5}^{+\infty} f_Y (y) dy = \int_{5}^{+\infty}  0.1 e^{-0.1y} dy = 0.1 \cdot (-10 e^{-0.1x}) \bigg\vert_{y=5}^{y=+\infty} = e^{-0.5} \approx0.6065
\]
\item Требуется найти условную вероятность $\P(Y>8 \mid Y \geq 3)$. Для этого предварительно найдём вероятности $\P(Y>8)$ и $\P(y \geq 3)$:
\[
\P(Y>8) = \int_{8}^{+\infty} f_Y (y) dy  = \int_{8}^{+\infty}  0.1 e^{-0.1y} dy = 0.1 \cdot (-10 e^{-0.1x}) \bigg\vert_{y=8}^{y=+\infty} = e^{-0.8}
\]
\[
\P(Y \geq 3) =  \int_{3}^{+\infty} f_Y (y) dy   =  \int_{3}^{+\infty}  0.1 e^{-0.1y} dy = 0.1 \cdot (-10 e^{-0.1x}) \bigg\vert_{y=3}^{y=+\infty} = e^{-0.3}
\]
Теперь находим требуемую условную веростность:
\[
\P(Y>8 \mid Y \geq 3) = \frac{\P(Y > 8 \cap
Y \geq 3)}{\P(Y \geq 3)} = \frac{\P(Y>8)}{\P(Y \geq 3) } = \frac{e^{-0.8}}{e^{-0.3}} = e^{-0.5} \approx0.6065
\]
\item Сначала найдём условную плотность распределения случайной величины $X$ при условии $Y=y$:
\begin{align*}
f_{X \mid Y} (x \mid y) &=
\begin{cases}
\frac{f_{X \mid Y} (x, y)}{f_Y (y)} & \text{при } f_Y (y) \\
0 & \text{иначе}
\end{cases} \\
&=
\begin{cases}
\frac{0.005e^{-0.05x-0.1y}}{0.1e^{-0.1y}} & \text{при } x>0, \quad y>0 \\
0 & \text{иначе}
\end{cases} \\
&= \begin{cases}
0.05 e^{-0.05x} & \text{при } x>0, \quad y>0 \\
0 & \text{иначе}
\end{cases}\\
&=
\begin{cases}
f_X (x) & \text{при } y > 0 \\
0 & \text{при } y \leq 0
\end{cases}
\end{align*}
Теперь находим условное математическое ожидание
\[
\E(X \mid Y=5) = \int_{-\infty}^{+\infty} xf_{X\mid Y} (x \mid 5) dx =  \int_{-\infty}^{+\infty} xf_{X} (x) dx = \E(X) = \frac{1}{0.05} =20
\]
Здесь мы воспользовались известным фактом, что если $X\sim Exp(\lambda)$, то $\E(X) = \frac{1}{\lambda}$
\item Требуется найти вероятность $\P(X-Y > 2)$. Для этого введём множества

$B:=\{(x, y) \in \mathbb{R} : y < x-2 \}$ и $C := \{ (x,y) \in \mathbb{R} : y < x-2, x>0, y> 0  \}$.

Заметим, что искомая вероятность  $\P(X-Y > 2)$ может быть записана в виде
\[
\P(X-Y > 2) = \P((X, Y) \in B ) = \int \int_B f_{X, Y} (x, y) dx dy = \int \int_C f_{X, Y} (x, y) dx dy
\]
Стало быть, искомая вероятность
\begin{align*}
\P(X-Y > 2) &= \int \int_C f_{X, Y} (x, y) dx dy = \int_{2}^{+\infty} \left[ \int_{0}^{x-2} f_{X, Y} (x, y) dy \right] dx \\
&= \int_{2}^{+\infty} \left[ \int_{0}^{x-2} 0.005e^{-0.05x-0.1y}dy \right] dx \\
&= \int_{2}^{+\infty} \left[ 0.005e^{-0.05x} \cdot (-10e^{-0.1y}) \bigg\vert_{y=0}^{y=x-2} \right] dx  \\
&=  \int_{2}^{+\infty} \left[ 0.005e^{-0.05x} \cdot\left(1-e^{-0.1(x-2)}  \right) \right] dx  = \int_{2}^{+\infty} 0.005e^{-0.05x} dx \\
&- \int_{2}^{+\infty} 0.005e^{-0.05x-0.1x+0.2} dx
= 0.05 \cdot \left( -\frac{1}{0.05}e^{-0.05x}  \right) \bigg\vert_{x=2}^{x=+\infty} \\
&- e^{0.02} \cdot 0.05 \cdot \left( \frac{1}{0.15} e^{-0.15x} \right) \bigg\vert_{x=2}^{x=+\infty}
= e^{-0.1} -\frac{1}{3} e^{-0.1} = \frac{2}{3}e^{-0.1}  \approx 0.6032
\end{align*}
\end{enumerate}
\item Для решения задачи воспользуемся хорошо известными соотношениями:
\begin{align*}
&\int_{-\infty}^{+\infty} \frac{1}{\sqrt{2\pi\sigma^2}} e^{-\frac{(x-\mu)^2}{2\sigma^2}} dx = 1 \\
&\int_{-\infty}^{+\infty} x\frac{1}{\sqrt{2\pi\sigma^2}} e^{-\frac{(x-\mu)^2}{2\sigma^2}} dx = \mu \\
&\int_{-\infty}^{+\infty} x^2 \frac{1}{\sqrt{2\pi\sigma^2}} e^{-\frac{(x-\mu)^2}{2\sigma^2}} dx = \mu^2 + \sigma^2
\end{align*}
\begin{enumerate}
\item Указанная в задании функция $f_X$ является плотностью распределения, так как она удовлетворяет двум условиям:  $f_X$ является неотрицательной и интеграл от функции $f_X$ в пределах от  $-\infty$ до $+\infty$ равен единице:
\[
\int_{-\infty}^{+\infty} f_X (x) dx = \frac{1}{2} \cdot \int_{-\infty}^{+\infty}  \frac{1}{\sqrt{2\pi}} e^{-\frac{(x-1)^2}{2}} dx  + \frac{1}{2} \cdot \int_{-\infty}^{+\infty}  \frac{1}{\sqrt{2\pi}} e^{-\frac{(x+1)^2}{2}} dx = 1
\]
\item $\E(X) =\int_{-\infty}^{+\infty} xf_X (x) dx =  \frac{1}{2} \cdot \int_{-\infty}^{+\infty} x \frac{1}{\sqrt{2\pi}} e^{-\frac{(x-1)^2}{2}} dx  + \frac{1}{2} \cdot \int_{-\infty}^{+\infty}  x \frac{1}{\sqrt{2\pi}} e^{-\frac{(x+1)^2}{2}} dx = 1- 1=  0$
\item  \begin{align*}
\E(X^2) &= \int_{-\infty}^{+\infty} x^2 f_X (x) dx  =
\frac{1}{2} \cdot \int_{-\infty}^{+\infty} x^2 \frac{1}{\sqrt{2\pi}} e^{-\frac{(x-1)^2}{2}} dx  +
\frac{1}{2}  \cdot \int_{-\infty}^{+\infty} x^2 \frac{1}{\sqrt{2\pi}} e^{-\frac{(x+1)^2}{2}} dx \\
&= \frac{1}{2}(1^2 + 1^2 + (-1)^2 + 1^2) =  2
\end{align*}
\item $\Var(X) = \E(X^2) - (\E(X))^2 = 2 - 0^2 = 2$
\end{enumerate}
\end{enumerate}



\subsection[2015-2016]{\hyperref[sec:kr_02_2015_2016]{2015-2016}}
\label{sec:sol_kr_02_2015_2016}


\begin{enumerate}
\item
\begin{enumerate}
\item $ \E({\xi_1 \cdot \xi_2}) = \int_{0}^1 \int_0^1 xy f(x,y) \, dx \, dy = \int_0^1 \int_0^1 \frac{1}{2}\cdot x^2y + \frac{3}{2}\cdot xy^2 \, dx \, dy = \int_{0}^1 \frac{y}{6} + \frac{3y^2}{4} \,dy = \frac{1}{3}$
\item $f_{\xi_1 | \xi_2} (x | y) = \frac{f_{\xi_1, \xi_2}(x, y)}{f_{\xi_2}(y)} = \frac{0.5x + 1.5y}{0.25 + 1.5y}, \text{ при } y \in (0,1)$
\item
\begin{align*}
\E(\xi_1 | \xi_2 = y) &= \int_0^1 x f_{\xi_1 | \xi_2} (x | y) dx = \int\limits_{0}^{1}  x \frac{0.5x + 1,5y}{0.25 + 1.5y} dx \\
&= \frac{1}{0.25 + 1,5y}  \left. \left( \dfrac{0.5x^3}{3} +  \dfrac{1.5yx^2}{2} \right) \right|_0^1  =  \frac{1/6 + 3/4y}{0.25 + 1.5y}
\end{align*}
\item
Для того, чтобы функция являлась совместной плотностью для пары случайных величин, должно выполнятся следующее:
\[
\int_{\Omega} kx f(x,y) \, dx \, dy = 1
\]
Вычислим, чему равняется левая часть:
\[
1 = \int_{\Omega} kx f(x,y) \, dx \, dy = \int_{0}^1 \int_{0}^1 kx \left(\frac{x + 3y}{2}\right) dx \, dy = \int_{0}^1 \frac{k}{6} + \frac{3ky}{4} \, dy = \frac{k}{6} + \frac{3k}{8} \Rightarrow
\]
\[
k = \frac{24}{13}
\]
\end{enumerate}
\item Заметим, что $\xi + \eta = 10$, тогда $\Cov(\xi, \eta) = \Cov(\xi, 10-\xi) = -\Var(\xi)$.

Представим случайную величину $\xi$ в виде суммы случайных величин $\xi = \xi_1 + \ldots + \xi_{10}$, где
\[
\xi_i = \begin{cases}
1, & \text{если у студента есть хотя бы один незачёт}, p=0.2 \\
0, & \text{иначе}, p=0.8
\end{cases} \quad i = 1, \ldots, 10
\]

Поскольку результаты каждого из стуентов независимы, $\Var(\xi) = 10\Var(\xi_1)$
\[
\Cov(\xi, \eta) = -10(1^2 \cdot 0.2 - \left(1\cdot 0.2)^2\right) = -1.6
\]

Так как случайные величины $\xi$ и $\eta$ связаны соотношением $\xi = 10 - \eta$, $\Corr(\xi, \eta)=-1$.

Подставив в $\Cov(\xi - \eta, \xi)$ выражение $\eta = 10 - \xi$, получим:
\[
\Cov(\xi - \eta, \xi) = 2 \Cov(\xi, \xi) = 2 \cdot 0.16 = 0.32
\]
Случайные величины $\xi - \eta$ и $\xi$ не являются независиыми.
\item Найдем ожидаемую доходность и риск портфеля $R = \alpha \xi + (1-\alpha) \eta$
для любого $\alpha$, тогда при $\alpha = 1$ получим результаты Пети,
при $\alpha = 0.5$ — результаты Васи.
\[
\E R = \alpha + (1 - \alpha) = 1 \: \, \forall \, \: \alpha \in [0,1]
\]

Находим дисперсию:
\[
\Var(R) = \alpha^2 \cdot 4 + (1-\alpha)^2 \cdot 9 - 6\alpha (1-\alpha) = 19\alpha^2 -24\alpha + 9 \to \min_{\alpha}
\]

Теперь, найдем оптимальное $\alpha$:
\[
\alpha = \frac{24}{38}
\]

Финальные цифры:
\[
\begin{cases}
\Var(R)^{P} = 4 \Rightarrow \sigma_{P} = 2 \\
\Var(R)^{V} = 1.75 \Rightarrow \sigma_{V} \approx 1.32 \\
\Var(R)^{M} = \frac{27}{19} \Rightarrow \sigma_{M} \approx 1.19 \\
\end{cases}
\]
\item
\begin{enumerate}
\item Пусть $S$ количество мальчиков, тогда используя \href{https://en.wikipedia.org/wiki/Markov%27s_inequality}{неравенство Маркова} получаем:
\[
\P(S \ge 750) \le \frac{\E(S)}{750} = \frac{2}{3}
\]
\item Пусть, теперь, $\bar{X}$ доля мальчиков, то есть, $\bar{X} = \sum_{i=1}^n X_i /n$, где
\[
X_i =
\begin{cases}
1, \text{ если }i\text{-ый ребёнок — мальчик }\\
0, \text{ иначе }
\end{cases}
\]
тогда используя \href{https://en.wikipedia.org/wiki/Markov%27s_inequality}{неравенство Чебышева} получаем:
\[
\P(|\bar{X} - 0.5| \ge 0.25) \le \frac{\Var(\bar{X})}{0.25^2} = \frac{1/4000}{0.25^2} = 0.004
\]
\item Вероятность из предыдущего пункта можно записать в таком виде:
\begin{align*}
\P\left(\left|\bar{X} - 0.5\right| \ge 0.25\right) &= \P\left(\bar{X} \ge 0.75\right) + \P\left(\bar{X} \le 0.25\right) = 2\P\left(\bar{X} \ge 0.75\right) = \\
&= 2\P(\cN(0;1)\geq 0.25\sqrt{4000})=2\P(\cN(0;1)\geq 15.8) = 1.3 \cdot 10^{-56} \approx 0
\end{align*}
\end{enumerate}
\item Пусть случайная величина $S$ —  это валютный курс через полгода. Заметим, что $S = 100 + \delta_1 + \ldots + \delta_{171}$.
Тогда по ЦПТ $S \sim \cN(142.75, 203.0625)$. Теперь можно искать нужную вероятность:
\[
\P(S > 250) = \P \left(\frac{S -  142.75}{\sqrt{203.0625}} > \frac{250-142.75}{\sqrt{203.0625}} \right) = \P(\cN(0, 1) > 7.6) \approx 0
\]
%\item $\lambda p$
\end{enumerate}



\subsection[2014-2015]{\hyperref[sec:kr_02_2014_2015]{2014-2015}}
\label{sec:sol_kr_02_2014_2015}


\begin{enumerate}
\item
\begin{enumerate}
\item
Так как $(X, Y)$ имеют совместное равномерное распределение,
$\P\left\{X+Y > 1/2 \right\}$ можно рассчитать как отношение
соответствующих площадей:

\begin{center}
\definecolor{zzttqq}{rgb}{0.6,0.2,0.}
\begin{tikzpicture}[line cap=round,line join=round,>=triangle 45,x=5.692307692307692cm,y=5.692307692307692cm]
\draw[->,color=black] (-0.1,0.) -- (1.2,0.);
\foreach \x in {,0.2,0.3,0.4,0.5,0.6,0.7,0.8,0.9,1.,1.1}
\draw[shift={(\x,0)},color=black] (0pt,2pt) -- (0pt,-2pt) node[below] {\footnotesize $\x$};
\draw[->,color=black] (0.,-0.1) -- (0.,1.2);
\foreach \y in {0.2,0.3,0.4,0.5,0.6,0.7,0.8,0.9,1.,1.1}
\draw[shift={(0,\y)},color=black] (2pt,0pt) -- (-2pt,0pt) node[left] {\footnotesize $\y$};
\draw[color=black] (0pt,-10pt) node[right] {\footnotesize $0$};
\clip(-0.1,-0.1) rectangle (1.2,1.2);
\fill[color=zzttqq,fill=zzttqq,fill opacity=0.1] (0.,0.5) -- (0.5,0.) -- (1.,0.) -- (0.,1.) -- cycle;
\draw (0.,1.)-- (1.,0.);
\draw [color=zzttqq] (0.,0.5)-- (0.5,0.);
\draw [color=zzttqq] (0.5,0.)-- (1.,0.);
\draw [color=zzttqq] (1.,0.)-- (0.,1.);
\draw [color=zzttqq] (0.,1.)-- (0.,0.5);
\draw (0.344196516991,0.436512272899) node[anchor=north west] {$S_0$};
\draw (0.129676687465,0.218325437741) node[anchor=north west] {$S_1$};
\draw (0.0233335241101,1.16441289103) node[anchor=north west] {$y$};
\draw (1.06109611823,0.08) node[anchor=north west] {$x$};
\end{tikzpicture}
\end{center}

Соответственно:
\[
\P\left(X+Y > \frac{1}{2}\right) = \frac{S_0}{S_0+S_1} = \frac{0.5-S_1}{0.5} = \frac{\frac{1}{2}-\frac{1}{8}}{\frac{1}{2}} = \frac{3}{4}
\]

\item
\[
f_Y(y) = \int \limits_{0}^{1-y} f_{XY}(x,y) dx
\]

Поэтому, нам сначала нужно найти $f_{XY}(x,y)$, которая для равномерного
распределения должна быть константой. Это можно сделать из условия:
\begin{align*}
&\int\limits_{0}^{1}\int \limits_{0}^{1-x} f_{XY}(x, y) dxdy = \int\limits_{0}^{1}\int \limits_{0}^{1-x} C dxdy = 1 \Rightarrow \\
& \int\limits_{0}^{1}\int \limits_{0}^{1-x} C dxdy = \int\limits_{0}^{1} C(1-x) dx = \left. \left( Cx-C\frac{x^2}{2}\right) \right| _{0}^{1} =
\frac{C}{2}  = 1
\end{align*}

Откуда имеем $f_{XY}(x, y) = C = 2$.
Теперь можем найти плотность распределения расходов Васи:
\[
f_Y(y) = \int \limits_{0}^{1-y} 2 dx = 2(1-y)
\]

\item В данном случае площади немного другие, но смысл тот же:

\begin{center}
\definecolor{zzttqq}{rgb}{0.6,0.2,0.}
\begin{tikzpicture}[line cap=round,line join=round,>=triangle 45,x=5.692307692307692cm,y=5.692307692307692cm]
\draw[->,color=black] (-0.1,0.) -- (1.2,0.);
\foreach \x in {,0.2,0.3,0.4,0.5,0.6,0.7,0.8,0.9,1.,1.1}
\draw[shift={(\x,0)},color=black] (0pt,2pt) -- (0pt,-2pt) node[below] {\footnotesize $\x$};
\draw[->,color=black] (0.,-0.1) -- (0.,1.2);
\foreach \y in {,0.2,0.3,0.4,0.5,0.6,0.7,0.8,0.9,1.,1.1}
\draw[shift={(0,\y)},color=black] (2pt,0pt) -- (-2pt,0pt) node[left] {\footnotesize $\y$};
\draw[color=black] (0pt,-10pt) node[right] {\footnotesize $0$};
\clip(-0.1,-0.1) rectangle (1.2,1.2);
\fill[color=zzttqq,fill=zzttqq,fill opacity=0.1] (0.,0.5) -- (0.,1.) -- (0.333333333333,0.666666666667) -- (0.333333333333,0.5) -- cycle;
\draw (0.,1.)-- (1.,0.);
\draw[dashed] (1/3, 2/3) -- (1/3,0);
\draw (0.1004007648034,0.729040237615) node[anchor=north west] {$S_0$};
\draw (0.343364031073,0.5908861537646) node[anchor=north west] {$S_1$};
\draw (0.0233335241101,1.16441289103) node[anchor=north west] {$y$};
\draw (1.06109611823,0.08) node[anchor=north west] {$x$};
\draw (0.,0.5)-- (0.5,0.5);
\draw [color=zzttqq] (0.,0.5)-- (0.,1.);
\draw [color=zzttqq] (0.,1.)-- (0.333333333333,0.666666666667);
\draw [color=zzttqq] (0.333333333333,0.666666666667)-- (0.333333333333,0.5);
\draw [color=zzttqq] (0.333333333333,0.5)-- (0.,0.5);
\end{tikzpicture}
\end{center}


\[\P\left( \left. X<\frac{1}{3} \hspace{1mm} \right| Y>\frac{1}{2}\right) = \frac{S_0}{S_0+S_1} = \frac{\frac{1}{8} - S_1}{\frac{1}{8}} = \frac{\frac{1}{8}-\frac{1}{72}}{\frac{1}{8}} = \frac{8}{9}\]

\item При $Y = 1/2$, $X$ распределен равномерно от 0 до $1/2$, поэтому его плотность
равна
\[
f_X(x) = \frac{1}{\frac{1}{2} - 0} = 2
\]
Соответственно, условное математическое ожидание:
\[
\E \left( X \left| Y =\frac{1}{2}\right.\right) = \frac{1}{4}
\]

\item $\E(\E(X|Y)) = \E(X)$, а маргинальную функцию плотности для $X$ мы можем найти так же, как искали для $Y$, и получим $f_X(x) = 2(1-x)$. Отсюда:
\[
\E(X) = \int \limits_{0}^{1} 2x(1-x)dx = \left .\left(x^2 - \frac{2}{3}x^3\right) \right|^{1}_{0} = \frac{1}{3}
\]

\item Если вспомнить формулу для корреляции:
\[
\rho_{XY} = \frac{\Cov(X, Y)}{\sigma_X\sigma_Y}  = \frac{\E(XY) - \E(X)\E(Y)}{\sigma_X\sigma_Y}
\]

то станет более-менее очевидно, что надо найти $\E(XY)$ и дисперсии $X$ и $Y$.

\begin{align*}
\E(XY) &= \int \limits_{0}^{1} \int \limits_{0}^{1-x} 2xy dxdy = \int \limits_{0}^{1} 2xdx \int \limits_{0}^{1-x}ydy =  \int \limits_{0}^{1}x(x^2-2x+1)dx = \\
&= \left. \left(\frac{x^4}{4} - \frac{2}{3}x^3 + \frac{x^2}{2}\right) \right|^{1}_{0} = \frac{3}{4}-\frac{2}{3} = \frac{1}{12}
\end{align*}

Соответственно:

\[
\Cov(X, Y) = \frac{1}{12} - \frac{1}{3}\cdot \frac{1}{3} = -\frac{1}{36}
\]

Найдем теперь дисперсии $X$ и $Y$ (они будут одинаковыми, как и математические ождания, в силу симметрии):

\[
\E\left(X^2\right) = \int \limits_{0}^{1} 2x^2(1-x)dx = \left. \left( \frac{2}{3}x^2 - \frac{x_4}{2} \right) \right|_{0}^{1} = \frac{1}{6}
\]

Поэтому:
\[
\Var(X) = \E\left(X^2\right) - (\E(X))^2 = \frac{1}{6} - \frac{1}{9} = \frac{1}{18}
\]

Теперь наконец-то можем найти корреляцию:
\[
\rho_{XY} = -\frac{\frac{1}{36}}{\sqrt{\frac{1}{18}}\sqrt{\frac{1}{18}}} = -\frac{1}{2}
\]
\end{enumerate}

\item

\begin{enumerate}

\item Закон больших чисел гласит, что $\bar{X} \rightarrow
\E(X)$ при $n\rightarrow \infty$. Проверим его выполнение в данном случае:
\[
\E(X_n) = \frac{1}{2n}(-\sqrt{n}) + \left(1-\frac{1}{n}\right)\cdot0 + \frac{1}{2n}\sqrt{n} = 0
\]

\[
\lim\limits_{n\rightarrow\infty} \bar{X} = \lim\limits_{n\rightarrow\infty} \frac{X_1 +\dots + X_n}{n} = 0
\]
так как числитель ограничен, а знаменатель бесконечно возрастает.
Видим, что ЗБЧ в данном случае, конечно, выполняется.

Как вариант, можно было сказать, что дисперсия ограничена, и из этого также следует выполнение ЗБЧ.
\item Неравенство Чебышева:
\[
\P(|X-\E(X)|\geqslant \varepsilon) \leqslant \frac{\Var(X)}{\varepsilon^2}
\]

Соответственно, искомую вероятность можем оценить следующим образом:
\[
\P\left(\left|\bar{X}\right| \leqslant 1\right) = 1 -\P\left(\left|\bar{X}\right| \geqslant 1\right) \Rightarrow \P\left(\left|\bar{X}\right| \leqslant 1\right) \geqslant 1 - \frac{\Var\left(\bar{X}\right)}{1}
\]
\[
\Var\left(\bar{X}\right) = \Var\left(\frac{\sum\limits_{i=1}^{n} X_i}{n}\right) = \frac{1}{n^2}\sum \limits_{i=1}^{n} \Var{X_i}
\]
В свою очередь:

\[
\E\left(X_i^2\right) = 2\cdot\frac{1}{2n}\cdot n + \left(1-\frac{1}{n}\right)\cdot0 = 1 \Rightarrow \Var(X_i) = 1 \Rightarrow \Var\left(\bar{X}\right) = \frac{1}{n}
\]

Поэтому:
\[
\P\left(\left|\bar{X}\right| \leqslant 1 \right) \geqslant 1 - \frac{1}{n}
\]

\item
\[
1 - \frac{1}{n} = 0.9  \Rightarrow n = 10
\]

\end{enumerate}

\item

Обозначим за $R$ — необходимое количество наличных денег в банке. Пусть $X$ — случайная величина, показывающее размер суммарной выплаты $60$ ($n$ — достаточное большое для применения ЦПТ) клиентам. Ясно, что т.к. выплаты отдельным клиентам независимы: \( \E X = 60 \cdot 5000 = 3\cdot 10^5 \); \( \Var X = 60 \cdot 2000^2 = 2.4 \cdot 10^8 \); \( \sigma_X = \sqrt{2.4} \cdot 10^4 \approx 1.55 \cdot 10^4\)

Теперь по ЦПТ:
\begin{align*}
&\P(R \geqslant X) = 0.95 \\
&\P \left(\frac{X - \E X}{\sigma_X} \leqslant \frac{R - \E X}{\sigma_X} \right) = 0.95 \\
&\P \left( Z \leqslant \frac{R - 3 \cdot 10^5}{1.55 \cdot 10^4} \right) = 0.95
\end{align*}
Слева функция распределения; подставляя 95-\% квантиль стандартного нормального распределения, получаем:
\[
\frac{R - 3 \cdot 10^5}{1.55 \cdot 10^4} = 1.64 \Rightarrow R = 325420
\]


\item

\begin{enumerate}
\item По предельной теореме Муавра-Лапласа:
\[ \frac{\hat{p} - p}{\sqrt{p(1-p)/n}} \sim \cN (0,1) \]
\[ \P \left( \frac{|\hat{p} - p|}{\sqrt{p(1-p)/n}} \leqslant \frac{0.1}{\sqrt{p(1-p)/n}} \right) \geqslant 0.99 \]
\[ \P \left( |Z| \leqslant \frac{0.1}{\sqrt{p(1-p)/n}} \right) \geqslant 0.99 \]
Из симметричности стандартного нормального распределения и зная его 99.5-\% квантиль, равный приблизительно 2.58, получаем:
\[ \frac{0.1}{\sqrt{p(1-p)/n}} \geqslant 2.58 \]
\[ \frac{\sqrt{n}}{\sqrt{p(1-p)}} \geqslant \frac{2.58}{0.1} \]
\[ \sqrt{n} \geqslant \frac{2.58}{0.1} \sqrt{p(1-p)} \]
\[ n \geqslant 665.64 \cdot p(1-p) \]

С помощью неравенства Чебышева:
\[ \P \left( |\hat{p} - p| \leqslant 0.1 \right) \geqslant 0.99 \]
\[ \P \left( |\hat{p} - p| \geqslant 0.1 \right) \leqslant 0.01 \]
Теперь просто смотрим на неравенство Чебышева и на строчку выше, на неравенство Чебышева и на строчку выше\ldots
\[ \frac{p(1-p)/n}{0.1^2} = 0.01\]
\[ n = 10^4 p(1-p) \]
Принимаются оба ответа!

\item По предельной теореме Муавра-Лапласа:
\[ \P \left( \frac{|\hat{p} - p|}{\sqrt{p(1-p)/n}} \leqslant \frac{\varepsilon}{\sqrt{p(1-p)/1000}} \right) \geqslant 0.99 \]
\[ \P \left( |Z| \leqslant \frac{\varepsilon}{\sqrt{p(1-p)/1000}} \right) \geqslant 0.99 \]
Аналогично пункту 1:
\[ \frac{\varepsilon}{\sqrt{p(1-p)/1000}} \geqslant 2.58 \]
\[ \varepsilon \geqslant 0.082 \sqrt{p(1-p)} \]

С помощью неравенства Чебышева:
\[ \P \left( |\hat{p} - p| \leqslant \varepsilon \right) \geqslant 0.99 \]
\[ \P \left( |\hat{p} - p| \geqslant \varepsilon \right) \leqslant 0.01 \]
Аналогично пункту 1:
\[ \frac{p(1-p)/1000}{\varepsilon^2} = 0.01\]
\[ \varepsilon^2 = \frac{p(1-p)}{10} \]
\[ \varepsilon = \sqrt{\frac{p(1-p)}{10}} \approx 0.316 \sqrt{p(1-p)} \]

\end{enumerate}

Нужно было показать, как мастерство владения теоремой Муавра-Лапласа, так и неравенством Чебышева.

\end{enumerate}




\subsection[2013-2014]{\hyperref[sec:kr_02_2013_2014]{2013-2014}}
\label{sec:sol_kr_02_2013_2014}


\begin{enumerate}
\item
\begin{enumerate}
\item
\begin{align*}
\P\left(Y<X^2\right) &= \int_0^1 \int_0^{x^2} (x+y) dy dx = \left. \int_0^1 \left(xy \frac{y^2}{2} \right) \right|_0^{x^2} dx =  \int_0^1 \left(x^3 + \frac{x^4}{2} \right) dx = \\
&= \left. \frac{x^4}{4} + \frac{x^5}{10} \right|_0^1 = 0.35
\end{align*}
\item $f_X (x) = \int_0^1 (x+y) dy = \left. xy + \frac{y^2}{2} \right|_0^1 = x + 0.5$

$\E(X) = \int_0^1 f_X (x) \cdot x dx = \int_0^1 (x^2 + 0.5x) dx = 7/12$
\item $f_{X|Y=2}(x) = \frac{f_{X, Y}(x, 2)}{f_Y (2)} = \frac{x+2}{2.5}$
\end{enumerate}
\item
\begin{enumerate}
\item Из условия находим, что $X \sim \cN (0, 9)$, тогда
\[
\P(X>1) = \P \left(\frac{X-0}{3} > \frac{1-0}{3} \right) = \P\left(\cN(0,1) > \frac{1}{3} \right) = 0.37
\]
\item Подготовимся: $\E(2X+Y) = 0$, $\Var(2X+Y) = 4\Var(X) + \Var(Y) + 4 \Cov(X, Y) = 36$
\[
\P(2X+Y > 3) = \P\left(\frac{2X+Y-0}{6} > \frac{3-0}{6} \right) = \P(\cN(0, 1) > 0.5) = 0.31
\]
\item $ \P(2X+Y > 3 | X=1) = \P(Y > 1) = \P \left(\frac{Y-0}{2} > \frac{1}{2} \right) = \P(\cN(0, 1) > 0.5) = 0.31$
\item Заметим, что $\frac{X^2}{9}+\frac{Y^2}{4} \sim \chi^2_2$,
и тогда по таблице находим, что $\P\left(\frac{X^2}{9}+\frac{Y^2}{4} >12\right) = 0.0025$
\item Совместная функция плотности имеет вид:
\[
f(x, y) = \frac{1}{2\pi}\cdot \frac{1}{2\cdot 3\sqrt{1- \left(-\frac{1}{6}\right)^2}}\cdot \exp \left(- \frac{1}{2} \cdot \frac{1}{4\cdot9} \left(1- \left(-\frac{1}{6}\right)^2\right) \left(9x^2 + 2xy + 4y^2 \right) \right)
\]
\end{enumerate}
\item
\begin{enumerate}
\item Заметим, что $\frac{X_1}{\sqrt{\frac{X_3^2+X_4^2+X_5^2}{3}}} \sim t_3$.
По таблице находим искомую вероятность: $0.15$.
\item
\item Заметим, что $\frac{X_1^2}{X_2^2+X_3^2} \sim F_{1, 2}$.
Нужное значение находим в таблице: $0.95$.
\end{enumerate}
\item
\begin{enumerate}
\item По неравенству Чебышёва:
$\P(|X_i-0.5|\geqslant 0.3) \leqslant \frac{1/12}{9/100} = \frac{25}{27}$
\item По неравенству Маркова: $\P(X_i \geqslant 0.8) \leqslant \frac{5}{8}$
\item $\E\left(\bar{X}\right) = \frac{1}{2}$,
$\Var\left(\bar{X}\right) = \frac{1}{36\cdot 12}$, $\bar{X}\sim\cN(\frac{1}{2}, \frac{1}{36\cdot12})$

$\P\left(\bar{X} > 0.8\right) = \P\left(\frac{\bar{X} - \frac{1}{2}}{\sqrt{ \frac{1}{36\cdot 12}}} \geq \frac{0.8-0.5}{ \frac{1}{36\cdot 12}} \right) = \P(\cN(0,1) \geq 6.235) \approx 0$
\item $ \P(|X_i-0.5|\geqslant 0.3) = 1 - \P(|X_i-0.5|\geqslant 0.3) = 1 - \P(-0.3 \leqslant X_i - 0.5 \leqslant 0.3) = 0.4$
\item Нужно воспользоваться неравенством Берри-Ессеена.
\[
\E\left(|X_1 - 0.5|^3\right) = \int_0^1 |x_1 -0.5|^3 \cdot 1 dx = 2 \int_{0.5}^1 (x_1 - 0.5)^3 dx = \frac{1}{2^5}
\]
\item $\P\left(\bar{X} - 0.5 > 0.3\right) = \frac{25}{27n} \to_{n \to \infty} 0$
\end{enumerate}
\item $\P\left(\hat{p} \leq 0.25\right) = \P\left(\frac{\hat{p} - 0.2}{\sqrt{\frac{0.2\cdot0.8}{n}}} \leq \frac{0.25}{\sqrt{\frac{0.2\cdot0.8}{n}}}\right) = 0.99$

По таблице: $\frac{0.25}{\sqrt{\frac{0.2\cdot0.8}{n}}} = 2.33 \Rightarrow n = 348$
\item
\begin{enumerate}
\item $3, 4, 5, 7, 8, 9$
\item $F(x) = \begin{cases}
0 &  x < 3 \\
1/6 & 3 < x \leq 4 \\
2/6 & 4 < x \leq 5 \\
3/6 & 5 < x \leq 7 \\
4/6 & 7 < x \leq 8 \\
5/6 & 8 < x \leq 9 \\
1 &  x > 9 \\
\end{cases}$
\item $\bar{X} = 6$, $\widehat{\Var}(X) = 28$
\end{enumerate}
\end{enumerate}




\subsection[2012-2013]{\hyperref[sec:kr_02_2012_2013]{2012-2013}}
\label{sec:sol_kr_02_2012_2013}

\begin{enumerate}
\item  $f(s,t)=f(s)\cdot f(t|s)=\frac{1}{3s}$ при $0\leq t\leq s\leq 3$.
Бонус тем, кто прочитал условие, $\P(S>T)=1$.
\[
\E\left(T^2\right) = \int_0^3\int_0^s \frac{t^2}{3s}\,dt\,ds = 1
\]
\item
\begin{enumerate}
\item $\P(X>0.5) = \P(Z > 0.13) \approx 0.45$, $\sigma_X \approx 0.38$
\item $\P(X+Y>0.5) =
\P(Z>-0.65)\approx 0.74$,
$\sigma_{X+Y} \approx 0.17$, $\E(X+Y) = 0.61$

\item $X = 0.25$ при нормировке даёт $\tilde{X} = -0.53$.
Получаем:
\begin{align*}
& \E\left(\tilde{Y} \mid \tilde{X} = -0.53\right) = 0.47 \\
& \Var\left(\tilde{Y} \mid \tilde{X} = -0.53\right) = 0.19
\end{align*}

Значит $\E\left(Y \mid \tilde{X} = -0.53\right) = 0.34$,
$\Var\left(Y \mid \tilde{X} = -0.53\right)=0.027$.

\item $\P\left(Y > 1/3 \mid \tilde{X} = -0.53\right) = \P(Z > -0.04) = 0.52$
\item Ноль
\end{enumerate}
\item $\E(\hat{p})=0.5$, $\Var(\hat{p})=0.25/n=1/16000$. По Чебышёву:
\[
\P(|\hat{p}-0.5|\leq 0.01)\geq 1-\frac{\Var(\hat{p})}{0.01^2}=\ldots=0.375
\]
Используя нормальную аппроксимацию:

\[
\P(|\hat{p}-0.5|\leq 0.01)=\P(|Z|\leq 1.26)\approx 0.79
\]
\item Обозначим $N$ — количество подключенных абонентов, тогда $N\sim Bin(n,0.3)$.
При больших $n$ биномиальное распределение можно заменить на нормальное,
$N\sim \cN(0.3n,0.21n)$.

\[
\P(120N>1\,080\,000)=\P(N>9000)=\P\left(Z>\frac{9000-0.3n}{\sqrt{0.21n}}\right)=0.99
\]

Из таблицы находим, что

\[
\frac{9000-0.3n}{\sqrt{0.21n}} = -2.33
\]

Решаем квадратное уравнение, находим корни, один — отрицательный, другой, $n\approx 30622$.
\item

Вариационный ряд: $3$, $5$, $6$, $7$, $8$, $9$. $\bar{X}\approx 6.3$,
$\frac{\sum (X_i-\bar{X})^2}{n-1}\approx 4.7$,
$\frac{\sum (X_i-\bar{X})^2}{n}\approx 3.9$


\begin{minipage}{0.6\textwidth}
\begin{center}
\includegraphics[scale=0.5]{auto_figures_tikz/2012_2013_fig_02_empirical_dist.pdf}
\end{center}
\end{minipage}
\end{enumerate}



\subsection[2011-2012]{\hyperref[sec:kr_02_2011_2012]{2011-2012}}
\label{sec:sol_kr_02_2011_2012}

\begin{enumerate}
\item
\begin{enumerate}
\item $\P(X + Y > 1) = 4/5$. Здесь нужно брать интеграл...
\item $\E(X) = 13/20 = 0.65$, $\E(XY) = 2/5 = 0.4$, $\Cov(X, Y) = -9/400 = -0.0225$
\item Нет, так как функция плотности не раскладывается в произведение $h(x) \cdot g(y)$.
\item Да, так как функция плотности симметрична по $x$ и $y$
\end{enumerate}
\item
\begin{enumerate}
\item Заметим, что величина $|X_i|$ распределена равномерно на $[0; b]$,
поэтому $\E(|X_i|) = b/2$ и $\Var(|X_i|) = b^2/12$. Значит $\E(\hat{b}) = cb$ и для
несмещённости $c = 1$.
\item Находим $MSE$ через $b$ и $c$:
\[
MSE=\Var(\hat{b})+(\E(\hat{b})-b)^2=2c^2\cdot \frac{b^2}{12}+(c-1)^2\cdot b^2=b^2\left(\frac{7}{6}c^2-2c+1\right)
\]
Отсюда $c=\frac{6}{7}$.
\end{enumerate}
\item
\begin{enumerate}
\item $\E(\hat{\mu}_1)=6\mu/6=\mu$, несмещённая
\item $\E(\hat{\mu}_2)=\alpha \mu+\beta \mu$ и $\Var(\hat{\mu}_2)=\alpha^2 \frac{\sigma^2}{3}+\beta^2 \frac{2\sigma^2}{3}$.
Для несмещённости необходимо условие $\alpha+\beta=1$. Для минимизации дисперсии
получаем уравнение
\[
\alpha-2(1-\alpha)=0
\]
Отсюда оценка имеет вид $\frac{2}{3}\bar{X}+\frac{1}{3}\bar{Y}$
\end{enumerate}
\item
\begin{enumerate}
\item $S=X_1+X_2+X_3$, слагаемых мало, использовать нормальное распределение
некорректно. Можно использовать неравенство Чебышева, $\E(S)=27$, $\Var(S)=27$, поэтому
\[
\P(S\in [20;34])=\P( |S-\E(X)| \leq 7) \geq 1-\frac{27}{7^2}=\frac{22}{49}
\]
\item Используем неравенство Маркова:
\[
\P(X_1 \geq 12)\leq \E(X_1)/12=9/12=0.75
\]
\item Если $S=X_1+\ldots+X_{50}$, то можно считать, что $S\sim \cN(450;450)$, поэтому
\[
\P(S \in [430;470])\approx \P( N(0;1) \in [-0.94;+0.94])\approx 0.6528
\]
\end{enumerate}
\item
\begin{enumerate}
\item Если $Y = X_1 + X_2$, то $\E(Y) = 3$ и $\Var(Y) = 1 + 9 - 2 = 8$, значит
$\P(Y > 1) = \P(\cN(0,1) > -2/\sqrt{8}) \approx \P(\cN(0,1) > -0.71) \approx 0.7602$
\item Находим $\Cov(X_1, Y) = 1 - 1 = 0$. Итого: вектор имеет совместное нормальное
распределение с
\[
(X_1,Y) \sim \cN\left(
\left(\begin{array}{l}
{1} \\
{3}
\end{array}\right);
\left(\begin{array}{cc}
{1} & {0} \\
{0} & {8}
\end{array}\right)
\right)
\]
\item Стандартизируем величины. То есть мы хотим представить их в виде:
\[
\begin{cases}
X_1 = 1 + aZ_1 + bZ_2 \\
X_2 = 2 + cZ_2
\end{cases}
\]
Единица и двойка — это математические ожидания $X_1$ и $X_2$. Мы хотим, чтобы
величины $Z_1$ и $Z_2$ были $\cN(0,1)$ и независимы.
Получаем систему:
\[
\begin{cases}
\Var(X_1) = 1 \\
\Var(X_2) = 9 \\
\Cov(X_1,X_2) = -1
\end{cases} \Leftrightarrow
\begin{cases}
a^2 + b^2 = 1 \\
c^2 = 9 \\
bc = -1
\end{cases}
\]
Одно из решений этой системы: $c = 3$, $b = -1/3$, $a = 2\sqrt{2}/3$

Используя это разложение получаем:
\begin{multline*}
\left( X_1 \mid X_2 = 2\right) \sim \left(1 + \frac{2\sqrt{2}}{3}Z_1-\frac{1}{3}Z_2 \mid 2 + 3Z_2 = 2\right)\sim \\
\sim\left(1 + \frac{2\sqrt{2}}{3}Z_1 - \frac{1}{3}Z_2\mid Z_2 = 0\right) \sim \left(1 + \frac{2\sqrt{2}}{3}Z_1\right)\sim \cN(1;8/9)
\end{multline*}

Еще возможные решения: выделить полный квадрат в совместной функции плотности, готовая формула, etc
\end{enumerate}
\item
\begin{enumerate}
\item $\Var(\hat{p}) = \frac{p(1-p)}{n}$. Максимально возможное значение $p(1-p)$
равно $1/4$, поэтому максимально возможное значение $\Var(\hat{p})=1/4n$.
\item У нас задано неравенство:
\[
\P(|\hat{p} - p| > 0.02) < 0.1
\]
Делим внутри вероятности на $\sqrt{\Var(\hat{p})}$:
\[
\P\left(|\cN(0;1)| > 0.02 \sqrt{4n} \right) < 0.1
\]
По таблицам получаем $0.02 \sqrt{4n} \approx 1.65$ и $n \approx 1691$

Если вместо ЦПТ использовать неравенство Чебышева, то можно получить менее точный
результат $n=6250$.
\end{enumerate}
\item
\begin{enumerate}
\item $\E(X_i) = (1 + 10) / 2 = 5.5$, $\E\left(X_1^2\right) = \frac{1}{10} \frac{10\cdot 11\cdot 21}{6}=77/2$,
$\Var(X_i) = 33/4 = \sigma^2$.
Можно найти $\Cov(X_1, X_2)$ по готовой формуле, но мы пойдем другим путем.
Заметим, что сумма номеров всех вариантов — это константа, поэтому
$\Cov(X_1, X_1 + \ldots + X_{40}) = 0$. Значит, $\Var(X_1) + 39\Cov(X_1, X_2) = 0$.
В итоге получаем $\Cov(X_1, X_2) = -\sigma^2 / 39$
\item $\E(\bar{X})=11/2$, $\Var(\bar{X})=4\frac{1}{52}$
\item Да, являются, так как и $X_1$ и $X_2$ — это номер случайно выбираемого варианта
\item Нет, если известно чему равно $X_1$, то шансы получить такой же $X_2$ падают
\end{enumerate}
\item
Если мы наняли $n$ работников, то ожидаемое количество рабочих человеко-дней равно:
\[
\E(X)=365\cdot n\cdot \left(\frac{364}{365}\right)^{n}
\]
Для удобства берем логарифм $\ln(\E(X)=c+\ln(n)+n\ln(364/365)$ и получаем условие
первого порядка $1/n+\ln(364/365)=0$. Пользуясь разложением в ряд Тейлора
$\ln(1+t)\approx t$ получаем: $1/n-1/365\approx 0$, $n\approx 365$
\end{enumerate}



\subsection[2010-2011]{\hyperref[sec:kr_02_2010_2011]{2010-2011}}
\label{sec:sol_kr_02_2010_2011}


\begin{enumerate}
\item Перед нами функция плотности двумерного нормального распределения!

Поэтому: $\E(X) = 0$, $\Var(Y)=1$, $\Cov(X, Y) = \rho$
\item С помощью таблицы находим, что $\P(X > \sqrt{2}) = 1 - \P(X<1.14) \approx 0.13$

Заметим, что $X^2 + Y^2 \sim \chi^2_2$, и находим искомую вероятность по таблице: $\P(X^2 + Y^2 >4) \approx 0.87$

\item Вспомним важные формулы:
\[
\E(X | Y = y) = \E(X) + \Cov(X, Y) \cdot \Var^{-1}(Y) \cdot (y - \mu_y)
\]
\[
\Var(X | Y = y) = \Var(X) - \Cov(X, Y) \Var^{-1}(Y) \cdot \Cov(Y, X)
\]
Применив их, получим: $\E(X | Y = 0) = \frac{22}{9}$, $\Var(X | Y = 0) = \frac{32}{9}$. Тогда:
\[
\P(X > 0 | Y = 0) = \P\left(\cN(0,1) > \frac{0 - \frac{22}{9}}{\sqrt{\frac{32}{9}}}\right) \approx 0.9
\]
Далее, найдём дисперсию портфеля и минимизиреум её по $\alpha$:
\begin{multline*}
\Var(\alpha X + (1-\alpha) Y) = \alpha^2 \Var(X) + (1-\alpha)^2 \Var(Y) + 2 \alpha (1-\alpha) \Cov(X, Y) =\\
= 4 \alpha^2 + 9(1-\alpha)^2 -4\alpha(1-\alpha)  = 17 \alpha^2 - 22 \alpha + 5 \to \min_{\alpha}
\end{multline*}
\[
\alpha = \frac{11}{17}
\]
Нельзя, так как из $\Cov(X+Y, 7X - 2Y) = 0$ не следует независимость $X+Y$ и $7X - 2Y$.
\item Сначала подготовимся и найдём дисперсию случайной величины $X$:
\[
\E(X) = \int_1^{\infty} \frac{3}{x^4} \cdot x dx = \left. \frac{3x^{-2}}{-2} \right|_1^{\infty} = \frac{3}{2}
\]
\[
\E(X^2) = \int_1^{\infty} \frac{3}{x^4} \cdot x^2 dx = \left. \frac{3x^{-1}}{-1} \right|_1^{\infty} = 3
\]
\[
\Var(X) = 3 - \left(\frac{3}{2}\right)^2 = \frac{3}{4}
\]
Перепишем исходное неравенство в виде: $\P(|\bar{X} -\E(X)|<0.1) \geq 1 - 0.02$.
\[
\frac{\Var(\overline{X})}{0.1^2} \leq 0.02 \Rightarrow \frac{\Var(X)}{n} \leq 0.0002 \Rightarrow n \geq 3750
\]
\item Нужно найти $\P(\hat{p} > \frac{60}{90})$. Воспользуемся теоремой Муавра-Лапласа:
\[
\P\left(\hat{p} > \frac{60}{90}\right) = \P \left(\frac{\hat{p} - 0.8}{\frac{0.8 \cdot 0.2}{90}} > \frac{2/3 - 0.8}{\frac{0.8 \cdot 0.2}{90}} \right) = \P(\cN(0,1) > -3.16) \approx 1
\]
Найдём объём выборки:
\[
\P \left(\frac{\hat{p} - 0.8}{\frac{0.8 \cdot 0.2}{n}} > \frac{0.02}{\frac{0.8 \cdot 0.2}{n}} \right) = 0.95 \Rightarrow \frac{0.02}{\frac{0.8 \cdot 0.2}{n}} = 1.65 \Rightarrow n =33
\]
\item %$\E\left(\frac{X_1+X_2}{2}\right) = 40.5$

%$\Var\left(\frac{X_1+X_2}{2}\right) = 3.9$

%$\Cov(X_1, X_2) = -0.45$
\item Необоходимо решить следующую систему уравнений:
\[
\begin{cases}
D(\overline{X}_S) = \sum_{l=1}^L \left(\frac{w_l^2 \sigma_l^2}{n_l} \right) \to \min \\
n = n_1 + n_2 + n_3
\end{cases}
\]
Выпишем лагранжиан:
\[
L  = \sum_{l=1}^L \frac{w_l^2 \sigma_l^2}{n_l} - \lambda(n_1 + n_2 + n_3 -n)
\]
\[
\frac{\partial L}{\partial n_l} = 0 \Rightarrow n_l^o = \sqrt{\frac{w_l^2 \sigma_l^2}{-\lambda}} \Rightarrow \sum_{l=1}^L w_l \sigma_l = \sqrt{-\lambda} n \Rightarrow \sqrt{-\lambda} = \frac{\sum_{l=1}^L w_l \sigma_l }{n}
\]
Тогда находим объём выборки каждой группы по формуле: $n^o = \frac{w_l \sigma_l }{\sum_{k=1}^L w_k \sigma_k } n$
\begin{itemize}
\item $n^o_{\text{недорогие}} = 0.255n$
\item $n^o_{\text{средние}} = 0.532n$
\item $n^o_{\text{дорогие}} = 0.213n$
\end{itemize}
\item Выборочное среднее: $-0.2$; выборочная дисперсия: $70.98$;

вариационный ряд: $-5.6, -3.2, -0.8, 1.1, 2.9, 4.4$.
\item Для проверки свойства несмещённости найдём математические ожидания оценок:
\[
\E(T_1) = \E(2\overline{X}) = 2\E(X_1) = \theta
\]
\[
\E(T_2) = \E((n+1)X_{(1)}) = (n+1) \frac{\theta}{2}
\]
Несмещённой является только оценка $T_1$.

Для проверки оставшихся свойств посчитаем дисперсию оценок:
\[
\Var(T_1) = 4\Var(\overline{X}) = \frac{\theta^2}{3n} \to_{n \to \infty} 0
\]
\[
\Var(T_2) = (n+1)^2 \cdot \frac{\theta^2}{12}
\]
Оценка $T_1$ является более эффективной, и она состоятельна. $T_2$ не является состоятельной оценкой.
\item По формуле Байеса:
\[
\P(X=k | X+Y = k) = \frac{\P(X=k \cap X+Y = n)}{\P(X+Y=n)} = \frac{\P(X+Y = k | X=k)\P(X=k)}{\P(X+Y=n)}
\]
Находим числитель:
\[
\P(X+Y = n | X=k) \cdot \P(X=k) = \P(Y = n-k) \cdot \P(X=k) = p\cdot (1-p)^{n-k-1} \cdot p \cdot p^{k-1} = p^2 \cdot (1-p)^{n-2}
\]
И знаменатель:
\begin{align*}
\P(X+Y = n) &= \sum_{i=1}^n \P(X = i \cap Y = n-i) =  \sum_{i=1}^n p \cdot (1-p)^{i-1} \cdot p \cdot (1-p)^{n-i-1} \\
&= \sum_{i=1}^n p^2 (1-p)^n = np^2 (1-p)^n
\end{align*}
Итого:
\[
\P(X=k | X+Y = k) = \frac{ p^2 \cdot (1-p)^{n-2}}{np^2 (1-p)^n} = \frac{1}{n(1-p)^2}
\]
Второе выражение:
\[
\P(Y=k | X=Y) = \frac{\P(Y=k \cap X=Y)}{\P(X=Y)} = \P(Y=k) = p(1-p)^{k-1}
\]
\end{enumerate}



\subsection[2009-2010]{\hyperref[sec:kr_01_2009_2010]{2009-2010}}
\label{sec:sol_kr_01_2009_2010}

\begin{enumerate}
\item Из условия: $\Var(X) = 5^2 = 25$, $\Var(Y) = 4^2 = 16$, $\Var(X - Y) = 2^2 = 4$.
Есть такое тождество, $\Var(X - Y) = \Var(X) + \Var(Y) - 2\Cov(X,Y)$. Отсюда находим
$\Cov(X, Y) = 37/2$ и $\Corr(X, Y) = 37/40$.
\item По таблице: $\P(X<\sqrt{3}) = 0.9582$

Заметим, что $X^2 + Y^2 \sim \chi^2_2$. По таблице находим искомую вероятность:
$\P(X^2 + Y^2<6) = 0.95$
\item
\begin{enumerate}
\item $\P(|X - 16| > 4) \leq 0.75$
\item $\P(|X - 16| > 4) = 1 - \P(-4 < X-16 < 4) = 1 - \P(12 < X < 20) = \frac{1}{3}$
\item $\P(|X - 16| > 4) =  1 - \P\left(\frac{-4}{\sqrt{12}} < \frac{X-16}{\sqrt{12}}
< \frac{4}{\sqrt{12}}\right) = 2 \P\left(\cN(0, 1)<\frac{4}{\sqrt{12}}\right) = 0.75$
\end{enumerate}
\item $\E(2X + Y) = 2$, $\Var(2X + Y) = 3$

$\P(2X+Y>1) = 1 - \P\left(\frac{2X+Y - 2}{\sqrt{3}} < \frac{1-2}{\sqrt{3}}\right) =
1 - \P\left(\cN(0,1) < \frac{-1}{\sqrt{3}}\right)= 0.72$

$\P(2X+Y>1 \mid Y=2) = \P(2X>-1) = \P\left(\frac{X+1}{1} > \frac{-0.5+1}{1} \right) =
1 - \P(\cN(0,1)< -0.5) = 0.31$
\item Если $S$ — финальная стоимость акции, то $S = 1000 + X_1 + X_2 + \ldots + X_{100}$.
Тогда по ЦПТ $S\sim \cN(1000,100)$ и $\P(S > 1010) = \P(Z > 1)$.
\end{enumerate}



\subsection[2008-2009 Демо-версия]{\hyperref[sec:kr_01_2008_2009_demo]{2008-2009 Демо-версия}}
\label{sec:sol_kr_01_2008_2009_demo}

\begin{enumerate}
\item
$\P(Y>2X)=\int_0^{0.5} \int_{2x}^{1} (x+y) dydx = 5/24$

$f_X(x) = \int_0^1 (x+y) dy = x + \frac{1}{2} \Rightarrow \E(X) = \int_0^1 \left(x+\frac{1}{2}\right)x dx = \frac{7}{12}$
\item
\begin{enumerate}
\item $\E(X_1 + 3X_2) = -1$, $\Var(X_1 + 3X_2) = 207$

$\P(X_1 + 3X_2 > 20) = \P\left(\frac{X_1 + 3X_2 +1}{\sqrt{207}} > \frac{20+1}{\sqrt{207}} \right) = \P(\cN(0,1)>1.46) \approx 0.07$
\item $\E(X_1|X_2 = 0) = 2 - 4.5\cdot\frac{1}{25}(0+1) = 1.82$, $\Var(X_1 | X_2 = 0) = 9 - (-4.5) \cdot \frac{1}{25} \cdot(-4.5) = 8.19$

$X_1 | X_2 = 0 \sim \cN(1.82, 8.19)$
\end{enumerate}
\item
\begin{enumerate}
\item $\E(X)=1000$, $\Var(X) = 10^6$
\item $X$ — случайная величина, сумма выплат по одному контракту, $X \sim \exp(0.001)$

$\P(1000X>110000) = \P(X>100) = \int_{110}^{+\infty} 0.001 \exp(-0.001x) dx \approx 0.9$
\end{enumerate}
\item $\P\left(\frac{(X-30)^{2}}{\Var(X)}<3\right) = \P(|X-30|< \sqrt{3\Var(X)}) \geq 1 - \frac{\Var(X)}{ \sqrt{3\Var(X)}} = \frac{2}{3}$

$\P(|X-30|< \sqrt{3\Var(X)}) = \P(X< \sqrt{3\cdot900}+30) = \int_{0}^{82}\frac{1}{30} \exp\left(-\frac{1}{30}x\right)dx\approx 0.94$
\item $a \approx 1.28$
\item[9-Б.]
Подразумевая под точками концы гирлянды, докажем следующее утверждение.

Бросим $2n \geq 4$ точек $X_1, X_2, \ldots, X_{2n}$ случайным образом на отрезок $[0;1]$. Пусть для $1 \leq i \leq n$ $J_i$ — это отрезок с концами $X_{2i-1}$ и $X_{2i}$.
Тогда вероятность того, что найдётся такой отрезок $J_i$, который пересекает все другие отрезки, равна $2/3$ и не зависит от $n$.

Доказательство. Бросим $2n+1$ точек на окружность, тогда $2n$ точек образуют пары, а оставшуюся обозначим $X$ и будем считать её и началом, и концом отрезка.
Каждому получившемуся отрезку присвоим уникальное имя.
Далее, будем двигаться от точки $X$ по часовой стрелке до тех пор, пока не встретим одно и то же имя дважды, например «а».
После этого пойдём в обратную сторону, и будем идти, пока не встретим какое-нибудь другое имя дважды, например, «б».
Теперь посмотрим на получившуюся последовательность между «б» и «а» на концах пути, читая её по часовой стрелке от «б» до «а».
Нас интересует взаимное расположение $X$, второй «а» и второй «б».
Зная, что «а» стоит после $X$, выпишем все возможные случаи, где может стоять «б»:
\begin{enumerate}
\item перед $X$
\item между $X$ и «а»
\item после «а»
\end{enumerate}
Покажем, что во втором и третьем случае отрезок «б» пересекает все остальные, а в первом такого отрезка вообще нет. Попутно заметим, что появление каждого и случаев равновероятно.

Действительно, если «б» стоит после $X$, и отрезок соответствующий этому имени, не пересекает какой-нибудь другой отрезок «в», то последовательность выглядела бы как «бвв$X$б» или «б$X$ввб», что противоречит описанному построению.
Если «б» стоит перед $X$ и отрезок «в» пересекает оба отрезка «а» и «б», то мы снова приходим в противоречие с построением.
В итоге, получаем, что искомая вероятность равна $2/3$.
\end{enumerate}


\subsection[2008-2009]{\hyperref[sec:kr_01_2008_2009]{2008-2009}}
\label{sec:sol_kr_01_2008_2009}

\begin{enumerate}
\item
\begin{enumerate}
\item $\int_{0}^{1}\int_{x}^{1}p(x,y)dydx = 5/12$

$\int_{0}^{1}\int_{0}^{1}y\cdot p(x,y)dydx = 13/24$
\item Нет, так как совместная функция плотности не разлагается в произведение
индивидуальных
\end{enumerate}
\item
\begin{enumerate}
\item $\P(|X_{1}+X_{2}+...+X_{7}|>14)\leq \frac{7}{14^2}=\frac{1}{28}$

$\P(X_{1}^{2} + \ldots + X_{7}^{2} > 14) = \P(X_{1}^{2} + \ldots + X_{7}^{2} - 7 > 7)
= \P(|X_{1}^{2} + \ldots + X_{7}^{2} - 7| > 7) \leq \frac{2 \cdot 7}{7^2} = \frac{2}{7}$
\item $\P(|X_{1} + \ldots + X_{7}| > 14) = \P(|\cN(0;1)| > 14/\sqrt{7}) = \P(|\cN(0;1)| > 5.29) \approx 0$

$\P(X_{1}^{2} + X_{2}^{2} + \ldots + X_{7}^{2} > 14) \approx 0.05$
\end{enumerate}
\item
\begin{enumerate}
\item $X_{1} + 2X_{2} \sim \cN(5;89)$, $\P(Z > 1.59) = 0.056$
\[
\Var(X_1 + 2X_2) = \Var(X_1) + 4\Var(X_2) + 4\Cov(X_1, X_2) = 89
\]
\item Нормальное, причем $\cN(1.4;8)$, корреляция равна $-1/3$
\end{enumerate}
\item $\beta = \frac{1}{3}a^{2}$

$\E(XY) = \frac{3}{4}a^{2}$

$\E\left(Y^{2}\right) = 3a^{2}$

$\hat{\beta}_{1} = \frac{4}{9}XY$

$\hat{\beta}_{2} = \frac{1}{9}Y^{2}$

Так как обе оценки несмещенные вместо сравнения дисперсий можно сравнить квадраты
ожиданий

$\frac{16}{81}\E\left(X^{2}Y^{2}\right)$ vs $\frac{1}{81}\E\left(Y^{4}\right)$

\ldots

$16 a^4$ vs $\frac{81}{5} a^{4}$

Дисперсия васиной оценки меньше.
\item Заметим, что Пуассоновская величина с положительной вероятностью принимает
значение ноль, значит бывает, что монстры дохнут от одного устрашающего взгляда Васи :)
\begin{enumerate}
\item Сумма трех независимых пуассоновских величин - пуассоновская с параметром: $3\lambda=6$.

$\P(X=6)=e^{-6}\frac{6^6}{6!}\approx 0.16$

Ответ с факториалам считается полным.
\item Сумма 80 величин имеет пуассоновское распределение, но при большом количестве
слагаемых пуассоновское мало отличается от нормального.

$\E(S)=160$, $\Var(S)=160$

$\P(S>200)=\P\left(\frac{S-160}{\sqrt{160}}>3.16\right)\approx 0$
\end{enumerate}
\item
\begin{enumerate}
\item $\lambda = 1/10$, $\P(X < 7) = 0.5$
\item $\P(\bar{X} > 0.55) = \P(\cN(0;1) > \frac{0.05\sqrt{1000}}{0.5}) = \P(\cN(0;1) > 3.16) \approx 0$
\end{enumerate}
\item $\Var(X) = 5 \cdot 0.1 \cdot 0.9 + 5 \cdot 0.5 \cdot 0.5 = 1.7$

$\Var(Y) = 3 \cdot 0.1 \cdot 0.9 + 7 \cdot 0.5 \cdot 0.5 = 2.02$

Пусть $Z$ — число правильных ответов на вопросы с 3-го по 7-ой (у Пети и у Васи)
\begin{align*}
\Cov(X,Y) &= \Cov(Z + (X - Z), Z + (Y - Z)) = \Var(Z) \\
&+ \Cov(X - Z, Z) + \Cov(Z, Y - Z) + \Cov(X - Z, Y - Z) = \Var(Z)
\end{align*}
$Y - Z$ — это сколько правильных ответов дал лично Вася и оно не зависит от числа
$Z$ правильных списанных ответов, значит, $\Cov(Y-Z,Z)=0$.

Аналогично все остальные ковариации равны нулю.

$\Var(Z)=3\cdot 0.1\cdot 0.9+2\cdot 0.5\cdot 0.5=0.77$
\item Любые совпадения с курсом экономической и социальный статистики случайны и
непреднамеренны.

Чтобы оценка среднего по всем трем стратам была несмещена, она должна строиться
по формуле:

$\bar{X}=w_{1}\bar{X}_{1}+w_{2}\bar{X}_{2}+w_{3}\bar{X}_{3}$
(здесь $\bar{X}_{i}$ — среднее арифметическое по $i$-ой страте)

Поэтому $\Var(\bar{X})$ (минимизируемая функция) равняется:

$\Var(\bar{X})=\sum \frac{w^{2}_{i}\sigma^{2}_{i}}{n_{i}}$

Принцип кота Матроскина\footnote{«Чтобы продать что-нибудь ненужное, нужно сначала
купить что-нибудь ненужное. А у нас денег нет!»} (aka бюджетное ограничение):
$4n_{1}+16n_{2}+25n_{3}=7000$

Решаем Лагранжем и получаем ответ: 35, 35, 252.

Некоторые маньяки наизусть знают:

$n_{i}=\frac{C}{\sum w_{i}\cdot \sigma_{i}\cdot\sqrt{c_{i}}}\frac{w_{i}\cdot \sigma_{i}}{\sqrt{c_{i}}}$
\item[9-А.] Замечание: неудачные переноски считаются, так как иначе решение
тривиально — пробовать нести по 1000 шашек.
\begin{enumerate}
\item  Так как $p$ небольшая будем считать, что $\ln(1-p)\approx -p$. Уже страшно, да?
\item Допустим, что $s(n)$ оптимальная стратегия, указывающая, сколько нужно брать
сейчас шашек, если осталось перенести $n$ шашек. Возможно, что $s$ зависит от $n$.
Обозначим $e(n)$ ожидаемое количество переносов при использовании оптимальной стратегии.
\item Начинаем:

$s(1)=1$, $e(1)=1/(1-p)$

$s(n)=\argmin_{a}(1/(1-p)^{a}+e(n-a))$, $e(n)=\min_{a}(1/(1-p)^{a}+e(n-a))$

Замечаем, что поначалу (где-то до $1/p$ шашек) все идет хорошо, а затем плохо...
\item Ищем упрощенное решение вида $s(n)=s$.

Ожидаемое число переносок равно $\frac{1000}{s}\frac{1}{(1-p)^{s}}$

Минимизируем по $s$. Получаем: $s=-1/\ln(1-p)\approx 1/p$.
\item Для тех кому интересно, точный график (10000 шашек, p=0.01):
\end{enumerate}

$[$Здесь оставлено место для картины Усама-Бен-Ладен будь он не ладен таскает шашки.$]$
%\begin{figure}[h]
%    \includegraphics{usama.eps}
%\end{figure}

ps. В оригинале мы сканировали ксерокопию учебника Микоша. Сканер был очень умный:
в него нужно положить стопку листов, а на выходе он выдавал готовый pdf файл.
Проблема была в том, что он иногда жевал бумагу. В этом случае он обрывал
сканирование и нужно было начинать все заново. Возник вопрос, какого размера должна
быть партия, чтобы минимизировать число подходов к ксероксу.

\item[9-Б.]
\begin{enumerate}
\item Если сейчас 0 долларов, то брать 1 доллар.

Назовем ситуацию, «шоколадной» если можно выиграть без риска. То есть если игр
осталось больше, чем недостающее количество денег.
\item Если игрок не в шоколаде, то оптимальным будет рисковать на первом ходе.

Почему? Получение одного доллара можно перенести на попозже.
\item В любой оптимальной стратегии достаточно одного успеха для выигрыша.

Почему? Допустим стратегии необходимо два успеха в двух рискованных играх. Заменим
их  на одну рискованную игру. Получим большую вероятность.

Оптимальная стратегия:

Если сейчас 0 долларов, то брать доллар.

Пусть $d$ — дефицит в долларах, а $k$ — число оставшихся попыток.

Если $d\le k$, то брать по доллару.

Если $d>k$, то с риском попробовать захапнуть $1+d-k$ долларов.
\end{enumerate}
\end{enumerate}



\subsection[2007-2008 Демо-версия]{\hyperref[sec:kr_01_2007_2008_demo]{2007-2008 Демо-версия}}
\label{sec:sol_kr_01_2007_2008_demo}


\begin{enumerate}
\item $c = 0.4$

$\P(Y>X) = \P(Y=1, X=-1) + \P(Y=2, X=-1) + \P(Y=2, X=1) = 0.7$

$\E(XY) = 0.1 -0.4 - 0.4 -0.1 + 0.1 +0.2 =-0.5$

$\E(X|Y>0) = -1\cdot\frac{0.6}{0.8} + 1\cdot \frac{0.2}{0.8} = -0.5$

Случайны величины $X$ и $Y$ не являются независимыми.
\item
\begin{enumerate}
\item Найдём распредление случайной величины $Z = X_1 + X_2$:
\[
\E(Z) = -1, \Var(Z) = \Var(X_1) + \Var(X_2) + 2 \Cov(X_1, X_2) = 37
\]
Получили, что $Z \sim \cN(-1, 37)$, тогда
\[
\P(Z>0) = \P\left(\frac{Z+1}{\sqrt{37}} > \frac{0+1}{\sqrt{37}}\right) = 0.4364
\]
\item $\E(X_1 | X_2 = -1) = -2 -4 \cdot \frac{1}{36} \cdot (-1 -1) = -\frac{16}{9}$

$\Var(X_1 | X_2 = -1) = 9 - (-4) \cdot \frac{1}{36} \cdot (-4) = \frac{77}{9}$

$X_1 | X_2 = -1 \sim \cN(-\frac{16}{9}, \frac{77}{9})$
\end{enumerate}
\item $c=6$

$\P(3Y>X) = \int_{0}^{1} \int_{0}^{3y} 6(x-y) dx dy = 3$

$f_{X} (x) = \int_{0}^{x} 6(x-y) dy = 3x^2 \Rightarrow \E(X) = \int_{0}^{1} 3x^3 dx = 0.75$
\item Введём следующие случайные величины:

$
X = \begin{cases}
1 & \text{в субботу не будет дождя}, p=0.5 \\
0 & \text{иначе}, p=0.5
\end{cases}
$
\hspace{0.5cm}
$
Y = \begin{cases}
1 & \text{в воскресенье не будет дождя }, p=0.7 \\
0 & \text{иначе }, p=0.3
\end{cases}
$

Найдем их математические ожидания и дисперсии: $\E(X)=0.5$, $\Var(X)=0.25$,
$\E(Y)=0.3$, $\Var(Y)=0.21$.

В условии дана корееляция $X$ и $Y$, найдём ковариацию: $\Cov(X, Y) =
r \cdot 0.5 \sqrt{0.21}$.
По определению, $\Cov(X, Y) = \E(XY)-\E(X)\E(Y)$, откуда можно найти $\E(XY)$:
$\E(XY) = r \cdot 0.5 \sqrt{0.21} + 0.5 \cdot 0.7$.

Заметим, что $\E(XY)$ — это и есть искомая вероятность, потому что при подсчёте
совместного математического ожидания в~сумме будет только одно слагаемое, в~котором
$X = 1$ и $Y = 1$, остальные же будут равны нулю.
\item Пусть $X$ — случайная величина, обозначающая количество проданных книг.
Будем считать, что продажи каждой книги — независимые события.

$\E(50 + 5X) = 100$, $\Var(50 + 5X) = 25$
\item Пусть $X$ — случайная величина, обозначающая изменение цены акции за день,
a $S$ —  финальную стоимость акции.
\begin{enumerate}
\item $\E(S) = \E(1000 + 60X) = 1000 + 60 (0.5 \cdot 3 + 0.5 \cdot 5) = 1240$

$\Var(S) = \Var(1000 + 60X) = 3600(0.5 \cdot 9 + 0.5 \cdot 25 - 16) = 3600$
\item $\P(S > 900) = \P\left(\frac{S-1240}{60} > \frac{900-1240}{60} \right) =
1 - \P(\cN(0,1)< -\frac{17}{3}) \approx 1$
\end{enumerate}
\item $\P\left(|\hat{p} - 0.6| <0.01\right) = 0.99 \Rightarrow \P\left(\frac{|\hat{p}
- 0.6|}{\sqrt{\frac{0.6\cdot0.4}{n}}} < \frac{0.01}{\sqrt{\frac{0.6\cdot0.4}{n}}} \right) = 0.99 $

По таблице: $\frac{0.01}{\sqrt{\frac{0.6\cdot0.4}{n}}}  = 2.57 \Rightarrow n = 62$
\item
\begin{enumerate}
\item $\P(-2 < \cN(0,1) < 2) = 0.9544$, $1 - \frac{1}{4} < \P(-2\sigma<X-\mu<2\sigma) < 1$
\item $\P(8<X<12) = 0.2$, $1 - \frac{20^2}{12} < \P(-2 < X - \E(X) < 2)< 1$
\item $\P(-1 < X < 3) = \int_{-1}^{3} e^{-x} dx \stackrel{x>0}{=} 1 - e^{-3}$, $1- \frac{1}{4} < \P(-2<X-\E(X)<2) < 1$

\item[9-А.]

Если убийц нечетное число, то в живых останется только один убийца.
Если убийц чётное число, то в живых либо не останется никого, либо мирные граждане.

Следовательно, если кроме гостя в городе нечётное число убийц $u$, то шансов у гостя
никаких нет. Если гость мирный, то в живых останется в финале один из убийц,
если гость — убийца, то в финале убийц не останется.

Если кроме гостя в городе чётное число убийц $u$, и гость будет новым нечётным убийцей,
то в финале останется один убийца, и вероятность выжить для гостя равна $1/(u+1)$.

Рассмотрим случай чётного числа убийц и мирного гостя. Замечаем, что прочие мирные
лишь отдаляют по времени разборки и не влияют на вероятность выжить гостя.
Поэтому уберём остальных мирных жителей.

Чтобы гость выжил, все встречи должны быть между убийцами. Следовательно, вероятность
выжить равна:

\[
\frac{u}{u-1}\frac{u-1}{u-2} \cdot \frac{u-2}{u-3}\frac{u-3}{u-4}\cdot \ldots
\cdot \frac{2}{3}\frac{1}{2} = \frac{1}{u+1}
\]

От стратегии гостя ничего не зависит.
И вероятность выжить гостя либо 0, либо $1/(u+1)$ в зависимости от чётности числа убийц.

\item[9-Б.]
Подразумевая под точками концы гирлянды, докажем следующее утверждение.

Бросим $2n \geq 4$ точек $X_1, X_2, \ldots, X_{2n}$ случайным образом на отрезок
$[0;1]$. Пусть для $1 \leq i \leq n$ $J_i$ — это отрезок с концами $X_{2i-1}$ и $X_{2i}$.
Тогда вероятность того, что найдётся такой отрезок $J_i$, который пересекает все
другие отрезки, равна $2/3$ и не зависит от $n$.

Доказательство. Бросим $2n+1$ точек на окружность, тогда $2n$ точек образуют пары,
а оставшуюся обозначим $X$ и будем считать её и началом, и концом отрезка. Каждому
получившемуся отрезку присвоим уникальное имя. Далее, будем двигаться от точки $X$
по часовой стрелке до тех пор, пока не встретим одно и то же имя дважды, например «а».
После этого пойдём в обратную сторону, и будем идти, пока не встретим какое-нибудь
другое имя дважды, например, «б». Теперь посмотрим на получившуюся последовательность
между «б» и «а» на концах пути, читая её по часовой стрелке от «б» до «а». Нас
интересует взаимное расположение $X$, второй «а» и второй «б». Зная, что «а» стоит
после $X$, выпишем все возможные случаи, где может стоять «б»:
\begin{enumerate}
\item перед $X$
\item между $X$ и «а»
\item после «а»
\end{enumerate}
Покажем, что во втором и третьем случае отрезок «б» пересекает все остальные, а в
первом такого отрезка вообще нет. Попутно заметим, что появление каждого и случаев
равновероятно.

Действительно, если «б» стоит после $X$, и отрезок соответствующий этому имени,
не пересекает какой-нибудь другой отрезок «в», то последовательность выглядела бы
как «бвв$X$б» или «б$X$ввб», что противоречит описанному построению.
Если «б» стоит перед $X$ и отрезок «в» пересекает оба отрезка «а» и «б», то мы
снова приходим в противоречие с построением. В итоге, получаем, что искомая вероятность
равна $2/3$.
\end{enumerate}
\end{enumerate}



\subsection[2007-2008]{\hyperref[sec:kr_01_2007_2008]{2007-2008}}
\label{sec:sol_kr_01_2007_2008}

\begin{enumerate}
\item $c = 0.2$, далее $\P\left(Y>-X\right)=0.5$  и $\E\left(X\cdot Y\right)=0.1$

$\Corr(X,Y)=\frac{-0.02}{\sqrt{0.24\cdot 1.41}}$

$\E\left(Y|X>0\right)=0.25$
\item
\begin{enumerate}
\item $\E(S)=-1$, $\Var(S)=207$, $\P(Z>1.47)=1-0.9292=0.0708$
\item $p(x_{1}|0)\sim \exp\left(-\frac{1}{2}\left(\begin{array}{cc} {x_{1}-2} & {0+1} \end{array}\right) \left(\begin{array}{cc} {9} & {-4.5} \\ {-4.5} & {25}
\end{array}\right)^{-1}\left(\begin{array}{c} {x_{1}-2} \\ {0+1}
\end{array}\right)\right)$

$p(x_{1}|0)\sim \exp\left(-\frac{1}{2det}(25(x_{1}-2)^{2}+9(x_{1}-2)+9)\right)$

$p(x_{1}|0)\sim \exp\left(-\frac{1}{2\cdot 8.19}(x_{1}-1.82)^{2}\right)$

$\Var(X_{1}|X_{2}=0)=8.19$, $\E(X_{1}|X_{2}=0)=1.82$

Есть страшные люди, которые наизусть помнят, что:

$\Var(X_{1}|X_{2}=x_{2})=(1-\rho^{2})\sigma_{1}^{2}$

$\E(X_{1}|X_{2}=x_{2})=\mu_{1} + \rho\frac{\sigma_{1}}{\sigma_{2}}(x_{2}-\mu_{2})$
\end{enumerate}
\item $\P(Y>2X)=\int_{0}^{1}\int_{0}^{y/2}(x+y)dxdy=\frac{5}{24}$

$\E(X)=\int_{0}^{1}\int_{0}^{1}x(x+y)dxdy=\frac{7}{12}$

Зависимы
\item Рассмотрим $X=8-($Васин бал$)$ и $Y=($Петин бал$)-6$

$\Corr(X,Y)=-0.7$ (т.к. при линейном преобразовании может поменяться только знак корреляции)

$\Var(X)=\frac{1}{2}\left(1-\frac{1}{2}\right)$

$\Var(Y)=\frac{1}{3}\left(1-\frac{1}{3}\right)$

Интересующая нас величина - это $\P(X=1\cap Y=1)=\E(XY)=\Cov(X,Y)+\E(X)\E(Y)$

answer: $\frac{10-7\sqrt{2}}{60}\approx 0.001675$

key point: $\Cov=-\frac{7\sqrt{2}}{60}$
\item $\frac{37}{40}=0.925$
\item Частая ошибка в «а» — решение другой задачи, где проценты заменены на копейки.

Пусть $N$ — число подъемов акции.
\begin{enumerate}
\item
\begin{multline*}
\P(100\cdot 1.08^N\cdot 0.95^{64-N}>110)= \P(N \ln(1.08)+(64-N) \ln(0.95)> \ln(1.1))= \\
= \P\left(N>\frac{\ln(1.1)-64\ln(0.95)}{\ln(1.08)-\ln(0.95)}\right)
\end{multline*}
Заметим, что $N$ - биномиально распределена, примерно $N\left(64\cdot\frac{1}{2},64\cdot\frac{1}{4}\right)$

$Z=\frac{N-32}{4}$ - стандартная нормальная и $\P(Z>-1,42)=0.92$
\item $\E(N\ln(1.08)+(100-N)\ln(0.95))$

На этот раз $\E(N)=50$ и $\E(\ln(P_{100}))=1.28$
\end{enumerate}
\item $p_{break}=1-\exp(-5/7)=0.51=\int_{0}^{5}\frac{1}{7}e^{-\frac{t}{7}}dt$

$p=0.8\cdot 0.51\approx 0.4$

$\E(S)=1000p=400$, $\Var(S)=1000p(1-p)=240$

$\P(S>350)=\P(Z>-3.23)\approx 1$
\item
\begin{enumerate}
\item $\P(X^{2}>2.56\Var(X))=\P(|X-0|>1.6\sigma)\le
\frac{\Var{X}}{2.56\Var(X)}=\frac{100}{256}\approx 0.4$
\item $\P(X^{2}>2.56\Var(X))=\P(|Z|>1.6)=0.11$
\end{enumerate}
\item[9-А.] б) Искомая вероятность равна $\P(A)=f(k+1,n-k)/f(k,n-k)$, где

$f(a,b)=\int_{0}^{1}x^{a}(1-x)^{b}dx$

Проинтегрировав по частям, видим, что $f(a,b)=f(a+1,b-1)\frac{b}{a+1}$

Отсюда $f(a,b)=\frac{a!b!}{(a+b+1)!}$

Подставляем, и получаем: $\P(A)=\frac{k+1}{n+2}$

Если кто получит этот ответ более интуитивным образом — тому большой
дополнительный балл (!) — обращайтесь на \href{mailto:boris.demeshev@gmail.com}{boris.demeshev@gmail.com}

\item[9-Б.] Занумеруем детей в порядке появления на свет. Обозначим $M_{i}$ —
индикатор того, что $i$-ый ребенок — мальчик, и $F_{i}$ — индикатор того, что
$i$-ый ребенок — девочка. Конечно, $F_{i}+M_{i}=1$ и $F_{i}M_{i}=0$.
$M$, $F$ — общее число мальчиков и девочек соответственно.

Запасаемся простыми фактами:

$\E(F_{i})=\E(M_{i})=\E(F_{i}^{2})=\E(M_{i}^{2})=\frac{1}{2}$

$\E(F)=\E(M)=\frac{n}{2}$

$\Var(F_{i})=\Var(M_{i})=\frac{1}{4}$

$\Var(F)=\Var(M)=\frac{n}{4}$

$\E(F^{2})=\E(M^{2})=\Var(F)+\E(F)^{2}=\frac{n(n+1)}{4}$

$\E(FF_{i})=\frac{n+1}{4}$

Заметим, что $X_{i}=X_{i}+M_{i}F_{i}=M_{i}F$. Таким образом,

$X=MF=nF-F^{2}$

$Y_{i}=F-F_{i}-X_{i}$

$Y=(n-1)F-MF=(n-1)F-nF+F^{2}=F^{2}-F$

Далее берём математическое ожидание (легко) и дисперсию (громоздко):  $\E(X)=\E(Y)=\frac{n(n-1)}{4}$

\ldots (если кто решил до сих пор, то наверняка, он смог и дальше решить) \ldots
\end{enumerate}



\subsection[2006-2007]{\hyperref[sec:kr_01_2006_2007]{2006-2007}}
\label{sec:sol_kr_01_2006_2007}

\begin{enumerate}
\item $c=0.3$, $\P(Y>-X)=0.5$, $\E(XY^{2})=0.5$, $\E(Y|X>0)=\frac{0.1}{0.4}=0.25$
\item $\E(Y)=-1$, $\Var(Y)=207$, $\P(Y>20)=\P(Z>\frac{21}{\sqrt{207}})=\P(Z>1.46)=0.07$
\item $\P(Y>2X)=\int_{0}^{1}\int_{0}^{y/2}(x+y)dxdy=\frac{5}{24}$

$\E(X)=\int_{0}^{1}\int_{0}^{1}x(x+y)dxdy=\frac{7}{12}$
\item Используя метод множителей Лагранжа:

$L=\frac{(0.1\cdot 24)^{2}}{a}+\frac{(0.3\cdot 12)^{2}}{b}+\frac{(0.6\cdot 10)^{2}}{c}+\lambda(10-a-b-c)$

\ldots

$a=2$, $b=3$, $c=5$, можно было использовать готовую формулу
$n_{i}=\frac{w_{i}\sigma_{i}}{\sum w_{j}\sigma_{j}}$

$\Var(\bar X^{s})=14.4$
\item
\begin{enumerate}
\item $(2\theta-0.2)(1.2-3\theta)\rightarrow\max$

$\hat{\theta}=0.25$
\item $2.4-7\hat{\theta}=1$, $\hat{\theta}=0.2$
\end{enumerate}
\item $\P(\bar X>0.33)=\P\left(\frac{\bar{X}-0.3}{\sqrt{\frac{0.3\cdot
0.7}{200}}}>\frac{0.33-0.3}{\sqrt{\frac{0.3\cdot
0.7}{200}}}\right)=\P(Z>1.03)=0.15$
\item $\bar{X}=1$, $\hat{\sigma}^{2}=1$

$\P(\hat{\sigma}^{2}>3\sigma^{2})=\P\left(2\frac{\hat{\sigma}^{2}}{\sigma^{2}}>6\right)=\P(\chi_{2}^{2}>6)=0.05$
\item
\begin{enumerate}
\item $\P(X^{2}>4\Var(X))=\P(|X-0|>2\sigma)\le
\frac{\Var(X)}{4\Var(X)}=\frac{1}{4}$
\item $\P(X^{2}>4\Var(X))=\P(|Z|>2)=0.05$
\end{enumerate}
\item
\begin{enumerate}
\item $\bar X$
\item Да
\item $\Var(\bar X)=\frac{\theta^{2}}{n}$;
\item Да: несмещенность и предел дисперсии равный нулю
\end{enumerate}
\item
\begin{enumerate}
\item $\E(\bar X)=\frac{a}{2}$
$\theta=\frac{1}{\P(X_{i}<5)}=\frac{1}{5/a}=\frac{1}{5}a$

$\hat{\theta}=\frac{2}{5}\bar X$
\item $\Var(\hat{\theta}_{n})=(\frac{2}{5})^{2}\cdot\frac{a^{2}}{12n}$
\item $\lim \Var(\hat{\theta}_{n})=0$, оценка несмещённая,
следовательно, состоятельная.
\end{enumerate}
\item[11-А.] Обозначим $e_{n}$ — сколько дней осталось в~среднем ждать, если
уже набрано $n$ копеек.

Тогда:

$e_{100}=0$

$e_{99}=1$

$e_{98}=\frac{1}{100}e_{99}+\frac{99}{100}e_{100}+1=1+\frac{1}{100}$

$e_{97}=\frac{1}{100}e_{98}+\frac{1}{100}e_{99}+\frac{98}{100}e_{100}+1=(1+\frac{1}{100}))^{2}$

$e_{96}=\frac{1}{100}e_{97}+\frac{1}{100}e_{98}+\frac{1}{100}e_{99}+\frac{97}{100}e_{100}+1=(1+\frac{1}{100})^{3}$

\ldots

По~индукции легко доказать, что $e_{n}=(1+\frac{1}{100})^{99-n}$

Таким образом, $e_{0}=(1+\frac{1}{100})^{99}=2.718 \ldots$

\item[11-Б.]  $p_{0}=0.1$, $p_{1}=0.7\cdot 0.1$;

\[
p_{n}=\P(\text{в~первый день Петя познакомился с~одной девушкой})p_{n-1}+
\P(\text{в~первый день Петя познакомился с~двумя девушками})p_{n-2};
\]

Разностное уравнение: $p_{n}=0.7p_{n-1}+0.2p_{n-2}$
\end{enumerate}



\subsection[2005-2006]{\hyperref[sec:kr_01_2005_2006]{2005-2006}}
\label{sec:sol_kr_01_2005_2006}

\begin{enumerate}
\item $\P(X=0) = e^{-0.003} \approx 0.997$
\item $c=0.3$, $\P(Y > -X) = 0.4$, $\E(X\cdot Y^2) = 0.5$, $\E(Y|X>0) = 1/3$
\item $\P(X_1 +3X_2 > 20) = \P\left(\frac{X_1 +3X_2 +1}{\sqrt{207}}  > \frac{20+1}{\sqrt{207}}\right) = \P(\cN(0,1) > 1.46) = 0.0721$

$\E(X_1 +3X_2) = -1$, $\Var(X_1 +3X_2) = 9 + 9\cdot 25 +6 \cdot(-4.5) = 207$
\item $\P(Y > X) = \int_0^1 \int_x^1 (x+y) dy dx = 0.5$

$\E(X) = \int_0^1 (x+0.5)x dx = 7/12$

% f_{X| Y>X} = \int_x^1 (x+y) dy  = -1.5 x^2 + x + 0.5

% \E(X|Y>X) = \int_0^1 (-1.5 x^2 + x + 0.5)x dx  = 5/24
\item $\P(\hat{p} < 0.21) = \P\left(\frac{\hat{p} - 0.8}{\sqrt{\frac{0.8(1-0.8)}{200}}} < \frac{0.21 - 0.8}{\sqrt{\frac{0.8(1-0.8)}{200}}} \right) \approx 0$ % проверить!
\item
\item
\begin{enumerate}
\item $0.5118$
\item $7/30$
\item $-e^{-\frac{1}{20\cdot 23}} + e^{-\frac{1}{20\cdot 16}} \approx 0.13$
\end{enumerate}
\item Если $S$ — финальная стоимость акции, то $S=1000+X_1+X_2+\ldots+X_{100}$. Тогда по ЦПТ $S\sim \cN(1000,100)$ и $\P(S>1030)\approx 0.001$.
\item $\E(X) = -c \sqrt{\frac{\pi}{2}}$, $\Var(X) = c \frac{5}{4} \frac{\pi}{2} - c^2 \frac{\pi}{2}$
\item
\begin{multline*}
\P(X=k|X+Y=50) = \frac{\P(X+Y =50 | X=k)\P(X=k)}{\P(X+Y=50)} = \frac{\P(Y=50-k)\P(X=k)}{\P(X+Y=50)} = \\
\frac{\left(\frac{e^{-15} 15^{50-k}}{(50-k)!}\right) \left(\frac{e^{-5} 5^k}{k!}\right)}{\frac{e^{-(5+15)}(5+15)^{50}}{50!}} = C_{50}^k \frac{5^k \cdot 15^{50-k}}{(5+15)^{50}} = C_{50}^k \left(\frac{5}{5+15}\right)^k \left(\frac{15}{5+15}\right)^{50-k}
\end{multline*}
Получили биномиальное распределение с параметрами $p=1/4$, $n=50$.
\end{enumerate}



\subsection[2004-2005]{\hyperref[sec:kr_01_2004_2005]{2004-2005}}
\label{sec:sol_kr_01_2004_2005}


\begin{enumerate}
\item $\P\left(|X-\E(X)|>2\sqrt{\Var(X)}\right) = \P\left(\frac{|X-\E(X)|}{\sqrt{\Var(X)}} > 2\right) = 2\P(\cN(0,1) >2) \approx 0.05$
\item $\mu = -1$, $\sigma^2 = 9$
\item $C_{200}^4 \left(\frac{1}{130}\right)^4 \left(\frac{129}{130}\right)^{196}$
\item $C_{20}^5 0.52^{15}0.48^9$
\item $\P(|X| > 15) \leq \frac{16}{15^2}$
\item Подготовимся: $\E(2X-Y) = -5$, $\Var(2X-Y) = 16 + \Var(Y)$

$ \P(Z>0) = 0.9 \Rightarrow \P\left(\frac{2X-Y+5}{\sqrt{16+\Var(Y)}} > \frac{5}{\sqrt{16+\Var(Y)}}\right) \Rightarrow \frac{5}{\sqrt{16+\Var(Y)}} = 1.28  $

$ \Var(Y) = 0.74$
\item $\E(X) = \Var(X) = \lambda = 0.09$
\[
\P(| X - 0.09| > 0.18) = 1 - \P(-0.18 < X - 0.09 < 0.18)  \stackrel{X\geq0}{=} 1 - \P(X=0) = 1 - e^{-0.09}
\]
\item Пусть $X$ — случайная величина, число страховых случаев, $X \sim Bin(n=1000, p=0.05)$. $S$ — размер резерва.

Тогда условие можно записать в виде: $\P(1500X \leq S) = 0.95$
\[
\P\left(\frac{X - 50}{\sqrt{1000 \cdot 0.05 \cdot 0.95}} \leq \frac{\frac{S}{1500}-50}{\sqrt{1000 \cdot 0.05 \cdot 0.95}} \right) =0.95 \Rightarrow \frac{\frac{S}{1500}-50}{\sqrt{1000 \cdot 0.05 \cdot 0.95}} \approx 1.65 \Rightarrow S \approx 92058
\]
\item
\begin{enumerate}
\item Да
\item Нет
\item $1356$
\end{enumerate}
\item Пусть $X$ — случайная величина, число сочинённых песенок в день, когда Винни-Пуха кусает пчела, $X \sim Bin(n,p)$.

Из данной в условии выборки находим $\overline{X} = 19/18$, поскольку число наблюдений достаточно велико, $\E(X) = np = 36p = 19/18$, откуда получаем $p=19/(18\cdot36)$ и $\Var(X) = np(1-p) \approx 1$.

Нормальная аппроксимация: $X \sim \cN(19/18, 1)$.
\[
\P(X>1) = \P\left(X - \frac{19}{18} > -\frac{1}{18} \right) \approx 0.52
\]
\item
\begin{enumerate}
\item Заметим, что величину $X_t$ можно представить в виде:
\[
X_t = A_t \cdot X_{t-1} = A_t \cdot A_{t-1} \cdot X_{t-1} = \ldots = A_t \cdot A_{t-1} \cdot \ldots \cdot A_{2} \cdot X_1
\]
Тогда и предел тоже можно переписать:
\[
\lim_{n\to\infty}\frac{\ln X_n}{n} = \lim_{n\to\infty} \frac{\ln A_n + \ldots + \ln A_2 + \ln X_1}{n} \stackrel{X_1 = 1}{=} \lim_{n\to\infty} \frac{\ln A_n + \ldots + \ln A_2}{n} \stackrel{\text{ЗБЧ}}{=} \E(\ln A_1)
\]
Осталось найти математическое ожидание $\ln A_1$:
\[
\E(\ln A_1) = \int_0^{2a} \frac{1}{2a} \cdot \ln x dx = \ln(2a)-1
\]
\item Из неравенста $\ln(2a)-1>0$ получаем, что темп роста будет положительным при $a>e/2$.
\end{enumerate}
\end{enumerate}
